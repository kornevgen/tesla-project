\section{Introduction}

System functional testing of microprocessors uses many assembly
programs (\emph{test programs}). Such programs are loaded to the
memory, executed, execution process is logged and analyzed. But
modern processors testing requires a lot of test programs. Technical
way of test program generation was proposed
in~\cite{IEEEhowto:kamkin}. This way based on the microprocessor's
model. Its first stage is systematic generation abstract test
programs (\emph{test templates}). This abstract form doesn't contain
initial state of microprocessor but contain sequence of instructions
with arguments (registers) and with \emph{test situations} (behavior
of this instruction; these can be overflow, cache hits, cache
misses). The second stage is generation of initial microprocessor
state for given test template. This stage is test data generation.
Technical way from~\cite{IEEEhowto:kamkin} is useful for aimed
testing when aim is expressed by instruction sequence with specific
behavior. Initial microprocessor state includes initial values of
registers and initial contents of cache-memory. Based on this state
the third, final, stage is generation the sequence of instructions
to reach initial microprocessor state. These sequence of
instructions with test template get ready assembly program. This
paper devoted to the second stage, i.e. initial state generation.

Known researches about test data generation problem contain the
following methods of its solving:
\begin{enumerate}
\item combinatorial methods;
\item ATPG-based methods;
\item constraint-based methods.
\end{enumerate}

Combinatorial methods are useful for simple test templates (each
variable has explicit directive of its domain, each value in domain
is possess)~\cite{IEEEhowto:combinatorial}. ATPG-based methods are
useful for structural but not functional
testing~\cite{IEEEhowto:ATPG}. Constraint-based methods are the most
promising methods. Test template is translated to the set of
constraints (predicates) with variables which represented test data.
Then special solver generates values for variables to satisfy all
constraints. This paper contains constraint-based method also. IBM
uses constraint-based method in
Genesys-Pro~\cite{IEEEhowto:GenesysPro}. But it works inefficiently
on test templates from~\cite{IEEEhowto:kamkin}. Authors of another
constraint-based methods restrict on registers only and don't
consider cache-memory.
