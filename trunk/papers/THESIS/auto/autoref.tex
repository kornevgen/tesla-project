% !Mode:: "TeX:UTF-8"
\documentclass[14pt,autoref,href
%,fixint=false
,facsimile=false
]{disser}

\usepackage{cmap}
\usepackage[a4paper, nohead, includefoot, mag=1000,
            margin=2cm, top=1.5cm, bottom=1.5cm, footskip=1cm]{geometry}
%\usepackage[T2A]{fontenc}
\usepackage[utf8]{inputenx}
\usepackage[english,russian]{babel}
\usepackage{tabularx}
%\usepackage{epstopdf}
\usepackage{amsthm}
\usepackage{xspace}
\usepackage{ifthen}


\renewcommand{\baselinestretch}{1.4}

% Поддержка нескольких списков литературы в одном документе
%\usepackage{multibib}
% Создание команд для цитирования собственных работ диссертанта
% в отдельном разделе. В данном случае ссылка будет иметь вид \citemy{...}.
%\newcites{my}{Список публикаций}

\newtheorem{utv}{Утверждение}
\newtheorem{theorem}{Теорема}
\newtheorem{assumption}{Предположение}
%\newtheorem{heuristics}{Эвристика}
\newtheorem{definition}{Определение}{\bfseries}{\itshape}
%% Путь к файлам с иллюстрациями
%\graphicspath{{fig/}}

% Теперь "русифицируем" окружение enumerate:
\makeatletter
\def\labelenumi{\theenumi)}      % чтобы после номера шла скобка;
\def\theenumii{\@asbuk\c@enumii}   % чтобы на втором уровне шли русские,
\def\labelenumii{\theenumii)}    % а не латинские буквы
\def\p@enumii{\theenumi}         % а это для \ref
\def\labelenumiii{{\bf--}}       % а на третьем уровне пусть будут лишь тире,
\let\theenumiii\relax            % и отдельных ссылок на него не будет
\def\p@enumiii{\theenumi\theenumii}
\makeatother

\usepackage{ifpdf}

\ifpdf
% we are running pdflatex, so convert .eps files to .pdf
% run pdflatex with --shell-escape and thesis.aux
\usepackage[pdftex]{graphicx}
\usepackage{epstopdf}
\else
% we are running LaTeX, not pdflatex
\usepackage{graphicx}
\fi

% Включение файла с общим текстом диссертации и автореферата
% (текст титульного листа и характеристика работы).
\section{���������� ����������� ��� ��������� �������� ������}\label{common_algorithm}

� ������ ������ ������������ �������� ���������� �������� ������
(�.�. ��������� �������� ���������, ����� ���-������, ������
���������� �������, ����� ��� � ��.) ��� �������� ��������. ��������
����������� � ������������ ��������� ����� ���������� �����������
��� ������ ������� � ��������� ����������� �����
�����~\cite{ConstrProp}. ����� �������, �� ��������� ������� �����
��������� ������� �����������, ����� ��� ����� ���������, �
���������� ���� ����� �������� �������� ������.

� ������������ � ��������� ������������ ���������������, �������
��������� ������� � ���������� �������� ��� ���������� � �������, �
������� ����������� ����� ��������� ����������:
\begin{itemize}
\item ������� ����� TLB ��� ������ ���������� ��������� �������
\item ��������� �������� ���������, ��������������� � �������
\item ��������� ��������� �������
\item ����������� ������ ��� ������ ���������� ��������� �������
\item ���� ����� TLB, ��������������� � �������� ��������� (� ������
���� <<r>>, <<vpn/2>>, <<mask>>, <<g>>, <<asid>>, �����)
\item ���� ���������� ������� ��� ������ ���������� ���������
������� (� ������, ���, ���, ������ � ������ ���-������)
\end{itemize}

�������� �� ���������� ���������� ��������������� �������, � ���� ��
�����, ������� ������������� ������������� � �������� ���������. ���
��������� ����������� ��������� ������ ������� �����������.

�������� ����� ����������� � ���� ��������� ������������������
�����:
\begin{enumerate}
\item\label{alg_indextlb} ���������� �������� ����� TLB ��� ������ ���������� ���������
������� �� ������ �������� �������� � ������ TLB (TLB-�������� ���
TLB-���������)
\item\label{alg_testsit} ��������� ����������� �� ��������� �������� ���������, ������
�� �������� �������� ����������, �� ���������� � �������
\item\label{alg_virtual} ��������� �����������-����������� ����������� ������� ��
��������� ��������� ��� ����������, ���������� � �������
\item\label{alg_virtonerow} ��������� ����������� �� ����������� ������, ��������������� �
�������� ������� ����� ������ TLB
\item\label{alg_tlbconsist} ��������� ����������� �� ���� ��������������� � �������� �������
����� TLB, ����������� �������� ��������������� TLB (������
����������� ����� ����� ��������������� �� ����� ����� ������ TLB)
\item\label{alg_physonerow} ��������� ����������� �� ���� ���������� ������� ���
����������, ������������ � ���� ������ TLB
\item\label{alg_cache} ��������� �����������, ������ �� �������� �������� �
���-������ (���-�������� � ���-���������)
\item\label{alg_ozu} ��������� ����������� �� ����������� ������ � ��������
���������, ����������� ������ � ��� (���������� ��������� ��������
���������� ��� ���������� ���������� �������)
\end{enumerate}

������������� ��������� �������� ��, ��� � ���� �����������
���������� <<������������� �����>> ���������� � ������� �� ����, ���
���� �����, ��������, Genesys-Pro: ������ ����, ����� �� ������ ����
(�.�. ��� ������ ��������� ����������) ��������� �������� �������,
���������, �������� (� �������� ��� ���������� ��������),
������������� ����� ���� �������� �� ���������� �������, ���������,
�������� ������ ����������.

���~\ref{alg_indextlb}. ��� ���� -- ��������� ������ ����� TLB, �
������� ���������� ���������� ������ � ������� � �������� �������. �
������ ����� ���� ����� ���������� � ��������� �����������.
����������� ������������ �� ������ �������� �������� � ������ TLB,
��������� � �������. ��������� ����� ����� ���� ��� ���� ��� ����,
��������� ����������� �� ���� ���� ��������� � ����������
����������� �� ����~\ref{alg_cache}. �� ���� ����� ���������� �
�������~\ref{eqs} ������ ������. ���� ���������� � ����������
���������� ����������� ������ ����� �� �������� � ���������� �
���������� ������� ����������� �� ����������� ������ � ��������,
����� �������� ������� � ����� ������ ������� ����� TLB.

��������� ���~\ref{alg_testsit}. ��� ���� -- �������� ����������� ��
��������� �������� ���������, ������ �� �������� ��������
����������, �� ���������� � ������� (��������, ��������������
������������, ������� �� ����). �� ���� ���� ������� ���������� �
�������� ��������������� �������� �������� � ������������� ��� �
����� ����������� ���, ��� ��� �������� ��� ������������ ��������.

���~\ref{alg_virtual}. � ���������� ����� ���� ������ ����������
�����������, ����������� ����������-����������� ������ ���������� �
����������-�������� ���������. ����������� �������� �� ������
�������� �������� �������� ����������, ��� ����������� �����
��������� ����� �� ���������� ��������� AddressTranslation. ��������
����������� ������ ������������ �� ���� ����� �������� ��������,
����������� ����� �� ���������� ����������, � �����������������
��������, ����������� ������ ���������� ����������.

��������� ���~\ref{alg_virtonerow}. � ������ ����� ���� ��� ������
��������������� � �������� ������� ������ TLB ���������� ���
����������� ������ ����������, ���������� � ���� �������.
����������� ������ ���� ����� ���������� �������� ����������
����������:
\begin{enumerate}
\item ����, ��������������� ���� <<r>> ������ TLB, �����������
������� ���������
\item ����, ��������������� ���� <<vpn/2>> ������ TLB, �����������
����� <<mask>> ������ TLB, ���������
\end{enumerate}

���~\ref{alg_tlbconsist} ������� �������� ����������� �� ����
��������������� � ������� ����� TLB � ����� ������� ��������
��������������� TLB (� ������, ��� ������ ����������� ����� �����
��������������� �� ����� ����� ������ TLB): � ����� ���� ����������,
���������� � ������� �������� TLB, ���� ����������� ���� �����
<<r>>, ���� ����������� ���� ����� <<vpn/2>>, ����������� ������
<<mask>> ����� TLB.

�� ��������� ����~\ref{alg_physonerow} ���������� ���������
����������� �� ���� ���������� ������� ����������.
\begin{enumerate}
\item ������������ ���� ���������� ������� -- �������� �������
���������� ������� ��� �������� ������ ����������� �������
\item ������������ ������� � ������� ���-������ ���������� ������� --
�������� ������� ���������� ������� ��� �������� ������ �����������
�������
\item �������� ���������� ������� ����������, ������������ � ����
������ TLB: ���� ���������� ������� ��������� ����� � ������ �����,
����� ��������� ���� �������� ���������� �������� �����������
�������.
\end{enumerate}

���~\ref{alg_cache} ������� �������� ����������� �� ���� ����������
�������, ������ �� �������� �������� � ���-������. �� ���� �����
��������, � ����� ���� ���������� ��� ���������� ��������� �������.
�������� �������� ����������, ������������ � ���� ���, � ��������
��� ��� ����������� ���, ��� ��� ����� ������� � �������~\ref{eqs}.

�������������� ���~\ref{alg_ozu} ������ ����� �������� �����������
�� ����������� ������ � �������� ���������, ������ �� ����������
�������� ����������� ������: ��������, ����������� �� ����������
�������, ���������, ���� ����� ����� �������� �� ����� ������ ��
���� ������; ���� ������ ����, ������� ��������� ����� ������
���������� ��������. ����� ����� (<<$LOAD~x, a$>> -- ����� ����������
������ �� ������, ��� $�$ -- ��������� ��������, $�$ -- ����������
�����; <<$STORE~x, a$>> -- ����� ���������� ������ � ������, ��� $x$ --
������������ ��������, $�$ -- ���������� �����):

\parbox{\textwidth}{
��� ������ $LOAD~x, a_1$ �� �������

\hspace{0.5cm}��� ������ ���������� $LOAD~y, a_2$ �� �������

\hspace{0.5cm}\hspace{0.5cm}����� $a_3, a_4, ..., a_n$ -- ������ � $STORE$ ����� ����:

\hspace{0.5cm}\hspace{0.5cm}�������� ��������� $a_1 \in \{a_2\}\setminus\{a_3,a_4,...a_n\} \Rightarrow x = y$

\hspace{0.5cm}��� ������ ���������� $STORE~y, a_2$ �� �������

\hspace{0.5cm}\hspace{0.5cm}����� $a_3, a_4, ..., a_n$ -- ������ � $STORE$ ����� ����:

\hspace{0.5cm}\hspace{0.5cm}�������� ��������� $a_1 \in \{a_2\}\setminus\{a_3,a_4,...a_n\} \Rightarrow x = y$
}


% !Mode:: "TeX:UTF-8"
\newcommand{\LcurrentBody}{
Пусть $L_0$ -- множество ключей, расположенных в некотором регионе таблицы перед исполнением инструкций тестового шаблона, $x_1$, $x_2$, ..., $x_n$ -- последовательность ключей того же региона, по которым происходят обращения с промахами в том же порядке, что и в тестовом шаблоне, $x'_1, x'_2, ..., x'_n$ -- последовательность ключей, вытесняемых обращениями к $x_1$, $x_2$, ..., $x_n$. Тогда $L$ -- выражение для множества ключей таблицы того же региона после этих обращений имеет вид:
$$L \equiv L_0 \setminus \bigcup_{i=1}^n \{x'_i\} \cup \bigcup_{i=1}^n (
\{x_i\} \setminus \cup_{j~=~i+1}^n \{x'_j\}).$$
}

\newcommand{\HitMissEquations}{
Пусть $L_0$ -- множество ключей, расположенных в таблице в некотором регионе перед исполнением инструкций тестового шаблона, $x_1, x_2, ..., x_n$ -- последовательность ключей того же региона, по которым происходят обращения с промахами в том же порядке, что и в
тестовом шаблоне, $x'_1$, $x'_2$, ..., $x'_n$ -- последовательность ключей, вытесняемых обращениями к $x_1$, $x_2$, ..., $x_n$. Тогда для обращения после этих по ключу $x$ в тот же регион справедливы следующие уравнения:
\begin{itemize}
\item для обращения с попаданием:
$$
\left[
   \begin{array}{l}
    x \in L_0 \wedge x \notin \{x'_1, x'_2, ..., x'_n\} \\
    x = x_1 \wedge x \notin \{x'_2, ..., x'_n\} \\
    x = x_2 \wedge x \notin \{x'_3, ..., x'_n\} \\
    ...\\
    x = x_{n-1} \wedge x \notin \{x'_n\} \\
    x = x_n \\
   \end{array}
  \right.
$$

\item для обращения с промахом ($x'$ --- вытесняемый ключ):
$$
\left\{
   \begin{array}{l}

  \left[
   \begin{array}{l}
    x \notin L_0 \wedge x \notin \{x_1, x_2, ..., x_n\} \\
    x = x'_1 \wedge x \notin \{x_2, ..., x_n\} \\
    x = x'_2 \wedge x \notin \{x_3, ..., x_n\} \\
    ...\\
    x = x'_{n-1} \wedge x \notin \{x_n\} \\
    x = x'_n \\
   \end{array}
  \right. \\

  { }\\

  \left[
   \begin{array}{l}
    x' \in L_0 \wedge x \notin \{x'_1, x'_2, ..., x'_n\} \\
    x' = x_1 \wedge x \notin \{x'_2, ..., x'_n\} \\
    x' = x_2 \wedge x \notin \{x'_3, ..., x'_n\} \\
    ...\\
    x' = x_{n-1} \wedge x \notin \{x'_n\} \\
    x' = x_n \\
   \end{array}
  \right. \\

  { }\\

  displaced(x')\\

%  { }\\
%
%  R(x) = R(x')\\
%
  \end{array}
\right.
$$

\end{itemize}
}

\newcommand{\CorrectnessMirror}{
%Если тестовый шаблон является совместным (т.е. для него существует хотя бы одна тестовая программа), то тестовая программа (инициализация плюс инструкции тестового шаблона), построенная по предлагаемому методу, соответствует тестовому шаблону.
Если система ограничений, построенная для последовательности обращений к таблице $(S_1, k_1, R_1)$, $(S_2, k_2, R_2)$, ..., $(S_n, k_n, R_n)$ c дополнительным предикатом $P(k_1,$ $k_2, ..., k_n, R_1, R_2, ..., R_n)$, является совместной, то ее решение $t_1, t_2, ..., t_m$, $r_1$, $r_2$, ..., $r_m$, $k_1, k_2,$ ..., $k_n$, $R_1, R_2, ..., R_n$, удовлетворяет последовательности $S_1, ..., S_n$ и $P$ при любом начальном состоянии таблицы.
}

\newcommand{\FullnessMirror}{
Фиксируем некоторое начальное состояние $L$ таблицы со стратегией вытеснения не \texttt{none}. Если при нем для последовательности обращений к таблице $(S_1, k_1, R_1)$, $(S_2, k_2, R_2)$, ..., $(S_n, k_n, R_n)$ c дополнительным предикатом $P(k_1$, $k_2, ..., k_n, R_1$, $R_2, ..., R_n)$ существуют значения переменных $k_1,$ $k_2,$ ..., $k_n$ и $R_1$, $R_2, ..., R_n$, которые удовлетворяют последовательности $S_1, S_2, ..., S_n$ и $P$, то система ограничений, построенная согласно  алгоритму генерации ограничений на ключи обращений для таблицы со стратегией вытеснения не none,
%из раздела~\ref{sec:constraints_generation_section}
будет совместной, если стратегия вытеснения является <<существенно вытесняющей>>.
}

\newcommand{\FullnessMirrorNone}{
Фиксируем некоторое начальное состояние $L$ таблицы со стратегией вытеснения \texttt{none}. Если при нем для последовательности обращений к таблице $(S_1, k_1, R_1)$, $(S_2, k_2, R_2)$, ..., $(S_n, k_n, R_n)$ c дополнительным предикатом $P(k_1$, $k_2, ..., k_n, R_1$, $R_2, ..., R_n)$ существуют значения переменных $k_1, k_2, ..., k_n$ и $R_1$, $R_2, ..., R_n$, которые удовлетворяют последовательности $S_1, S_2, ..., S_n$ и $P$, то система ограничений, построенная согласно  алгоритму генерации ограничений на ключи обращений для таблицы со стратегией вытеснения none,
%из раздела~\ref{sec:constraints_generation_section}
 будет совместной.
}

\newcommand{\PseudoLRUEssential}{
Стратегия вытеснения \PseudoLRU является существенно вытесняющей.
}

\newcommand{\UpperBoundLRUMirror}{
Фиксируем некоторое начальное состояние $L$ таблицы со стратегией вытеснения \LRU. Если при нем для последовательности обращений к таблице $(S_1, k_1, R_1)$, $(S_2, k_2, R_2)$, ..., $(S_n, k_n, R_n)$ c дополнительным предикатом $P(k_1$, $k_2, ..., k_n, R_1$, $R_2, ..., R_n)$ существуют значения переменных $k_1, k_2, ..., k_n$ и $R_1$, $R_2, ..., R_n$, которые удовлетворяют последовательности $S_1, S_2, ..., S_n$ и $P$, то система ограничений, построенная согласно  алгоритму генерации ограничений на ключи обращений будет совместной для любого $m > n\cdot w + M$, где $M$ -- количество элементов последовательности $S_1, S_2, ..., S_n$, равных miss.}

\newcommand{\PseudoLRUInvariant}{
Пусть ($\alpha_1~\alpha_2~\dots~\alpha_W$)
--- \PseudoLRU-ветвь некоторой позиции $i$. Тогда изменение этой
ветви согласно стратегии вытеснения \PseudoLRU определяется только
относительной позицией (относительно $i$) и происходит следующим
образом при обращении к ключу с (абсолютной) позицией $j$: если
$\pi^i_j \in [\frac{w}{2^k},~\frac{w}{2^{k-1}})$ для некоторого
$k=1,2,\dots,W$, то происходит изменение $\alpha_1 := 0,~\alpha_2 :=
0,~\dots,~ \alpha_{k-1} := 0,~\alpha_k := 1$; если $\pi^i_j = 0$, то
происходит изменение $\alpha_1 := 0,~\alpha_2 := 0,~\dots,~\alpha_W
:= 0$; вытеснение ключа на позиции $i$ происходит в том случае, когда
$\alpha_1 = 1~\wedge~\alpha_2 = 1~\wedge~\dots~\wedge~\alpha_W = 1$.
}

%\newcommand{\DiapazonLRU}{
%Решение системы (ключ $x'$)
%$$
%\left\{
%   \begin{array}{l}
%    x' = y \\
%    R(y) \cap (L \setminus \{x_1, x_2, ..., x_n\} ) = \{y\}\\
%   \end{array}
%  \right.
%$$
%где последовательность ключей $y, x_1, x_2, ..., x_n$ -- диапазон
%вытеснения, а $L$ -- множество ключей в таблице перед концом
%диапазона, является вытесняемым ключом для стратегии вытеснения \LRU
%согласно определению на перестановках.
%}
%
%\newcommand{\DiapazonFIFO}{
%Решение системы (ключ $y'$)
%$$
%\left\{
%   \begin{array}{l}
%    y' = y \\
%    R(y) \cap (L \setminus \{y_1, y_2, ..., y_n\} ) = \{y\}\\
%   \end{array}
%  \right.
%$$
%где последовательность ключей $y, y_1, y_2, ..., y_n$ -- диапазон
%вытеснения, является вытесняемым ключом для стратегии вытеснения
%\FIFO согласно определению на перестановках.
%}
%
%\newcommand{\MaxUpperBoundLRU}{
%$$0 \leqslant k \leqslant n \cdot w_1$$
%$$0 \leqslant h \leqslant n \cdot (w_1 + w_2 + 2)$$
%где $w_1$ -- ассоциативность кэш-памяти первого уровня, $w_2$ --
%ассоциативность кэш-памяти второго уровня, $n$ -- количество
%инструкций тестового шаблона.
%}

\newcommand{\LRUusefuls}{
Пусть $(t_1, r_1)$, $(t_2,r_2),$ ..., $(t_m,r_m)$ -- ключи и регионы инициализирующих обращений, $(k_i, R_i)$ -- ключ и регион обращения, для которого описывается вытеснение (будем его называть <<вытесняемым>>), причем $(k_i||R_i) \in \{(t_1||r_1)$, ..., $(t_m||r_m)$, $(k_1||R_1)$, ..., $(k_{i-1}||R_{i-1})\}$ и $\{(t_1||r_1), ...,(t_m||r_m)\}$ --- все разные. Тогда $k_i$ не вытеснен из региона $R_i$ согласно определению \LRU на перестановках тогда и только тогда, когда $$\sum\limits_{j=1}^{m+n} [u_{k_i,R_i}(s_j)] < w$$
где последовательность $s \equiv \langle (t_1||r_1), ..., (t_m||r_m), (k_1||R_1), ..., (k_n||R_n)\rangle$,\\$R(s_i)$~---~вторая компонента $s_i$ (регион), а формула полезного обращения такая:
$$u_{k_i,R_i}(s_j) \equiv ((k_i||R_i) \notin \{s_j, ..., s_{m+n}\} \wedge
R_i = R(s_j) \wedge s_j \notin\{s_{j+1},..., s_{m+n}\})$$
}

\newcommand{\PLRUusefuls}{
Пусть $(t_1, r_1)$, $(t_2,r_2),$ ..., $(t_m,r_m)$ -- ключи и регионы инициализирующих обращений, а $(k_i, R_i)$ -- ключ и регион обращения, для которого описывается вытеснение (будем его называть <<вытесняемым>>), причем $(k_i||R_i) \in \{(t_1||r_1)$, ..., $(t_m||r_m)$, $(k_1||R_1)$, ..., $(k_{i-1}||R_{i-1})\}$ и $\{(t_1||r_1), ...,(t_m||r_m)\}$ --- все разные. Тогда $k$ не вытеснен из региона $R$ согласно трактовке \PseudoLRU в терминах ветвей бинарного дерева тогда и только тогда, когда $$\sum\limits_{j=1}^{m+n} [u_{k_i,R_i, \pi'_i}(s_j)] < W$$
где последовательность $s \equiv \langle (t_1||r_1), ..., (t_m||r_m), (k_1||R_1), ..., (k_n||R_n)\rangle$,\\$R(s_i)$ --- вторая компонента $s_i$ (регион), а формула полезного обращения такая:
$$u_{k_i,R_i,\pi'_i} (s_j) \equiv \left\{\begin{array}{l}
(k_i||R_i) \notin \{s_j, s_{j+1}, ..., s_{m+n}\}\\
R_i = R(s_j)\\
(\pi'_i||R_i) \in \{(\pi_j||R_j), (\pi_{j+1}||R_{j+1}), ..., (\pi_{m+n}||R_{m+n})\}\\
\bigwedge\limits_{k = j+1}^{m+n} (R_j = R_k ~\wedge~(\pi'_i||R_i) \notin \{(\pi_j||R_j), (\pi_{j+1}||R_{j+1}), ..., (\pi_k||R_k)\}\\
\hspace{2cm} \rightarrow P(\pi_j \oplus \pi'_i, \pi_k \oplus \pi'_i))\\
\end{array}\right.
$$
$$P(x, y) \equiv (y > x~~\wedge~~x \oplus y > x)$$

$\pi'_i$ определено следующим образом: если $S_i$ = hit, то
$$
\begin{array}{l}
\texttt{(ite~} ((k_i||R_i) = (k_{i-1}||R_{i-1})) ~~ (\pi'_i = \pi_{i-1})\\
\texttt{(ite~} ((k_i||R_i) = (k_{i-2}||R_{i-2})) ~~ (\pi'_i = \pi_{i-2})\\
... ~(\pi'_i = 0)\texttt{)))...)}\\
\end{array}
$$

если $S_i = $ miss, то ( $\{\pi_i, ..., \pi_j\}_m$ обозначает подмножество множества позиций от $i$'го до $j$'го обращения из неуспешных обращений):
$$
\begin{array}{l}
\texttt{(ite~} ((k_i||R_i) = (k_{i-1}||R_{i-1})) ~~ (\pi'_i = \pi_{i-1})\\
\texttt{(ite~} ((k_i||R_i) = (k_{i-2}||R_{i-2})) ~~ (\pi'_i = \pi_{i-2} \wedge \pi'_i \in \{\pi_{i-1}\}_m)\\
\texttt{(ite~} ((k_i||R_i) = (k_{i-3}||R_{i-3})) ~~ (\pi'_i = \pi_{i-3} \wedge \pi'_i \in \{\pi_{i-1}, \pi_{i-2}\}_m)\\
... ~(\pi'_i = 0)\texttt{)))...)}\\
\end{array}
$$
} %eps/theorems}
% !Mode:: "TeX:UTF-8"
\newcommand{\ccite}[1]{%
\ifthenelse{\isundefined{\nocites}}{\cite{#1}}{}%
}

\newcommand{\Actuality}{%
Современные микропроцессоры --- сложные многокомпонентные системы. Размеры современных микропроцессоров оцениваются как $10^7-10^8$ вентилей\ccite{HennesyPatterson}. Естественно при разработке таких сложных систем в проекты микропроцессоров вносятся ошибки, порой довольно критичные\ccite{IntelValidation}. Поэтому для обнаружения этих ошибок в цикл разработки микропроцессора в обязательном порядке входят этапы функциональной верификации.

Чем позднее будут обнаружены ошибки в микропроцессорах, тем дороже обойдётся исправление ошибок: сделать это в готовой микросхеме, тем более выпущенной на рынок, практически невозможно. Тем актуальнее становятся методы обнаружения ошибок на ранних этапах разработки микропроцессоров. Цикл разработки предполагает подготовку микропроцессоров в виде исполнимых программных моделей на языках Verilog или VHDL\ccite{VHDL}. Это делает возможным проведение функциональной верификации на таких моделях (т.е. до производства самих микропроцессоров) и актуальным исследование методов такой верификации. Целью функциональной верификации программных моделей микропроцессоров является обнаружение ошибок реализации функциональности в программных моделях микропроцессоров.

Выделяют следующие виды функциональной верификации: экспертизу, имитационное тестирование и формальную верификацию\ccite{KamkinPopular}. Экспертиза предполагает анализ текстов моделей экспертами с целью оценки их корректности и обнаружения ошибок. Этот вид функциональной верификации эффективно применяется на ранних стадиях разработки. Однако ввиду наличия человеческого фактора после экспертизы ошибки в микропроцессоре всё же остаются. Методы формальной верификации позволяют дать исчерпывающий ответ на вопрос о корректности отдельных модулей и всего микропроцессора. Однако трудоемкость формальной верификации чрезвычайно велика. Например, при разработке Intel Pentium 4 были формально верифицированы модуль работы с плавающей точкой (FPU), модуль декодирования инструкции и логика внеочередного выполнения (out-of-order), было найдено порядка 20 новых ошибок, однако трудоемкость этого проекта составила порядка 60 человеко-лет\ccite{IntelValidation}.

Имитационное тестирование позволяет ценой меньших усилий обнаружить значительную часть ошибок, в том числе критичных ошибок. Имитационное тестирование проводят для отдельных модулей (тогда оно называется \emph{модульным тестированием}) и для всего микропроцессора в целом (тогда оно называется \emph{системным тестированием})\ccite{EDAbook}. Модули тестируются подачей на их входы специальных сигналов (\emph{модульных тестов}) со снятием выходных сигналов и последующим анализом выходных сигналов. Входом при системном тестировании являются программы на машинном языке (\emph{тестовые программы}). Проведение модульного тестирования требует кроме подготовки самих входных данных еще и подготовку тестирующей установки (testbench), выделение тестируемого модуля из всего проекта микропроцессора и т.п. Системное тестирование избавлено от этой необходимости. Поскольку размер и сложность отдельного модуля всегда меньше размера и сложности микропроцессора в целом, потенциально качество модульного тестирования может быть выше, чем системного. Однако для достижения высокого качества тестирования как число модульных тестов, так и совокупная трудоемкость их изготовления, получаются очень большими. Это вынуждает часть проверок проводить на модульном уровне, а другую часть на системном. Невысокая стоимость подготовки и проведения системного тестирования определила его наибольшую востребованность среди других методов функциональной верификации. Практически все разработчики микропроцессоров проводят системное тестирование.

Ключевым вопросом, определяющим качество тестирования, является вопрос выбора тестовых программ. Поскольку современные микропроцессоры обладают множеством инструкций (порядка сотен), длины конвейеров имеют порядок десятка стадий, количество различных состояний и ситуаций, в которых надо протестировать микропроцессор, измеряется десятками тысяч. Поэтому для тщательного системного тестирования нужно подобное же и количество тестовых программ. Это определяет актуальность задачи автоматического построения тестовых программ для системного тестирования.

Сложность микропроцессоров растет (увеличивается количество функциональных требований, количество ситуаций, в которых поведение микропроцессора должно обладать заданной спецификой). Это требует тестовых программ для проверки функциональных требований, которые не проверяются имеющимися тестовыми программами, и делает актуальными дальнейшие исследования в области построения тестов.

%%%%! подредактировать про "новое качество" - это непонятно.

К числу наиболее сложных механизмов современных процессоров (поэтому наиболее подверженных ошибкам), использующих конвейеры и многоуровневые буферы типа кэш-памяти, относится механизм доступа к памяти. Поэтому актуальной является задача построения тестовых программ для проверки подсистем управления (механизмами) памяти микропроцессоров.

%%%% тем более, что зачастую нет способов прямого создания ситуаций?

%%% нацеленные методы (итерация-фильтрация, прямые конструкторы, random expansion, csp) - систематичные
%%% ненацеленные методы не тестируют тщательно или не находят ошибки, если микропроцессор не сырой

%%% в обзоре про MMU добавить классификацию ситуаций, которые надо тестировать

%%% цели:
%%% 1) понять, какие тесты "хорошие" (определение)
%%% 2) проанализировать методы их получения
%%% 3) предложить улучшения с целью получения более качественных тестов

%%% NB: ситуации не обязательно задавать шаблонами

%%% в приложение поместить примеры описаний MIPS'овских инструкций в xml ?


%А) микропроцессоры сложные -> в них есть ошибки
%Современные микропроцессоры --- это сложные системы, поэтому вероятность появления ошибки как при проектировании микропроцессора, так и при его производстве становится всё выше. При этом <<цена ошибки>> в готовом микропроцессоре велика (как минимум, это означает перевыпуск микропроцессора заново). Поэтому актуально развитие методов верификации микропроцессоров.

%%В) основная доля ошибок на этапе разработки моделей (design'а)
%Современные технологии проектирования микропроцессоров представляют собой средства разработки \emph{модели (design) на специальных языках} типа VHDL или Verilog~\cite{VerilogDesign}. Эти технологии позволяют в конечном итоге построить так называемые <<синтезируемые модели>>, из которых автоматически получаются фотошаблоны, необходимые для производства. Основная доля ошибок появляется именно на этапе разработки моделей (design), поэтому основные усилия по их выявлению или даже предотвращению их появления, также приходятся на фазу разработки моделей. Поэтому данная работа также нацелена на выявление ошибок в моделях микропроцессоров.

%%Г) модульное и системное тестирование ->
%% интересные ситуации нельзя создать инструкциями
%Тестирование на модели бывает \emph{модульным} (unit-level verification) и \emph{системным} (core-level verification, full-chip level verification)~\cite{UnitCoreLevel}. Модульное тестирование модели микропроцессора предполагает генерацию тестовых воздействий на входы отдельных модулей, блоков, микропроцессора, описанных на одном из языков типа VHDL, Verilog, и проверку выходов таких блоков. В рамках системного тестирования проверяется работа всего микропроцессора в целом --- тестом здесь является некоторая тестовая программа (программа на машинном языке), которая загружается в память и выполняется микропроцессором (речь все время идет о некоторой программной модели микропроцессора). Поскольку размер и сложность отдельного блока всегда меньше, размера и сложности микропроцессора в целом, потенциально качество модульного тестирования может быть выше, чем системного. Однако для достижения высокого качества тестирования как число модульных тестов, так и совокупная трудоемкость их изготовления, являются очень большими. Это вынуждает часть проверок проводить на модульном уровне, а другую часть на системном.

%Сложность микропроцессора определяет количество системных тестов. Если выделить различные аспекты функционирования микропроцессора (конвейер, буферы подсистемы управления памяти), то особое функционирование возникает при различных комбинациях этих аспектов. Это означает, что количество тестов должно быть не меньше произведения количества разных аспектов. Количество инструкций измеряется сотнями, а цепочек инструкций, соответственно, порядками сотен, плюс если учесть возможные аспекты в конвейере, в кэш-памяти, количество тестов получается очень большим. Для избежания проблемы такого <<взрывного>> характера количества тестов, их объединяют в классы эквивалентности --- \emph{тестовые ситуации}.

%При этом есть проблема покрытия всех потенциально интересных тестовых ситуаций. Нет никаких прямых способов создать многие из таких ситуаций нет. Например, интересно, как происходит доступ в память, когда соответствующий адрес имеется в кэш-памяти или не имеется. Или еще более тонкий анализ --- адрес имеется/или не имеется в кэш-памяти второго уровня. Среди инструкций процессора нет таких, которые были бы предназначены специально для создания таких ситуаций. Эти ситуации создаются \emph{динамически} в ходе выполнения программ.


%%Д) схема системного тестирования, показать здесь смежные вопросы
%% (вопросы построения оракула, покрытия и др.)
%Рассмотрим традиционную схему системного тестирования, известные подходы к автоматизации построения тестов и выявим проблемы, которые мешают строить более эффективные тесты.

%Микропроцессор рассматривается как черный (или серый) ящик. Входными тестовыми данными является некоторая программа, которая загружается в память. Результатом прогона теста является либо финальное состояние памяти (возможно, включая состояние регистров) или (в случае <<серого ящика>>) трасса изменения значений ячеек памяти или регистров.
%В этой общей схеме тестирования пока не упомянуты:
%\begin{itemize}
%	\item	генератор тестов (или набор уже готовых тестов);
%	\item	подсистема проверки корректности полученного результата --- тестового оракула, или арбитра;
%	\item	перечень <<интересных>> ситуаций, которые надо воспроизвести в ходе выполнения тестов;
%	\item	некая система мониторинга, которая фиксирует прохождение <<интересных>> ситуаций --- оценивает полноту покрытия.
%\end{itemize}
%
%Тестовый оракул, или арбитр, строится по схеме с использованием <<эталонной>> модели (simulation-based verification)~\cite{SimulationBased}. Каждая тестовая программа выполняется на двух моделях --- на тестируемой (design) и на <<эталонной>>. Потом состояния памяти или трассы изменения состояния памяти для тестируемой и эталонной моделей сравниваются. Если оракул признает, что трассы не эквивалентны, это свидетельствует о наличии ошибки в тестируемой системе (или эталонной, но это происходит реже). Как правило, эталонная модель пишется на одном из языков программирования (например, Си или Си++) и не загромождается деталями.  На этом основании считается, что такая модель существенно проще тестируемой, в ней с меньшей вероятностью встречаются ошибки, именно поэтому к ней можно относиться как к <<эталонной>>.
%
%% критика этого подхода: он не позволяет проверить модули, работающие за счет внешних воздействий - For example, fast interrupt request (FIQ), interrupt request (IRQ), data abort exception (Dabort) and prefetch abort exception (Pabort) of ARM7. Это пишут в статье "Automatic Verification of External Interrupt Behaviors for Microprocessor Design", авторы Fu-Ching Yang, Wen-Kai Huang, Ing-Jer Huang.
%
%
%Методы автоматической генерации тестов делят на псевдослучайные/комбинаторные (pseudo-random) и целенаправленные (что не отменяет возможности использования уже готовых тестов)~\cite{HoPhD}. В случае псевдослучайной генерации инструкции, их порядок и аргументы выбираются случайным образом или перебираются некоторым комбинаторным способом. Целенаправленная генерация начинается с задания некоторого шаблона тестовой программы, который определяет набор инструкций, их последовательность и аргументы. В рамках целенаправленной генерации порядок инструкций и их аргументы должны быть подобраны таким образом, чтобы каждый новый тест покрывал новые, еще не покрытые тестовые ситуации. Целенаправленную генерацию можно реализовать как выполнение массовой генерации комбинаторных тестов с последующей фильтрацией, с тем чтобы оставлять только те тесты, которые дают дополнительное покрытие. Однако уже для достаточно коротких шаблонов (длиной 3-4 инструкции) перебор становится слишком большим.
%
%Целенаправленная генерация тестов дает по тесту на каждую ситуацию. Набор тестов, которые покрывают все ситуации, называют нацеленными тестами (нацеленными на эти ситуации). Набор ситуаций конечен, следовательно и набор нацеленных тестов конечен. Вопрос
%%(это и есть основная тема исследования)
%, как систематическим образом строить тестовые программы, чтобы в совокупности они воспроизвели все заданные <<интересные>> ситуации.
%
%
%Перечень (конечный) <<интересных>> ситуаций и мониторинг. В совокупности две эти возможности задают метрику и механизм оценки полноты тестирования. Мониторинг организовать относительно легко, поскольку мы работаем не с реальным процессором, а с его моделями. Как построить перечень «интересных» ситуаций» --- вопрос открытый --- это одно из направлений моей работы.
%
%%Е) нацеленное/ненацеленное тестирование
}

%%%%%%%%%%%%%%%%%%%%%%
%%%%%%%%%%%%%%%%%%%%%%
%%%%%%%%%%%%%%%%%%%%%%


\newcommand{\Objective}{%
%Ж) формулирование цели работы - исследование методов построения нацеленных тестов-программ (на память)
%Целью исследования является разработка методов целенаправленной генерации системных тестов, которые, в свою очередь, должны предлагать и адекватные методы задания метрики и оценки полноты покрытия в соответствии с предложенными метриками.

Целью диссертационной работы является исследование и разработка методов и программных средств построения тестовых программ для проверки подсистем управления памяти микропроцессоров.

%%% надо тут, видимо, более точно изложить цель - что целью является улучшение некоторых ппараметров!

Для достижения этой цели были поставлены следующие задачи:
\begin{enumerate}
	\item исследовать описанные в научной литературе методы построения тестовых программ на предмет их применимости для системного тестирования подсистем управления памяти микропроцессоров;
	\item разработать методы построения тестовых программ для системного функционального тестирования подсистем управления памяти.
\end{enumerate}
}

%%%%%%%%%%%%%%%%%%%%%%
%%%%%%%%%%%%%%%%%%%%%%
%%%%%%%%%%%%%%%%%%%%%%


\newcommand{\Novelty}{%

Научной новизной обладают следующие результаты работы:
\begin{enumerate}
    \item предложен подход к построению тестовых программ для проверки подсистем управления памяти микропроцессоров, сочетающий формализацию документации и технику ограничений;

    \item предложен метод моделирования устройств подсистемы управления памяти, использующий конечные автоматы специального вида;

    \item предложена формальная модель инструкций, описывающая отдельные пути их выполнения в виде утверждений о свойствах параметров инструкций и модельном состоянии устройств;

    \item в рамках предложенного подхода разработан метод формализации механизма вытеснения данных при помощи построения ограничений, эффективно разрешаемых современным инструментарием.
\end{enumerate}

}

%%%%%%%%%%%%%%%%%%%%%%
%%%%%%%%%%%%%%%%%%%%%%
%%%%%%%%%%%%%%%%%%%%%%


\newcommand{\PracticalValue}{%

Разработанные модели и методы могут быть использованы коллективами, занимающимися разработкой микропроцессоров, для автоматизации построения тестовых программ. Разработанный прототип системы построения тестовых программ использовался для генерации тестов подсистем управления памяти ряда микропроцессоров архитектуры MIPS64. Результаты работы могут быть использованы в исследованиях, которые ведутся в Институте системного программирования РАН, Московском государственном институте электроники и математики, НИИ системных исследований РАН, Институте точной механики и вычислительной техники им. С.А. Лебедева РАН, Институте проблем информатики РАН и других
научных и промышленных организациях.

}

%%%%%%%%%%%%%%%%%%%%%%
%%%%%%%%%%%%%%%%%%%%%%
%%%%%%%%%%%%%%%%%%%%%%

\newcommand{\Pub}{

По материалам диссертации опубликовано одиннадцать работ~\cite{my_syrcose_2008, my_isp_2008, my_lomonosov_2009, my_lomonosov_2010, my_miet_2009, my_nivc_2009, my_syrcose_2009, my_isp_2009, my_ewdts_2009, my_programmirovanie_2010, my_isp_2010}, в том числе одна~\cite{my_programmirovanie_2010} в издании, входящем в перечень ведущих рецензируемых научных журналов и изданий ВАК. Основные положения докладывались на следующих конференциях и семинарах:
\begin{enumerate}
  \item на втором и третьем весеннем коллоквиуме молодых исследователей в области программной инженерии (SYRCoSE) (2008 и 2009 гг.);
  \item на шестнадцатой и семнадцатой международной конференции студентов, аспирантов и молодых ученых <<Ломоносов>> (2009 и 2010 гг.);
  \item на шестнадцатой всероссийской межвузовской научно-технической конференции студентов и аспирантов <<Микроэлектроника и информатика - 2009>> (2009 г.);
  \item на седьмом международном симпозиуме по проектированию и тестированию под эгидой IEEE (EWDTS) (2009 г.);
  \item на российско-ирландской летней школе по научным вычислениям (2009 г.);
  \item на научной конференции <<Тихоновские чтения>> (2009 г.);
  \item на научной конференции <<Ломоносовские чтения>> (2010 г.);
  \item на объединенном научно-исследовательском семинаре имени М.Р. Шура-Бура (2010 г.);
  \item на семинаре Лаборатории вычислительных комплексов факультета вычислительной математики и кибернетики МГУ имени М.В.Ломоносова (2010 г.);
  \item на семинаре отдела Технологий программирования института системного программирования РАН (2009, 2010 гг.).
\end{enumerate}

}

%%%%%%%%%%%%%%%%%%%%%%
%%%%%%%%%%%%%%%%%%%%%%
%%%%%%%%%%%%%%%%%%%%%%


\newcommand{\Structure}{%Структура и объем диссертации

Работа состоит из введения, трех глав, заключения, списка литературы и приложений.
Общий объем основной части диссертации составляет 125 страниц.
Список литературы содержит 71 наименование.
}

\newcommand{\Results}{%
\begin{enumerate}
  \item	Предложен подход к построению тестовых программ для проверки подсистем управления памяти микропроцессоров, позволяющий понизить сложность построения некоторых классов тестовых программ. В таких тестовых программах имеется цепочка длиной от 6 до 12 инструкций обращения к памяти. Понижение сложности обосновано при помощи ряда экспериментов на прототипе программного средства построения тестовых программ. Теоретически обоснована корректность алгоритмов в рамках подхода к построения тестовых программ.

%  \item Предложен метод моделирования устройств подсистемы управления памяти, использующий расширенные конечные автоматы с заданным набором операций. Моделью состояния является последовательность ассоциативных массивов, операциями --- операции обращений в устройство при наличии искомого ключа в ассоциативных массивах и при его отсутствии.
%
%  \item Предложена модель инструкций для описания отдельных путей выполнения инструкций в виде набора утверждений о свойствах операндов инструкций и содержимого устройств и изменения содержимого устройств.
%
  \item В рамках предложенного подхода разработан метод моделирования механизма вытеснения данных, позволяющий в отличие от других методов моделирования выразить ряд свойств вытесняемых данных в виде ограничений, эффективно разрешаемых современным инструментарием. Теоретически обоснована корректность метода для ряда стратегий вытеснения.

%  \item На основе предложенных моделей и методов создан прототип системы построения тестовых программ для проверки подсистем управления памяти микропроцессоров архитектуры MIPS64 и проведены эксперименты для оценки эффективности разработанного прототипа.
\end{enumerate}
}
%eps/intro11}

\newcommand{\LRU}{LRU\xspace}
\newcommand{\FIFO}{FIFO\xspace}
\newcommand{\PseudoLRU}{Pseudo-LRU\xspace}

\renewcommand{\facsimile}{\vspace{0.5cm}\includegraphics[width=3cm]{my}}
% --------------------------------------------------------------------------
\begin{document}

% Внешняя сторона обложки
\title{АВТОРЕФЕРАТ\\
диссертации на соискание ученой степени\\
кандидата физико-математических наук}

\maketitle

% Внутренняя сторона обложки
\noindent
Работа выполнена на кафедре системного программирования факультета вычислительной математики и кибернетики Московского государственного университета имени М. В. Ломоносова.
\vskip1ex
\noindent\begin{tabularx}{\linewidth}{lp{0.3cm}X}
Научный руководитель:  & & \emph{доктор физико-математических наук,} \\
                       & & \emph{старший научный сотрудник} \\
                       & & \emph{\textbf{Петренко Александр Константинович}}.
\\
Официальные оппоненты: & & \emph{доктор физико-математических наук}, \\
                       & & \emph{профессор} \\
                       & & \emph{\textbf{Смелянский Руслан Леонидович}};\\
                       & & \emph{доктор физико-математических наук} \\
                       & & \emph{\textbf{Лацис Алексей Оттович}}.
\\
Ведущая организация:   & & \emph{Научно-исследовательский институт}\\
& & \emph{системных исследований РАН}
\end{tabularx}

\vskip2ex\noindent
Защита состоится %<<\underline{15}>> \underline{октября} 2010 года в \underline{11} часов
26 ноября 2010 года в 11 часов
на заседании диссертационного совета \emph{Д 501.001.44} \emph{Московского
государственного университета имени М.В. Ломоносова} по адресу:
\emph{119991, ГСП-1, Москва, Ленинские горы, МГУ имени М.В.Ломоносова, 2-й учебный корпус, факультет ВМК, ауд. 685.}

\vskip1ex\noindent
С диссертацией можно ознакомиться в библиотеке
\emph{факультета ВМК МГУ}.
С текстом автореферата можно ознакомиться на официальном сайте ВМК МГУ \underline{http://cs.msu.ru} в разделе <<Наука>> --- <<Работа диссертационных советов>> --- <<Д 501.001.44>>

\vskip1ex\noindent
Автореферат разослан <<14>> сентября 2010 г.
%\vskip2ex\noindent
%Отзывы и замечания по автореферату в двух экземплярах, заверенные печатью, просьба высылать по вышеуказанному адресу на имя ученого секретаря диссертационного совета.

\vfill\noindent
\raisebox{-30pt}[1pt][10pt]{\includegraphics[width=1cm]{sh}}
Ученый секретарь

\hspace{0.4cm}диссертационного совета

\hspace{0.4cm}\emph{профессор}%
%
%\makeatletter
% вставка файла, содержащего факсимиле ученого секретаря
%\ifDis@facsimile
  %\raisebox{100pt}{\includegraphics[width=4cm]{korolev}}\hfill
  \hspace{5cm}\raisebox{-4pt}[1pt][10pt]{\includegraphics[width=4cm]{korolev}}
%\fi%
%\makeatother%
%
\hfill
\emph{Н.П. Трифонов}

\clearpage

\section*{Общая характеристика работы}

\subsection*{Актуальность темы}
\newcommand{\nocites}{}
\Actuality

\subsection*{Цель диссертационной работы}
\Objective

\subsection*{Научная новизна}
\Novelty

\subsection*{Практическая значимость}
\PracticalValue

% Результаты и положения, выносимые на защиту
%\resultssection
%\resultstext

% Апробация работы
%\approbationsection
%\approbationtext

\subsection*{Апробация работы и публикации}
\Pub

% Личный вклад автора
%\contribsection
%\contribtext

\subsection*{Структура и объем диссертации}
\Structure


%
% -----------------------------------------------------------------------
%


\section*{Содержание работы}

\paragraph{Во Введении} обоснована актуальность диссертационной работы,
сформулирована цель и задачи исследований, сформулированы полученные результаты и показана их
практическая значимость.

%
% -----------------------------------------------------------------------
%


\paragraph{Первая глава} содержит обзор существующих методов построения тестовых программ, обзор подсистем управления памяти и уточнение задач исследования по результатам этих обзоров.

В разделе 1.1 дается схема системного тестирования микропроцессора, описываются его отдельные этапы, в том числе и этап построения тестовых программ. Рассматриваются методы псевдослучайного и целенаправленного автоматического построения тестовых программ. В разделе 1.2 кратко описываются функции и типичный состав подсистем управления памяти, способы повышения их эффективности и классы ошибок. В разделе 1.3 сделан обзор методов целенаправленного построения тестовых программ. Выделены методы на основе массовой генерации тестовых программ и методы непосредственного построения тестовых программ. В разделе 1.4 сделан анализ методов целенаправленного построения тестовых программ по применимости этих методов для тестирования подсистем управления памяти, масштабируемости, возможности <<нацеливания>> на функциональность. Наиболее перспективными являются методы непосредственного построения тестовых программ по тестовым ситуациям, формализованным в виде \emph{тестовых шаблонов} --- цепочек инструкций с указанием вариантов исполнения инструкций --- и методы, включающие разрешение ограничений (constraint satisfaction). В разделе 1.5 уточнены задачи в соответствии с проведенным исследованием методов построения тестовых программ.

\paragraph{Во второй главе} предложен подход целенаправленного построения нацеленных тестовых программ, сочетающий особенности перспективных методов построения тестовых программ. В разделе 2.1 описаны этапы построения тестовых программ согласно предлагаемому подходу: чтение документации по архитектуре, формализация архитектуры, выделение и формализация тестовых ситуаций (в виде шаблонов тестовых программ, далее \emph{тестовых шаблонов}), построение и решение системы ограничений для каждого тестового шаблона, конструирование текста тестовой программы на основе решения каждой совместной системы ограничений.

Тестовый шаблон содержит последовательность инструкций, для каждой из которых указан вариант исполнения (т.е. некоторый путь выполнения инструкции). Для алгоритмического построения ограничений и затем тестовых программ варианта инструкций должны быть формализованы в рамках этапа формализации микропроцессора. Раздел 2.2 более детально освещает этот этап . В этом разделе предложены модель функционирования устройств подсистемы управления памяти и модель вариантов инструкций. В рамках этого этапа следует выделить и формализовать те варианты инструкций, которые входят в тестовый шаблон, и задействованные в них устройства подсистемы управления памяти. Моделью состояния устройства подсистемы управления памяти (кэш-памяти, таблицы страниц и т.п.) предлагается последовательность ассоциативных массивов (далее эта последовательность будет называться \emph{таблицей}, а отдельный ассоциативный массив --- \emph{регионом}). Каждый регион состоит из \emph{строк}, каждая строка состоит из набора \emph{полей}, поля делятся на \emph{поля ключа} и \emph{поля данных}. Поля ключа задают ключи в ассоциативном массиве, поля данных --- значения. В модели устройства определены следующие операции: успешного обращения, успешного обращения с изменением, неуспешного обращения с замещением. На входе операции успешного обращения --- выражение, задающие ключ, и выражение, задающее номер ассоциативного массива из состояния устройства. Операция определена на тех входных данных и состоянии модели, при которых в соответствующем ассоциативном массиве есть строка, поля ключей которой \emph{соответствуют} аргументу-ключу этой операции. Операция возвращает поля данных соответствующей аргументу-ключу строки. Операция успешного обращения с изменением отличается от операции успешного обращения дополнительным аргументом --- полями данных, которые нужно заменить в строке, соответствующей аргументу-ключу. На входе операции  неуспешного обращения с замещением 3 аргумента: выражение, задающие ключ, выражение, задающее номер ассоциативного массива, и выражения для полей данных строки. Операция определена на тех входных данных и состоянии модели, при которых в соответствующем ассоциативном массиве нет строки, поля ключей которой \emph{соответствуют} аргументу-ключу этой операции. Эффект операции заключается в замене полей данных одной из строк в ассоциативном массиве, номер которого был передан в качестве одного из аргументов операции, на переданные операции поля данных. Выбор строки для замены определяется на основе стратегии вытеснения так же, как это делается в устройствах подсистемы управления памяти (например, в кэш-памяти). В рамках этапа формализации следует указать набор атрибутов, задающих модели устройств тестируемой подсистемы управления памяти: стратегия вытеснения, количество строк в массиве (<<ассоциативность>>), двоичный логарифм количества массивов, имена и битовые длины полей ключа и полей данных, предикат соответствия строки некоторой битовой строке (эта строка является аргументом-ключом в операциях над моделью). Алгоритмы построения систем ограничений параметризованы этими атрибутами. Таким образом моделируются кэш-память различных уровней, таблицы страниц, буферы трансляции адресов (TLB) и даже оперативная память.

В том же разделе предложены модели вариантов инструкций. Варианты инструкций соответствуют отдельным путям выполнения инструкций. Модель варианта инструкции формализует следующие виды требований: допустимые значения операндов инструкции, вычисление значений операндов-результатов инструкции, условия возникновения исключительных ситуаций (если вариант инструкции состоит в возникновении такой ситуации), вычисление адресов (физических, виртуальных, эффективных) в инструкции, обращения в устройства подсистемы управления памяти и успешность этих обращений, какие данные загружаются или сохраняются. При формализации для варианта инструкции объявляются операнды инструкции и последовательность операторов 4-х видов: оператор утверждения истинности свойства над битовыми строками (\texttt{assume}), оператор введения новой переменной (\texttt{let}), оператор попадания (\texttt{hit}), оператор промаха (\texttt{miss}). С помощью этой последовательности операторов задается набор условий на значения операндов инструкции и для моделей устройств, задействованных в этом варианте инструкции, условия на их состояние и изменение этого состояния в рамках данного варианта. Операторы \texttt{let} и \texttt{assume} имеют ту же семантику, которая используется при формализации императивных языков программирования. Операторы \texttt{hit} и \texttt{miss} специфичны инструкциям, оперирующим с памятью. Оператор попадания \texttt{hit<B>(k;R)\{loaded(d); [storing(d');] \}}, где \texttt{k, R, d, d'} --- выражения над определенными ранее переменными-битовыми строками, означает, что в данном варианте инструкции должно осуществляться успешное обращение в устройство \texttt{B} по ключу \texttt{k} в массив с номером \texttt{R}, причем ключу соответствуют данные \texttt{d} (если полей данных несколько, то соответствующие выражения для них перечисляются в скобках у \texttt{loaded}). Если указана секция \texttt{storing}, то в варианте инструкции должно осуществляться успешное обращение с изменением на поля данных \texttt{d'}. Оператор промаха \texttt{miss<B>(k;R)\{[replacing(d);]\}}, где \texttt{k, R, d} --- выражения над определенными ранее переменными-битовыми строками, означает, что в данном варианте инструкции должно быть неуспешное обращение в устройство \texttt{B} по ключу \texttt{k} в массив с номером \texttt{R}. Секция \texttt{replacing} задает поля данных \texttt{d'} вытесняющей строки. Если секция \texttt{replacing} отсутствует, изменение состояния устройства \texttt{B} не должно происходить.

Формализовав устройства подсистемы управления памяти и инструкции и составив тестовый шаблон, нужно определить те значения регистров и хранящиеся данные устройств, при которых исполнение инструкций, указанных в тестовом шаблоне, будет проходить согласно указанным там вариантам. Для этого составляется система ограничений, выражающая все необходимые условия на такие значения. В результате разрешения этой системы определяются искомые значения. В разделе 2.3 описывается предлагаемый алгоритм построения ограничений для тестового шаблона. Исполнение инструкций конвейеризовано, поэтому расположенные рядом инструкции в действительности будут выполняться с существенной долей параллелизма. Однако в алгоритме генерации ограничений считается, что инструкции выполняются последовательно, а тестовые шаблоны составлены таким образом, чтобы при работе соответствующих им тестовых программ проявились все нужные параллельные эффекты.

По каждому оператору алгоритм строит свою часть ограничений, которые выражают семантику этого оператора. Операторы обращений в устройства транслируются в ограничения без моделирования состояний устройств, несмотря на то, что определение этих операторов включало состояние устройства. Это позволяет существенно уменьшить количество переменных-битовых строк и размер ограничений и, тем самым, ускорить разрешение ограничений. Специальное представление выбрано и для начального содержимого устройств, а именно, последовательность обращений в это устройство. Поэтому в число переменных в ограничениях входят переменные, задающие аргументы этих, \emph{инициализирующих}, обращений: ключи, номера ассоциативных массивов. Для операторов \texttt{hit} и \texttt{miss} строятся ограничения на аргументы-ключи \texttt{k} и номера регионов \texttt{R} (эти ограничения должны гарантировать успешное обращение для \texttt{hit} и неуспешное --- для \texttt{miss}) и ограничения на аргументы-данные \texttt{d} и \texttt{d'} (обращения по одинаковым адресам должны давать одинаковые данные, если они не были изменены). Ограничения на аргументы-ключи и аргументы-номера регионов строятся согласно следующим определениям операторов: \texttt{hit(k$_i$;R$_i$)} / \texttt{miss(k$_i$;R$_i$)} происходит, если
\begin{itemize}
\item[$(\alpha)$] перед ним есть обращение по тому же ключу (\texttt{k}$_i$) в тот же регион (\texttt{R}$_i$),
\item[$(\beta)$] после которого и до этого обращения соответствующая строка \underline{не была} \underline{вытеснена} / \underline{была вытеснена} из таблицы.
\end{itemize}

Трансляция свойства $\alpha$ достаточно очевидна, трансляция свойства $\beta$ рассматривается в разделе 2.5.

В этом же разделе диссертации сформулированы и доказаны теоремы корректности и полноты алгоритмов, гарантирующие, что ограничений, построенные согласно алгоритму, не дают решений, не соответствующих тестовому шаблону, и задают все решения, соответствующие тестовому шаблону. Значение атрибута модели устройства, задающего стратегию вытеснения, равное \texttt{none}, означает, что при неуспешном обращении в это устройство замещение не производится. Здесь и далее символами $t_1, t_2, ..., t_m$ обозначаются аргументы-ключи инициализирующих обращений, $r_1, r_2, ..., r_m$ --- аргументы-регионы инициализирующих обращений, $k_1, k_2, ..., k_n$ --- аргументы-ключи обращений в инструкциях тестового шаблона, $R_1, R_2, ..., R_n$ --- аргументы-регионы обращений в инструкциях тестового шаблона, $S_1, S_2, ..., S_n$ --- успешности обращений в инструкциях тестового шаблона ($S_i = hit$ или $S_i = miss$).

\begin{theorem}[Корректность алгоритма генерации ограничений на ключи обращений для таблицы, стратегия вытеснения которой не \texttt{none}]\label{mirror_correctness}
\CorrectnessMirror
\end{theorem}

\begin{theorem}[Полнота алгоритма генерации ограничений на ключи обращений для таблицы, стратегия вытеснения которой есть \texttt{none}]\label{mirror_fullness_none}
\FullnessMirrorNone
\end{theorem}

Стратегию вытеснения будем называть \emph{существенно вытесняющей}, если $w$ промахов в один регион полностью вытесняют его предыдущее содержимое ($w$ --- значение параметра \texttt{lines} таблицы).

\begin{theorem}[Полнота алгоритма генерации ограничений на ключи обращений для таблицы, стратегия вытеснения которой не \texttt{none} и является \textbf{существенно вытесняющей}]\label{mirror_fullness}
\FullnessMirror
\end{theorem}

В разделе 2.3.2 рассматриваются \emph{таблицы вытеснения}, которые позволяют формализовать стратегии вытеснения. Таблицы вытеснения были предложены в 2008 году исследователями из Университета Саарланда. С использованием таблиц вытеснения в разделе 2.3.3 сформулированы и доказаны теоремы о том, что стратегии вытеснения LRU, FIFO и Pseudo-LRU являются существенно вытесняющими:
\begin{theorem}\label{thm:LRU_essential}
Стратегия вытеснения LRU является существенно вытесняющей.
\end{theorem}

\begin{theorem}
  Стратегия вытеснения FIFO является существенно вытесняющей.
\end{theorem}

\begin{theorem}\label{thm:PseudoLRU_essential} \PseudoLRUEssential \end{theorem}

В этом же разделе исследуется вопрос длины инициализирующей последовательности. Сформулирована и доказана верхняя оценка достаточного количества инициализирующих обращений в устройство подсистемы управления памяти:
\begin{utv}[Верхняя оценка количества инициализирующих обращений]
$$m \leqslant n \cdot (n + 2w)$$ где $w$ --- значение параметра \texttt{lines} таблицы, $n$ --- количество обращений в таблицу в шаблоне.
\end{utv}

Для стратегии вытеснения LRU в этом же разделе сформулирована и доказана более сильная оценка количества инициализирующих обращений:
\begin{theorem}[Верхняя оценка количества инициализирующих обращений для
стратегии вытеснения \LRU]\label{thm_mirror_lenth_lru} \UpperBoundLRUMirror
\end{theorem}

В разделе 2.4 предложено новое определение стратегии вытеснения \PseudoLRU. Обычно при определении \PseudoLRU предлагают рассматривать для региона упорядоченное бинарное дерево с пометками в нелистовых вершинах, листовым вершинам дерева соответствуют строки региона. Как и прежде, обращения осуществляются к одной из листовых вершин дерева, что приводит к изменению пометок вершин пути в неё из корневой вершины. Вытесняемая строка определяется также на основе пометок вершин дерева. Новое определение задает стратегию вытеснения \PseudoLRU как изменение битовых векторов (\emph{\PseudoLRU-ветвей}), сопоставленных строкам региона. Это определение позволит сократить количество ограничений и ускорить их разрешение. Сформулирована и доказана теорема, задающее изменение \PseudoLRU-ветвей и показывающая эквивалентность нового определения \PseudoLRU старому ($W = \log_2 w$, для стратегии вытеснения \PseudoLRU допустимы лишь те $w$, которые являются степенями двойки, \emph{абсолютная позиция}(или просто, \emph{позиция}) --- это номер строки в регионе, \emph{относительная позиция} позиции $i$ относительно позиции $j$ будем назыаать $i \oplus j$ и обозначать как $\pi_j^i$):
\begin{theorem}[Инвариантность преобразования \PseudoLRU-ветвей относительными
позициями]\label{thm_pseudoLRU_invariant} \PseudoLRUInvariant
\end{theorem}

В разделе 2.5 предложен метод построения ограничений для свойства <<быть вытесненным к моменту заданного обращения в таблицу>> --- \emph{метод полезных обращений}. С помощью него это свойство выражается в виде ограничений в операциях над битовыми строками и целочисленной линейной арифметике. В разделе предложена формализация понятия \emph{полезной для вытеснения} (или просто, \emph{полезной}) инструкции. \emph{Формулой полезного обращения} будем называть предикат, истинный для всех обращений, являющихся полезными, и ложный на всех обращениях, не являющихся полезными. В рамках метода полезных обращений свойство <<быть вытесненным>> рассматривается как ограничение снизу количества предыдущих обращений, являющихся полезными.

В разделе 2.5.1 предложено и формально обосновано представление свойства <<быть вытесненным>> для стратегии вытеснения LRU в виде ограничений, составленное по методу полезных обращений (<<||>> --- операция битовой конкатенации, выражение $[\phi]$ равно 1, если $\phi$ истинно, и равно 0 в противном случае):
\begin{theorem}[Выражение свойства <<быть вытесненным>> для \LRU]\label{correct_mirror_LRU} \LRUusefuls
\end{theorem}

В разделе 2.5.2 предложено и формально обосновано представление свойства <<быть вытесненным>> для стратегии вытеснения \FIFO  в виде ограничений, составленное по методу полезных обращений.

В разделе 2.5.3 предложено и формально обосновано представление свойства <<быть вытесненным>> для стратегии вытеснения \PseudoLRU в виде ограничений, составленное по методу полезных обращений:
\begin{theorem}[Выражение свойства <<быть вытесненным>> для \PseudoLRU]\label{correct_mirror_PLRU} \PLRUusefuls

$\pi_i$ --- позиции --- дополнительные переменные, для которых надо построить ограничения:
конъюнкцию для каждой пары $(s_i,\pi_i)$ и $(s_j, \pi_j)$ при $j > i$ ограничений:
\begin{itemize}
    \item если $j$'е обращение успешное, то $(\pi_i||R(s_i) =
\pi_j||R(s_j)~\wedge$ $$\pi_i||R(s_i) \notin \{\pi_{m_1}||R(s_{m_1}),
\pi_{m_2}||R(s_{m_2}), \dots, \pi_{m_n}||R(s_{m_n})\}) \rightarrow s_i = s_j$$
    \item в противном случае $(\pi_i||R(s_i) =
\pi_j||R(s_j)~\wedge$ $$\pi_i||R(s_i) \notin \{\pi_{m_1}||R(s_{m_1}),
\pi_{m_2}||R(s_{m_2}), \dots, \pi_{m_n}||R(s_{m_n})\}) \rightarrow s_i \neq
s_j$$
\end{itemize}
где $(\pi_{m_1},R(s_{m_1})), (\pi_{m_2},R(s_{m_2})), \dots,
(\pi_{m_n},R(s_{m_n}))$ --- позиции и регионы\\неуспешных обращений,
расположенных между $i$'м и $j$'м.
\end{theorem}

В разделе 2.5.4 освещаются некоторые вопросы сложности разрешения ограничений, которые строятся согласно методу полезных обращений.


В разделе 2.6 рассматривается последний этап построения тестовой программы --- этап конструирования текста программы. Текст тестовой программы состоит из инициализирующей части и инструкций тестового шаблона. Инициализирующая часть состоит из инструкций для помещения вычисленных на предыдущем этапе начальных значений в соответствующие регистры и последовательности инструкций, обращающихся в соответствующее устройство подсистемы управления памяти по вычисленным ключам в вычисленные регистры с вычисленными данными. Зачастую конструирование таких инструкций не представляет сложности. Пример противоположного случая --- инициализация многоуровневой кэш-памяти в случае, если инициализацию отдельных уровней кэш-памяти нельзя осуществить отдельно от остальных уровней. Поэтому в разделе 2.6 предложен и метод конструирования последовательности инициализирующих инструкций, учитывающий особенности многоуровневой кэш-памяти.

\paragraph{В третьей главе} дается оценка предлагаемым методам. В разделах 3.1, 3.2 и 3.3 показывается, что подсистемы управления памяти микропроцессоров архитектур MIPS, PowerPC и IA-32 можно представить в виде набора таблиц и формализовать варианты инструкций, оперирующих с памятью. Причем имеется документация по каждой архитектуре, где эти варианты уже описаны на соответствующем псевдокоде. В разделе 3.4 описана автоматизация некоторых этапов предлагаемого во второй главе подхода: генератор ограничений, решатель ограничений и конструктор текстов тестовых программ. Решатель ограничений --- это внешний инструмент (Z3) и разрабатывать его для автоматизации подхода не надо, что сильно сокращает трудоемкость автоматизации подхода. Затронут вопрос переиспользования перечисленных компонентов при смене микропроцессора. Был реализован прототип генератора тестовых программ для архитектуры MIPS64. В разделе 3.5 описаны эксперименты над этим прототипом по оценке времени построения тестовых программ и вероятности завершения построения за 60 секунд. Эксперименты показали увеличение допустимого размера тестовых шаблонов до 9-12 инструкций (по сравнению с 2-3 инструкциями в доступных инструментах). В разделе 3.6 проведено сравнение с инструментом Genesys-Pro (IBM): выделены сходства и отличия, преимущества и недостатки предложенных методов по сравнению с Genesys-Pro. В разделе 3.7 проведено сравнение с работами, проводящимися в Intel.

\paragraph{В Заключении} диссертационной работы перечисляются ее основные результаты.

% ----------------------------------------------------------------

\section*{Основные результаты работы}
\Results


% ----------------------------------------------------------------
\renewcommand\bibsection{\section*{Публикации по теме диссертации}}
%\section*{Публикации по теме диссертации}
\bibliographystyle{gost780s}
\bibliography{../delta/thesis}
% ----------------------------------------------------------------

\end{document}


%%%
%% формулировка результатов
%%% 