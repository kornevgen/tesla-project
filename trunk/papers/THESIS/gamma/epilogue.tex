% !Mode:: "TeX:UTF-8"
\chapter*{Заключение}
\addcontentsline{toc}{chapter}{Заключение}

Основные научные и практические результаты, полученные в
диссертационной работе и выносимые на защиту, состоят в следующем:

\begin{itemize}
  \item Разработан метод генерации ограничений для тестовых
шаблонов, нацеленных на тестирование инструкций обращения к памяти,
с использованием заданного начального состояния микропроцессора,
который использует свойство инструкций изменять состояния нескольких
кэширующих буферов (\emph{совместная генерация ограничений}). Метод
параметризован методом записи стратегии вытеснения в виде
ограничений. Доказана корректность метода для любых тестовых
шаблонов и полнота при некоторых дополнительных условиях на тестовые
шаблоны. Метод позволяет эффективно строить тестовые программы по
тестовым шаблонам, перед исполнением которых не меняется состояние
кэширующих буферов микропроцессора.

  \item Разработан метод генерации ограничений для тестовых
шаблонов, нацеленных на тестирование инструкций обращения к памяти,
без использования начального состояния микропроцессора
(\emph{зеркальная генерация ограничений}). Метод параметризован
методом записи стратегии вытеснения в виде ограничений. Доказана
корректность метода для любых тестовых шаблонов и полнота при
некоторых дополнительных условиях на стратегии вытеснения. Показано,
что наиболее часто используемые в микропроцессорах стратегии
вытеснения удовлетворяют этим дополнительным условиям, что
обеспечивает полноту зеркального метода генерации ограничений в
практически значимых случаях. Отсутствие требования предъявить
начальное состояние микропроцессора позволяет упростить процесс
тестирования, поскольку перед построением очередной тестовой
программы не нужно считывать текущее состояние микропроцессора. Тем
самым пакет тестовых программ может быть построен полностью до
проведения тестирования.

  \item Разработан метод генерации ограничений для описания
  стратегий вытеснения перебором частей тестовой программы,
  непосредственно влияющих на вытеснение (\emph{перебор диапазонов
  вытеснения}). Приведены ограничения, генерируемые этим методом,
  для стратегий вытеснения \LRU, \FIFO и \PseudoLRU, наиболее
  используемых стратегий вытеснения в микропроцессорах. Показана
  эквивалентность этих ограничений определениям стратегий
  вытеснения. Метод может быть эффективно использован в случае
  кэширующих буферов небольшого размера, каковыми являются,
  например, буферы трансляции адресов некоторых микропроцессоров.

  \item Разработан метод генерации ограничений для описания
  стратегий вытеснения, представляющий условие вытеснение в виде
  границы (верхней или нижней) на количество инструкций,
  непосредственно влияющих на вытеснение (\emph{метод функций
  полезности}). Приведены ограничения, генерируемые этим методом,
  для стратегий вытеснения \LRU, \FIFO и \PseudoLRU, наиболее
  используемых стратегий вытеснения в микропроцессорах. Показана
  эквивалентность этих ограничений определениям стратегий
  вытеснения. Метод может быть эффективно использован в случае
  тестовых шаблонов произвольного размера. В диссертации приведены
  некоторые классы тестовых шаблонов, для которых генерируемые
  ограничения имеют меньший размер, причем практика показывает, что
  для определения многих ошибок достаточно тестовых шаблонов из таких
  классов.
\end{itemize}
