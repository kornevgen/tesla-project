% !Mode:: "TeX:UTF-8"

\zhead{Посчитать значения функций}

\z Для следующей алгебраической спецификации посчитать значения указанных выражений:
\begin{lstlisting}
scheme Sets = class
  type Set, Elem
  value
        empty: Unit -> Set,
        add: Elem >< Set-> Set,
        delete: Elem >< Set -> Set,
        inset: Elem >< Set -> Bool
  axiom
    forall s: Set, x, y: Elem :-
        add(x, add(x, s)) is add(x, s),
        delete(x, empty() ) is delete(x, empty()),
        delete(x, add(x, s)) is delete(x, s),
        delete(x, add(y, s)) is add(y, delete(x,s)),
        ~inset(x, empty() ),
        inset(x, add(y, s)) is (x = y \/ inset(x,s)),
        inset(x, delete(y,s)) is x ~= y /\ inset(x,s)
end
\end{lstlisting}
\begin{itemize}
  \item inset(1, add(1, empty()));
  \item inset(1, delete(1, empty()));
  \item inset(1, delete(1, add(1, empty())));
  \item delete(1, add(1, empty()));
  \item delete(1, add(2, empty()));
  \item delete(1, add(1, add(1, empty())));
  \item delete(1, add(2, add(1, empty())));
  \item add(2, delete(1, add(2, add(1, empty()))));
  \item inset(3, add(2, delete(1, add(2, add(1, empty())))));
  \item inset(3, add(2, delete(3, add(2, add(1, empty())))));
  \item inset(3, add(3, delete(1, add(2, add(1, empty())))));
  \item inset(3, add(3, delete(3, add(3, add(3, empty())))));
\end{itemize}

Решение (1):
\begin{lstlisting}
inset(1, add(1, empty())) is
   1 = 1 \/ inset(1, empty()) is true
\end{lstlisting}

Решение (3):
\begin{lstlisting}
inset(1, delete(1, add(1, empty()))) is
    1 ~= 1 /\ inset(1, add(1, empty())) is false
\end{lstlisting}

Решение (4):
\begin{lstlisting}
delete(1, add(1, empty())) is delete(1, empty())
\end{lstlisting}
В данном случае произвести другие упрощения невозможно.

Обратите внимание, что
\begin{enumerate}
  \item аксиома delete(x, empty() ) is delete(x, empty()) бесполезная, потому что является тождественной истиной; ее можно безболезненно исключить из спецификации;
  \item легко выписать набор аксиом, с которым крайне тяжело работать, из них тяжело извлечь полезный смысл (тяжело ответить на нужные вопросы) --- они сводятся не к произведению полезных выводов (в т.ч. вычислений), а бездумным переписываниям.
\end{enumerate}

