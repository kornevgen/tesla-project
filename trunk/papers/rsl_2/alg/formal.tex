% !Mode:: "TeX:UTF-8"

\refstepcounter{problem_type}

Первые задачи несложные, в них надо формализовать указанные утверждения. Большинство задач взяты из~\cite{nepeivoda}.


\z Ни одному лысому не нужна расческа

Решение:
\begin{lstlisting}
scheme Exmp = class
  type Lys, R4
  value need: Lys >< R4 -> Bool
  axiom all l:Lys, r:R4 :- ~need(l,r)
end
\end{lstlisting}


\z Все мои тетки не справедливы.
\z Ни один кошмарный сон не приятен.
\z  Все битвы сопровождаются страшным шумом.
\z Не все двоечники ленивы.
\z  Каждый, кто упорно работает, добивается успеха.
\z  Ни один бездельник не станет знаменитостью.
\z  Некоторые художники не бездельники.
\z  Некоторые бездельники не художники.
\z  Некоторые подушки мягкие.
\z  Тот, кто может укрощать крокодилов, заслуживает уважения.
\z  Ни одна лягушка не имеет поэтической внешности.
\z  Ни одна тачка не комфортабельна.
\z  Всякий орел умеет летать.
\z  Некоторые свиньи не умеют летать.
\z  Некоторые свиньи --- не орлы.
\z  Ни один судья не справедлив.
\z  Ни один ребенок не любит прилежно заниматься.
\z  Все шутки для того и предназначены, чтобы смешить людей.
\z  Ни один парламентский акт не шутка.
\z  Пауки ткут паутину.
\z  Все лекарства имеют отвратительный вкус.
\z  Ни у одной ящерицы нет волос.
\z  Все свиньи прожорливы.
\z  Все, что сделано из золота, драгоценно.
\z  Некоторые секретари --- птицы.
\z  Все секретари заняты полезным делом.
\z  Ничто разумное не ставит меня в тупик.
\z  Логика часто ставит меня в тупик.
\z  Некоторые цыплята --- не кошки.
\z Точки A, B, C являются вершинами равнобедренного треугольника.
\z Иванов, Петров, Васильев и Сидоров могут вытащить эту машину из ямы, если они трезвы и видят бутылку.
\z Иванов, Петров, Васильев и Сидоров не могут решать квадратные уравнения, даже если они трезвы, но видят бутылку.
\z Не все углы, синус которых больше 1/2, больше $\pi$/6.
\z Квадратные корни из некоторых рациональных чисел иррациональны.
\z Синус и косинус равны друг другу тогда и только тогда, когда равны тангенс и котангенс.
\z  Когда кто-то поет больше часа, он надоедает.
\z Все девочки боятся лягушек и мышей.
\z Кошки бывают только белые и серые.
\z  Все ораторы либо честолюбивы, либо скучны.
\z Число делится на 25 в том и только том случае, когда оно делится на 50 либо дает при делении на 50 остаток 25.
\z Все комплексные числа действительны или становятся действительными после умножения на i.
\z Не все студенты отличники или спортсмены.
\z Для того чтобы выполнялось равенство $x = \sqrt{x^2}$, необходимо и достаточно, чтобы $x$ было положительным действительным числом.
\z Не все, что рассказывал барон К. Ф. И. фон Мюнхгаузен, ложь.
\z Некоторые людоеды — плохие люди.
\z Некоторые финансисты — мошенники, но не все.
\z Прапорщики любят порядок, и не только они.
\z Милиционеры замешаны в преступлениях, но не все.
\z Некоторые замки не отпираются, но запираются.
\z Если будешь хорошо учиться, поступишь в вуз, а иначе
провалишься.
\z Ничего не вижу, ничего не слышу, ничего не знаю.
\z Молодо—зелено.
\z Взялся за гуж --- не говори, что не дюж.
\z Чтобы не быть собакой, достаточно быть кошкой.
\z Чтобы не быть человеком, необходимо быть свиньей.
\z Некоторые кошки поют по ночам.
\z Все компакты совершенно нормальны.
\z Некоторые мюмзики не куздры.
\z Три точки A, B, C лежат на одной окружности.
\z Три точки A, B, C не лежат на одной прямой.
\z Числа a и b имеют одинаковый знак.
\z Одно из чисел a, b равно 0.
\z Числа a и b имеют разные знаки.
\z Ромео и Джульетта любят друг друга.
\z Гамлет и Клавдий ненавидят друг друга.
\z Мери любила Печорина, но не взаимно.
\z Чтобы прийти на свадьбу, необходимо приглашение жениха или невесты.
\z Некоторые лентяи не оптимисты, но жизнелюбы.
\z Все замки отпираются и запираются.
\z Никто из нашего класса не поехал в Москву и Париж.
\z Все мои одноклассники поехали в Москву и Париж.
\z Некоторые числа четные.
\z Некоторые лекции невозможно понять.
\z Всякому в Москве не перекланяешься.

\zhead{Более сложные задачи:}

\z Весна, и вы говорите: «Все парни и девушки сейчас влюблены друг в друга». Что это означает?
\z Когда Ясон кинул в середину войска, выросшего из зубов дракона, камень, все воины передрались и перебили друг друга.
Переведите это утверждение на формальный язык.
\z Ограниченная сверху и снизу на [a, b] функция непрерывна на нем.
\z Функция $f$ имеет разрыв 2-го рода в точке 0.
\z Функция, имеющая точку разрыва, не может быть непрерывной.
\z Монотонная на [a, b] функция ограничена снизу на [a, b].
\z Если последовательность ограничена, то она имеет предельную точку.
\z Чтобы установить рекорд, необходимо иметь способности, и прилежно тренироваться, и найти хорошего тренера.
\z Чтобы победить, нам необходимы и хорошие нападающие, и хорошие защитники, и хорошие вратари.
\z Если вчера Петров прогулял два занятия, то сегодня только одно.
\z Все члены Политбюро, избранного на XIV съезде ВКП(б), ненавидели друг друга.
\z Все философы критиковали друг друга.
\z Все рыцари сражались друг с другом на поединках.
\z Все начальники подсиживают друг друга.
\z Все мужчины --- подонки, а мой муж --- хороший человек.
\z Сдай экзамен на <<отлично>>, и поступишь в аспирантуру.
