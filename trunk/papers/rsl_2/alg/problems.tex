% !Mode:: "TeX:UTF-8"
\zhead{Выбор генераторов}

\z Для стека, определенного таким образом:
\begin{lstlisting}
type E, S == empty | push(S, E) | pop(S, E)
\end{lstlisting}
дать алгебраическую спецификацию операции проверки наличия заданного элемента
\begin{lstlisting}
value include: S >< E -> Bool
\end{lstlisting}

\textbf{Решение:}
\begin{lstlisting}
type E, S == empty | push(S, E) | pop(S)
value include: S >< E -> Bool
axiom forall e,e1:E, s:S :-
  ~ include(e, empty),
  include(e, push(s,e1)) is e = e1 \/ include(e,s),
  include(e, pop(s)) is
     if e = top(s) then count(e,s) > 1
             else count(e,s) > 0 end  pre s ~= empty
value count: E >< S -> Nat,
      top: S -~-> E
axiom forall e,e1:E, s:S :-
   count(e, empty) is 0,
   count(e, push(s,e1)) is count(e,s) +
      if e = e1 then 1 else 0 end,
   count(e, pop(s)) is count(e,s) -
      if e = top(s) then 1 else 0 end pre s ~= empty,

   top(push(s,e1)) is e1,
   top(pop(s)) is last(s,2) pre size(s) >= 2

value last: S >< Nat -~-> E,
      size: S -> Nat
axiom forall s:S, e:E, n:Nat :-
   size(empty) is 0,
   size(push(s,e)) is size(s) + 1,
   size(pop(s)) is size(s) - 1 pre s ~= empty,

   last(push(s,e), n) is
      if n = 1 then e else last(s, n-1) end
        pre n > 0 /\ n <= size(s),
   last(pop(s), n) is last(s, n+1)
        pre s ~= empty /\ n > 0 /\ n < size(s)
\end{lstlisting}

Обратите внимание, что
\begin{enumerate}
\item при помощи выбранных для описания стека генераторов значение стека будет иметь вид, например, push(push(pop(push(empty,1)),10),1); такое выражения значения в типе стек может казаться наиболее адекватным, ведь добавление и удаление элементов --- именно те операции, при помощи которых можно изменить значение (<<состояние>>) стека; однако посмотрите, насколько увеличивается спецификация и теряется ее наглядность, если в число генераторов включена лишняя функция (pop); сравните:
\begin{lstlisting}
type E, S == empty | push(S, E)
value include: S >< E -> Bool
axiom forall e,e1:E, s:S :-
  ~ include(e, empty),
  include(e, push(s,e1)) is e = e1 \/ include(e,s),
\end{lstlisting}

\item в этом примере синтаксически разные термы из генераторов могут означать одинаковые значения типа, например, push(pop(push(empty,1)),2) и push(pop(push(empty,3)),2); в таких случаях надо быть внимательными при выписывании аксиом: помнить и понимать, сколько разных возможностей есть для рекурсивной части цепочки (речь идет о переменной <<s>> в примерах) --- например, в аксиоме с генератором, удаляющим элемент, надо в том числе предполагать, что этот элемент может появиться много раз до этого, добавляться и удаляться.
\end{enumerate}

%% минимальный набор генераторов уменьшает спецификацию (оценить количество аксиом?)

%% при нескольких генераторах надо быть аккуратными

\z Для множества, определенного таким образом:
\begin{lstlisting}
type E, S == empty | add(S, E) | delete(S, E)
\end{lstlisting}
дать алгебраическую спецификацию операции проверки наличия заданного элемента
\begin{lstlisting}
value include: S >< E -> Bool
\end{lstlisting}


\zhead{Описание эффекта  на основе структуры терма}

Вы уже заметили, что основной принцип написания аксиомы --- понять, как вычисляется обсервер после последнего сработавшего генератора. При этом для выражения этой аксиомы используются аргументы, с которыми вызван обсервер и последний генератор. Однако не всегда просто описать эффект работы генератора на основе лишь аргументов последнего из них.

\z Дать алгебраическую спецификацию типа <<Ограниченная очередь>>. В эту очередь можно добавлять и удалять элементы, но только если количество хранящихся элементов не превышает заданную величину.

\textbf{Решение:}
\begin{lstlisting}
value capacity : Nat
type E, S == empty | add(E,S)
value delete: S -~-> E
axiom forall e:E, s:S :-
   delete(add(e,s)) is s
     pre size(s) < capacity

value size: S -> Nat
axiom forall e:E, s:S :-
   size(empty) is 0,
   size(add(e,s)) is size(s) + 1
       pre size(s) < capacity
\end{lstlisting}

Обратите внимание, что
\begin{enumerate}
\item пришлось добавить и описать дополнительный обсервер size;
\item существует множество способов реализации ограниченной очереди при помощи имеющихся в языках программирования средств (например, при помощи <<кольцевой очереди>>, реализованной при помощи массива и двух указателей), однако здесь предъявляется именно формализация функциональности операций, которая остается справедливой и неизменной для любой реализации <<ограниченной очереди>>;
\item данное описание не дает определения того, в каких случаях определена каждая операция в отдельности;
\item при написании аксиомы для рекурсивной части терма достаточно представлять только \emph{правильно построенные термы} --- такие термы, в которых все функции вызваны с аргументами правильных типов и в каждой функции выполнено ее предусловие; например, для аксиомы size(add(e,s)) не надо представлять, что она должна описывать и такой терм: size(add(e,add(e1,add(e2,empty)))) при capacity = 2.
\end{enumerate}

\z Дать алгебраическую спецификацию типа <<Исключающая очередь>>. В эту очередь элемент добавляется в том случае, если его не было, а если он там был, то он удаляется из очереди. Опишите операцию проверки наличия заданного элемента в такой очереди.


\zhead{<<Эффект>> операций}
%% приходится добавлять обсерверы, чтобы полностью описать эффект функции

\z В~\cite{tanenbaum_os} описаны операции с файлами, среди них описана операция Create следующим образом: <<\textsf{Create} (создание). Файл создается без данных. Этот системный вызов объявляет о появлении нового файла и позволяет установить некоторые его атрибуты.>> Напишите алгебраическую спецификацию файловой подсистемы с этой операцией. Естественно, вам понадобится сигнатура этой операции. Вот она:  \texttt{int creat(char *path, int mode)}, параметр \texttt{path} содержит полное или относительное имя файла, параметр \texttt{mode} устанавливает атрибуты прав доступа различных категорий пользователей к новому файлу при его создании (если файл уже существовал, то новый не создается), операция возвращает значение файлового дескриптора для открытого файла при нормальном завершении и значение -1 при возникновении ошибки.

Решение:
\begin{lstlisting}
scheme FS = class
  type Path, Mode, FID, FS
  value
        creat : Path >< Mode >< FS -~-> FS >< FID,
        size: FS >< FID -~-> Nat,
        known: FS >< Path -> Bool,
        access: FS >< FID -~-> Mode,
        first: FS >< FID -> FS
        first(a,b) is a
  axiom
    forall path: Path, mode: Mode, fs: FS :-
        size(creat( path, mode, fs )) is 0,
        known(first(creat( path, mode, fs )), path),
        access(creat( path, mode, fs)) is mode
end
\end{lstlisting}
Обратите внимание, что
\begin{enumerate}
  \item для описания эффекта функции \texttt{creat} были введены дополнительные операции-обсерверы;
  \item аксиомы напрямую выражают текст, описывающий операцию \texttt{creat} --- аксиомы формализуют \emph{требования} на эту операцию.
\end{enumerate}
Ответьте на следующие вопросы:
\begin{enumerate}
  \item эта спецификация неполная, почему? является ли она противоречивой? как, добавив 1 аксиому, полностью описать операцию known?
  \item имеют ли смысл сами по себе введенные дополнительные операции или они выполняют лишь вспомогательную для описания \texttt{creat} функцию?
  \item допустим, мы догадываемся, что Path = \textbf{Text}, а Mode = \textbf{Nat}; дополненная этим знанием спецификация, останется ли алгебраической ? станет ли полной ? останется ли непротиворечивой ? будет ли она соответствовать исходной постановке задачи ? не станет ли она допускать того, что не должно бы по условию ?
\end{enumerate}

\z В~\cite{tanenbaum_os} описаны операции с файлами, среди них описана операция Delete следующим образом: <<\textsf{Delete} (удаление). Когда файл уже более не нужен, его удаляют, чтобы освободить пространство на диске. Этот системный вызов присутствует в каждой операционной системе.>> Напишите алгебраическую спецификацию файловой подсистемы с этой операцией. Естественно, вам понадобится сигнатура этой операции. Вот она:  \texttt{void delete(int fid)}, параметр \texttt{fid} содержит значение файлового дескриптора.

\z В~\cite{tanenbaum_os} описаны операции с файлами, среди них описана операция Open следующим образом: <<\textsf{Open} (открытие). Прежде чем использовать файл, процесс должен его открыть. Системный вызов open позволяет системе прочитать в оперативную память атрибуты файла и список дисковых адресов для быстрого доступа к содержимому файла при последующих вызовах.>> Напишите алгебраическую спецификацию файловой подсистемы с этой операцией. Естественно, вам понадобится сигнатура этой операции. Вот она:  \texttt{void delete(int fid)}, параметр \texttt{fid} содержит значение файлового дескриптора.

\z В~\cite{tanenbaum_os} упомянут системный вызов mmap: <<Системный вызов mmap принимает на входе два параметра: имя фала и виртуальный адрес паямти, по которому операционная система отображает указанный файл. Для реализацияи отображения файлов на память изменяются системные внутренние таблицы.При обращении к памяти по адресу от 512 до 576К происходит прерывание из-за отсутствия страницы, обработчик которого предоставляет считанную в память страницу 0 файла.Если потом эта страница удаляется из памяти алгоритмом замены страниц, она записывается в соответствующее место файла.>>

%% не всегда просто понять, полна ли спецификация
\zhead{Противоречивость алгебраических спецификаций}

Если спецификация допускает подстановку и <<вычисление>> термов увеличивающейся длины, пытаться <<вычислять>> эти термы разными способами, которые допускают аксиомы, и проверять, получаются ли одинаковые результаты в разных способах. Если получились разные, значит, найдено противоречие.

\z Противоречива ли следующая спецификация?
\begin{lstlisting}
type E, T
value empty : T,
      add: T >< E -> T,
      check: T >< E -> Bool
axiom forall e,e1: E, t: T :-
   add(add(t,e),e) is add(t,e),
   add(add(t,e),e1) is add(add(t,e1),e),
   ~check(empty, e),
   check( add(t,e1), e) is e = e1 \/ check(t,e),
\end{lstlisting}

%\textbf{Решение:}
%check( add(empty,

\z Противоречива ли следующая спецификация?
\begin{lstlisting}
type Integer = Int
value MAXINT : Integer :- MAXINT > 0
axiom all i: Integer :- abs i < MAXINT
value add: Integer >< Integer -~-> Integer
axiom forall x, y: Integer :-
   add(x, 0) is x,
   add(x, y+1) is add(x,y)+1
\end{lstlisting}

\textbf{Решение:}
Из сигнатуры add следует, что all x, y : Int :- abs x < MAXINT /\ abs y < MAXINT /\ (add определена для x и y => abs add(x,y) < MAXINT). Из первой аксиомы следует, что add(MAXINT-1, 0) определен и равен MAXINT-1. Тогда по второй аксиоме add(MAXINT-1,1) определен и равен MAXINT --- получили противоречие с тем, что аргументы функции допустимы и функция для них определена, но ее значение выходит за рамки типа. Ответ: противоречива.

\z Противоречива ли следующая спецификация?
\begin{lstlisting}
type T
value empty : T,
    put: T >< Nat -> T,
    get: T >< Nat -> Bool
axiom forall t: T, x, y, z: Nat :-
  put( put(t, x), x ) is put(t, x),
  ~get( empty, x ),
  get( put(empty, x), y ) is (x = y /\ x\2 = 0),
  get(put(put(empty,x),y),z) is (z=x /\ y\2=0) pre x\2=0,
  get( put(t, y), x ) is get(t, x) pre y ~= x,
  get( put( put(t, x), y), x ) is get( put(t,x), x ),
  ~get(put(put(t,x),x+1),x+1) pre ~get(t,x+1) /\ get(t,x)
\end{lstlisting}

\z Противоречива ли следующая спецификация?
\begin{lstlisting}
type T
value empty: T,
    put: T >< Nat -> T,
    get: T >< Nat -> Bool
axiom forall t:T, x, y, z: Nat :-
  put(put(t, x), x) is put(t, x),
  get( put (put(t, 0), x ), x) is (x > 0),
  get( put (put(t, 2*x), x ), x) is (x > 0),
  ~get( empty, x ),
  ~get( put(empty,x), y ),
  get(put(put(empty,x),y),z) is (z>0 /\ z=abs(x-y) )
\end{lstlisting}

\z Противоречива ли следующая спецификация?
\begin{lstlisting}
type T = Nat
value empty: T,
    put: T >< Nat -> T,
    get: T >< Nat -> Bool
axiom forall t: T, x, y, z: Nat :-
  put( put(t, x), y ) is put(t,x) pre y <= x,
  ~get( empty, x ),
  get( put(empty, x), y ) is (x = y),
  get(put(put(empty,x),y),z) is (z = x \/ z = y /\ y > x),
  get( put(t,x), y ) is get(t,y) pre y ~= x,
  ~get(put(t,2*y),2*y) pre get(t,y) /\
       get(t,y+1) /\~get(t,2*y),
  get( put(put(t, 0), x), x ) is get( put(t,x), x )
\end{lstlisting}

\z Противоречива ли следующая спецификация?
\begin{lstlisting}
type T
value empty: T,
    put: T >< Nat -> T,
    get: T >< Nat -> Bool
axiom forall t: T, x, y, z: Nat :-
  put( put(t, x), y ) is put(t,x) pre y <= x,
  ~get( empty, x ),
  get( put(empty, x), y ) is (x = y),
  get(put(put(empty,x),y),z) is (z = x \/ z = y /\ y > x),
  get( put(t,x), y ) is get(t,y) pre y ~= x,
  ~get( put(t, 2*y), 2*y ) pre get(t,y) /\ get(t, y+1),
  get( put(put(t, 0), x), x ) is get( put(t,x), x )
\end{lstlisting}

\z Противоречива ли следующая спецификация?
\begin{lstlisting}
type T
value empty: T,
    put: T >< Nat -> T,
    get: T >< Nat -> Bool
axiom forall t: T, x, y, z: Nat :-
  put( put(t,x), x ) is put(t,x),
  ~get( empty, x ),
  get( put(empty, x), y ) is (x = y),
  get(put(put(empty,x),y),z) is (z=x \/ z=y /\ y>2),
  get( put(t,x), y ) is get(t, y) pre y ~= x,
  get( put(t,y), y ) pre get( put(t,x), x ) /\ x <= y,
  ~get(t,x) pre get( put(t, 0), 0),
  ~get(t, x) /\ ~get(t,y) pre x ~= y /\ get(put(t,1),1)
\end{lstlisting}

\z Противоречива ли следующая спецификация?
\begin{lstlisting}
type T
value empty: T,
    put: T >< Nat -> T,
    get: T >< Nat -> Bool
axiom forall t: T, x, y, z: Nat :-
  put( put(t,x), x ) is put(t,x),
  ~get( empty, x ),
  get( put(empty, x), y ) is (x = y),
  get( put( put(empty,x), y ), z ) is (z = x \/ z = y /\ y > 0),
  get( put(t,x), y ) is get(t, y) pre y ~= x,
  get( put(t,y), y ) pre get( put(t,x), x ) /\ x <= y,
  ~get(t,x) pre get( put(t, 0), 0) /\ ~get(t,0),
  ~get(t,x) \/ ~get(t,y) pre x~=y /\ get(put(t,1),1) /\ ~get(t,1)
\end{lstlisting}

\zhead{Полнота алгебраических спецификаций}

Не забывайте, что у нас есть только система аксиом и логика, т.е. правила получения новых выражений из имеющихся. За символами имен операций не стоит никакой семантики, даже если она <<предполагалась>> автором. Наоборот, эта система аксиом должна \emph{дать} нам семантику символов, т.е. дать нам возможность сделать с этими символами некие осмысленные действия.

\z Полно ли описывает следующая спецификация тип <<Множество>> ?
\begin{lstlisting}
type E, S
value empty: S,
      add: E >< S -> S,
axiom forall e1, e2: E, s : S :-
   add(e1, add(e1, s)) is add(e1, s),
   add(e1, add(e2, s)) is add(e2, add(e1, s))
\end{lstlisting}

\textbf{Решение:}
В этой спецификации нет ни одного обсервера, поэтому вопрос о полноте для нее некорректен.

\z Полна ли следующая спецификация типа <<Очередь>> ?
\begin{lstlisting}
type E, Q
value empty: Q,
      add: E >< Q -> Q,
      size: Q -> Nat
axiom forall e: E, q : Q :-
    size(empty) is 0,
    size(add(q, e)) is size(q) + 1
\end{lstlisting}

\textbf{Решение:}
Обсервер --- size. Он определен для empty и определен для генератора add. Значит, он определен для любого терма, дающего тип Q. Значит, функция size описана полно.

\z Полна ли следующая спецификация типа <<Очередь>> ?
\begin{lstlisting}
type E, Q
value empty: Q,
      add: Q >< E -> Q,
      size: Q -> Nat
axiom forall e: E, q : Q :-
    size(empty) is 0,
    size(add(q, e)) is size(q) + 1,
    add(add(q, e), e) is add(q, e)
\end{lstlisting}

\textbf{Решение:}
Без учета последней аксиомы size(add(add(q,e),e)) is size(add(q,e)) + 1 is size(q) + 2. А теперь с последней аксиомой: size(add(add(q,e),e)) is size(add(q,e)) is size(q) + 1. Получается, что size(q) + 1 = size(q) + 2. Иными словами, из аксиом следует ложь, система аксиом противоречива. А раз так, вопрос о полноте некорректен.


\z Полна ли следующая спецификация типа <<Очередь>> ?
\begin{lstlisting}
type E, Q
value empty: Q,
      add: Q >< E -> Q,
      first: Q -~-> E,
      size: Q -> Nat
axiom forall e: E, q : Q :-
    first(add(empty, e)) is e,
    first(add(q, e)) is first(q) pre q ~= empty,
    size(empty) is 0,
    size(add(q,e)) is size(q) + 1
\end{lstlisting}

\textbf{Решение:}
Любое значение в целевом типе Q имеет один из двух видов (просто напросто, нет других функций, возвращающих Q):
\begin{itemize}
  \item empty
  \item add(add(add(...add(add(empty, e1), e2)...)
\end{itemize}

Посмотрим на first с точки зрения определения достаточной полноты. Обе аксиомы для простоты можно объединить в одну: first(add(q,e)) is if q = empty then e else first(q) end. Тогда такое выражение <<вычислить>> можно: first(add(empty,e1)) is e1 (по первой аксиоме). Посмотрим такое выражение: first(add(add(empty,e1),e2)) is if add(empty,e1) = empty then e2 else e1 end. Чтобы закончить <<вычисление>>, надо понять истинность выражения add(empty,e1) = empty. В общем случае дать ответ на этот вопрос нельзя (\emph{проблема равенства термов алгоритмически неразрешима}). Однако в данном случае ответить на этот вопрос можно.

Для этого сделаем такой хитрый ход -- \emph{<<навесим>> на эти два выражения сверху другой обсервер}: size(empty) is 0, size(add(empty,e1)) is size(empty) + 1 is 0 + 1 is 1, т.е. size(empty) ~= size(add(empty,e1)). Поскольку size --- тотальная функция, то она детерминированная, т.е. all x, y : Q :- x = y => size(x) = size(y), что то же самое, что size(x) ~= size(y) => x ~= y. Теперь в качестве x возьмем empty, а в качестве y возьмем add(empty, e1). Получим, что add(empty, e1) ~= empty. Ура, желаемое доказано! Тем самым, можно и вычислить второй терм, он равен e1. Аналогично, можно вычислить и все остальные термы.

Осталось рассмотреть единственный терм: first(empty). Если бы существовали такие q' и e', что add(q',e') равнялось empty, то было бы возможно применение аксиом. Но, как следует из первой части, такие q' и e' не существуют, значит, <<вычислить>> first(empty) на основе данных аксиом нельзя. Ответ: неполна.

\z Полна ли следующая спецификация типа <<Очередь>> ?
\begin{lstlisting}
type E, Q
value empty: Q,
      add: Q >< E -> Q,
      first: Q -~-> E,
axiom forall e: E, q : Q :-
    first(add(empty, e)) is e,
    first(add(q, e)) is first(q) pre q ~= empty,
\end{lstlisting}

\textbf{Решение:}
Рассуждая аналогично предыдущей задаче, приходим к вопросу об истинности add(empty,e1) = empty и в данной системе аксиом дать ответ на этот вопрос нельзя. Ответ: неполна.

%% вставить примеры противоречивых систем аксиом из заданий экзамена прошлого года

%% понять, как описывать недопустимое поведение

%\zhead{Спецификация отношений <<многие-ко-многим>>}
%
%Алгебраические спецификации позволяют описать многие компоненты, осуществляющие отношение <<многие-ко-многим>>, совершенно не задумываясь о том, каким образом это отношение выразить чем-нибудь более известным (например, вспомните, сколько есть различных способов представления этого отношения для реляционной модели данных!)
%
%\z Специфицируйте компонент, отвечающий за хранение и модификацию данных о студентах Университета и спецкурсах. А именно, есть студенты, они добавляются в базу внутри этого компонента. Есть спецкурсы, которые также добавляются. Как-то студенты записываются на спецкурсы. Компонент позволяет

\zhead{Рекурсивные типы}

\z Специфицируйте операцию проверки вхождения элемента в бинарное дерево.

\textbf{Решение:}
\begin{lstlisting}
type Node, Tree == empty | add(Node, Tree, Tree)
value check: Node >< Tree -> Bool
axiom forall n, n1:Node, left, right: Tree :-
   ~check( empty ),
   check(n, add(n1,left,right) ) is (n = n1) \/
             check(n, left) \/ check(n, right)
\end{lstlisting}

\z Специфицируйте операцию вычисления высоты бинарного дерева.

\z Специфицируйте операцию проверки бинарного дерева на сбалансированность.

\z Специфицируйте операцию получения предка элемента бинарного дерева.

\z Специфицируйте добавление элемента в двоичное дерево поиска.

\z Специфицируйте удаление элемента из двоичного дерева поиска.

\z Специфицируйте добавление элемента в АВЛ-дерево. Определение операции предполагается найти самостоятельно.

\z Специфицируйте удаление элемента из АВЛ-дерева. Определение операции предполагается найти самостоятельно.

\z Специфицируйте добавление элемента в 2-3-дерево. Определение операции предполагается найти самостоятельно.

\z Специфицируйте удаление элемента из 2-3-дерева. Определение операции предполагается найти самостоятельно.

\z Специфицируйте добавление элемента в декартово дерево. Определение операции предполагается найти самостоятельно.

\z Специфицируйте удаление элемента из декартова дерева. Определение операции предполагается найти самостоятельно.

\z Специфицируйте добавление элемента в красно-чёрное дерево.  Определение операции предполагается найти самостоятельно.

\z Специфицируйте удаление элемента из красно-чёрного дерева.  Определение операции предполагается найти самостоятельно.

