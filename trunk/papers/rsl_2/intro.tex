% !Mode:: "TeX:UTF-8"
\pagebreak
\section*{Введение}
\addcontentsline{toc}{chapter}{Введение}

Данный сборник задач написан в поддержку курса <<Формальная спецификация и верификация программ>>, который читается студентам последних курсов факультета ВМиК МГУ.

Написать про 2 основных применения формальных спецификаций:
\begin{enumerate}
  \item уточнение требований (выявление и четкое формулирование концептуальных моментов того, как себя ведет программная система);
  \item один из этапов формальной верификации.
\end{enumerate}

На самом деле есть ещё третье применение: реализация через уточнение моделей, но сейчас он очень редко применяется.

%Под <<формальной>> спецификацией понимается строгое однозначное задание (описание) интерфейса или поведения программы. До сих пор необходимость доведения описаний до строгих однозначных форм ставится под сомнение, если речь идет о совершенно произвольных программах (как минимум, это сталкивается с высокой трудоемкостью формальной спецификации и особенной квалификацией тех, кто эту спецификацию составляет). Хотя полезность (и даже необходимость) строгого однозначного задания \emph{критичных} систем сомнений не вызывает. И тем не менее понижение трудоемкости и сближение формальных спецификаций с программистами-инженерами (т.е. существенное расширение области реального использования формальных спецификаций) является актуальной задачей в области технологий программирования (software engineering).
%
%Однако дабы не попадать в дискуссионную область, курс следует иному принципу. Реалии таковы, что кроме написания программ необходимо, чтобы эти программы были корректными, чтобы они удовлетворяли стандартам. Для решения задач обеспечения таких характеристик применяются \emph{в том числе и} математические методы. Это означает, что программа выражается в математических терминах в виде \emph{математической теории}, или \emph{математической модели}, и задача уже решается в рамках этой математической теории с применением математического аппарата. Эта идея может показаться малоприменимой на практике, поскольку обычно математики и программисты-инженеры живут <<в разных мирах>>. На самом же деле математические методы решения задач над программами (их еще называют \emph{формальными методами}, подчеркивая, что <<обычный>> программист-инженер работает в своем <<неформальном>> мире представления о своей программе) исторически возникли практически сразу с возникновением практического программирования (это 50-е годы ХХ века) и развиваются по настоящее время.
%
%Математическая теория, создаваемая для программы, это и есть формальная спецификация. От природы этой спецификации будут зависеть и математические методы, применяемые для решения задачи. Математическая теория не создается сама по себе --- она создается для конкретных целей, для решения определенных задач: формализация требований с целью, во-первых, их прояснения, во-вторых, для выяснения в них противоречий и неполных требований, автоматизация тестирования, чёткая документация, формальная верификация и даже разработка программ при помощи формальных моделей. Единожды проведя формализацию, можно существенно снизить <<человеческий>> фактор на последующих этапах жизненного цикла программы.
%
%Эта часть курса посвящена тому, какие на данный момент придуманы виды моделей, какой природы математические теории используются для описания программ. Вторая часть курса посвящена одному из применений формальных спецификаций --- формальной верификации программ.
%
%Читатели могут столкнуться с <<моделями программ>> не впервые. Студенты ВМиК МГУ слушают перед этим курсом курс по объектно-ориентированному анализу и проектированию программ и курс по верификации программ на моделях (model checking). Отличия этого курса от уже прослушанных заключаются в следующем. Курс ООАП также работает с моделями, но многие из этих моделей ориентированы только на последующее кодирование, а не на анализ программ. Грубо говоря, речь идет о моделировании структуры кода, а не семантики программы. Кроме того, строгий, формальный, подход практически никак не отражен в этом курсе. В курсе верификации на моделях рассматривается инструмент SPIN и моделирование на языке PROMELA. Остальные виды моделей программ в этом курсе не рассматриваются, но рассматриваются в данном курсе.
%
%Согласно одной из принятых классификаций выделяют следующие основные виды моделей программ:
%\begin{itemize}
%  \item логико-алгебраические модели (interface specification: property-based / state-based);
%  \item исполнимые модели (behavior specification);
%\end{itemize}
%Кроме того, выделяют модели, совмещающие в себя характеристики логико-алгебраических и исполнимых моделей.
%
%Исполнимые спецификации дают модель в виде программы для некоторой виртуальной машины, может быть, достаточно абстрактной. В основном, это различные виды конечных автоматов и систем переходов (LTS). К таким моделям относятся модели на PROMELA, уже знакомые читателям. Кроме того, с конечными автоматами они сталкивались достаточно часто в предыдущих курсах. Поэтому в этом курсе исполнимые модели не будут рассматриваться подробно.
%
%Логико-алгебраические модели рассматривают операции программы в математическом смысле, как отображения аргументов и пре-состояния на значения-результаты операций и пост-состояния\footnote{потому такие модели не являются исполнимыми в общем случае --- попробуйте для любой функции, заданной отображением, автоматически построить программу, которая ее исполняет!}. Чистые \emph{логические модели} представляют собой набор аксиом, из которых следуют эти отображения. \emph{Алгебраические модели} описывают эквивалентности суперпозиций операций (грубо говоря, эти модели состоят из требований эквивалентности разных термов--цепочек действий). К неисполнимым спецификациями принадлежат и такие виды моделей как \emph{программные контракты} --- набор логических свойств, которые должны быть выполнены при корректных входных данных и вычисленных по ним выходных. Грубо говоря, для задания семантики программы в неисполнимом виде применяются два подхода: <<чистый операционный>> (функциональный) подход (property-based) и подход, основанный на моделировании состояния программы (model-based, state-based). В функциональном подходе состояние не моделируется! И тем не менее, семантику операций удается задать. Вторая глава задачника посвящена функциональному подходу. Третья глава --- подходу, основанному на моделировании состояния программы. А первая глава посвящена тому языку, на котором все эти модели можно выражать --- языку RSL. Авторы языка попытались создать язык, который был бы языком программирования и языком спецификации одновременно\footnote{На самом деле обе эти цели можно воспринимать как моделирование --- первое является исполнимым моделированием, а второе неисполнимым.}. Единый языка выражения программы и ее семантики позволяют легче провести обучение языку и верификацию программы на такой модели.
%
%На практике для реализации алгоритмов в большинстве случаев всё же язык RSL не применяют. Тем не менее успешно применяемые языки спецификации проектируются таким образом, чтобы приблизить их к языкам программирования (что как минимум облегчит верификацию реализации). Язык RSL нацелен на такие языки программирования, как Си и Ада, за исключением некоторых их особенностей (указателей и динамической памяти). Языки спецификации имеют определенную область применения, где их использование наиболее эффективно, и при решении практических задач стоит аккуратно подбирать язык спецификации с учетом целевого языка программирования. Поэтому читателя не должно смущать в дальнейшем изложении отсутствие объектно-ориентированных возможностей, механизмов распараллеливания вычислений --- для спецификации этих возможностей подходят другие языки спецификации. 