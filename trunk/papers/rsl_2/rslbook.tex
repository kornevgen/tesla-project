% !Mode:: "TeX:UTF-8"
\documentclass[14pt, twoside]{extreport}
\usepackage{cmap}

%\usepackage{fix-cm}
\usepackage[utf8]{inputenx}
\usepackage[russian]{babel}
%%%%%%%%%%%%%%%%%%%%%%%%%%%%\usepackage{pscyr}
%\usepackage[T1]{fontenc} %cm-super
%\usepackage{type1cm}
\usepackage{indentfirst}
\usepackage{amsthm}
\usepackage{amssymb}
\usepackage{amsmath}
%\usepackage{dsfont}
\usepackage{xspace}
\usepackage[numbers,compress,sort]{natbib}
\usepackage{clrscode}

\pagestyle{headings}

\textheight 23cm % 29.7-2-2
\textwidth 16cm % 21-2.5-1.5
\hoffset 0.46cm %2.5-2.54 слева 3 см
\voffset -0.54cm %2-2.54 сверху 2 см
\oddsidemargin 0cm \evensidemargin 0cm  \headheight 0cm \headsep 1.5cm \topmargin 0cm

\usepackage{ccaption} % заменяем для рисунков ':' после номера рисунка на другой символ
\captiondelim{. } % разделитель точка и пробел

\usepackage{ifpdf}

\ifpdf
% we are running pdflatex, so convert .eps files to .pdf
% run pdflatex with --shell-escape and thesis.aux
\usepackage[pdftex]{graphicx}
\usepackage{epstopdf}
\else
% we are running LaTeX, not pdflatex
\usepackage{graphicx}
\fi

% Подправим команду \appendix : нумерация русскими буквами,
% а не латинскими.
\makeatletter
\renewcommand\appendix{\par
  \setcounter{chapter}{0}%
  \setcounter{section}{0}%
  \def\@chapapp{\appendixname}%
  \def\thechapter{\@Asbuk\c@chapter}}
\makeatother

% "русифицируем" окружение enumerate:
\makeatletter
\def\labelenumi{\theenumi)}      % чтобы после номера шла скобка;
\def\theenumii{\@asbuk\c@enumii}   % чтобы на втором уровне шли русские,
\def\labelenumii{\theenumii)}    % а не латинские буквы
\def\p@enumii{\theenumi}         % а это для \ref
\def\labelenumiii{{\bf--}}       % а на третьем уровне пусть будут лишь тире,
\let\theenumiii\relax            % и отдельных ссылок на него не будет
\def\p@enumiii{\theenumi\theenumii}
\makeatother

\usepackage{rsl}

\usepackage{listingsutf8}
\lstloadlanguages{RSL}
\lstset{numbers=left, language=RSL, extendedchars=true, numberstyle=\tiny}
%, inputencoding=utf8/latin1, commentstyle=\itshape, stringstyle=\bfseries}

\author{Евгений Корныхин}
\title{\huge{\textbf{\textsc{Задачи по формальной спецификации программ на RSL}}}}
%\date{Москва --- 2010}

\newcounter{problem_type}[chapter]
\newcounter{zadacha}[problem_type]
\newcommand{\z}{\vspace{0.5cm}\par\addtocounter{zadacha}{1}%
\textit{\arabic{chapter}.\arabic{problem_type}.\arabic{zadacha}}~~  }

\newcommand{\head}[1]{\vspace{1cm}\subsubsection*{#1}}
\newcommand{\zhead}[1]{\head{#1} \refstepcounter{problem_type}}


\begin{document}

\maketitle

\tableofcontents

\chapter*{Введение}
\addcontentsline{toc}{chapter}{Введение}

Данный сборник задач написан в поддержку курса <<Формальной спецификации и верификации программ>>, который читается студентам последних курсов факультета ВМиК МГУ.

Под <<формальной>> спецификацией в первую очередь понимается строгое однозначное задание (описание) интерфейса или поведения программы. До сих пор необходимость доведения описаний до строгих однозначных форм ставится под сомнение, если речь идет о совершенно произвольных программах (как минимум, это сталкивается с высокой трудоемкостью формальной спецификации и особенной квалификацией тех, кто эту спецификацию составляет). Хотя полезность (и даже необходимость) строгого однозначного задания \emph{критичных} систем сомнений не вызывает. И тем не менее понижение трудоемкости и сближение формальных спецификаций с программистами-инженерами (т.е. существенное расширение области реального использования формальных спецификаций) является актуальной задачей в области технологий программирования (software engineering).

Однако дабы не попадать в дискуссионную область, курс следует иному принципу. Реалии таковы, что кроме написания программ необходимо, чтобы эти программы были корректными, чтобы они удовлетворяли стандартам. Для решения задач обеспечения таких характеристик применяются \emph{в том числе и} математические методы. Это означает, что программа выражается в математических терминах в виде \emph{математической теории}, или \emph{математической модели}, и задача уже решается в рамках этой математической теории с применением математического аппарата. Эта идея может показаться малоприменимой на практике, поскольку обычно математики и программисты-инженеры живут <<в разных мирах>>. На самом же деле математические методы решения задач над программами (их еще называют \emph{формальными методами}, подчеркивая, что <<обычный>> программист-инженер работает в своем <<неформальном>> мире представления о своей программе) исторически возникли практически сразу с возникновением практического программирования (это 50-е годы ХХ века) и развиваются по настоящее время.

Математическая теория, создаваемая для программы, это и есть формальная спецификация. От природы этой спецификации будут зависеть и математические методы, применяемые для решения задачи. Математическая теория не создается сама по себе --- она создается для конкретных целей, для решения определенных задач: формализация требований с целью, во-первых, их прояснения, во-вторых, для выяснения в них противоречий и неполных требований, автоматизация тестирования, чёткая документация, формальная верификация и даже разработка программ при помощи формальных моделей. Единожды проведя формализацию, можно существенно снизить <<человеческий>> фактор на последующих этапах жизненного цикла программы.

Эта часть курса посвящена тому, какие на данный момент придуманы виды моделей, какой природы математические теории используются для описания программ. Вторая часть курса (не вошедшая в этот сборник задач) посвящена одному из применений формальных спецификаций --- формальной верификации программ.

Читатели могут столкнуться с <<моделями программ>> не впервые. Студенты ВМиК МГУ слушают перед этим курсом курс по объектно-ориентированному анализу и проектированию программ и курс по верификации программ на моделях (model checking). Отличия этого курса от уже прослушанных заключаются в следующем. Курс ООАП также работает с моделями, но многие из этих моделей ориентированы только на последующее кодирование, а не на анализ программ. Грубо говоря, речь идет о моделировании структуры кода, а не семантики программы. Кроме того, строгий, формальный, подход практически никак не отражен в этом курсе. В курсе верификации на моделях рассматривается инструмент SPIN и моделирование на языке PROMELA. Остальные виды моделей программ в этом курсе не рассматриваются, но рассматриваются в данном курсе.

Согласно одной из принятых классификаций выделяют следующие основные виды моделей программ:
\begin{itemize}
  \item логико-алгебраические модели (interface specification: property-based / state-based);
  \item исполнимые модели (behavior specification);
\end{itemize}
Кроме того, выделяют модели, совмещающие в себя характеристики логико-алгебраических и исполнимых моделей.

Исполнимые спецификации дают модель в виде программы для некоторой виртуальной машины, может быть, достаточно абстрактной. В основном, это различные виды конечных автоматов и систем переходов (LTS). К таким моделям относятся модели на PROMELA, уже знакомые читателям. Кроме того, с конечными автоматами они сталкивались достаточно часто в предыдущих курсах. Поэтому в этом курсе исполнимые модели не будут рассматриваться подробно.

Логико-алгебраические модели рассматривают операции программы в математическом смысле, как отображения аргументов и пре-состояния на значения-результаты операций и пост-состояния\footnote{потому такие модели не являются исполнимыми в общем случае --- попробуйте для любой функции, заданной отображением, автоматически построить программу, которая ее исполняет!}. Чистые \emph{логические модели} представляют собой набор аксиом, из которых следуют эти отображения. \emph{Алгебраические модели} описывают эквивалентности суперпозиций операций (грубо говоря, эти модели состоят из требований эквивалентности разных термов--цепочек действий). К неисполнимым спецификациями принадлежат и такие виды моделей как \emph{программные контракты} --- набор логических свойств, которые должны быть выполнены при корректных входных данных и вычисленных по ним выходных. Грубо говоря, для задания семантики программы в неисполнимом виде применяются два подхода: <<чистый операционный>> (функциональный) подход (property-based) и подход, основанный на моделировании состояния программы (model-based, state-based). В функциональном подходе состояние не моделируется! И тем не менее, семантику операций удается задать. Вторая глава задачника посвящена функциональному подходу. Третья глава --- подходу, основанному на моделировании состояния программы. А первая глава посвящена тому языку, на котором все эти модели можно выражать --- языку RSL. Авторы языка попытались создать язык, который был бы языком программирования и языком спецификации одновременно\footnote{На самом деле обе эти цели можно воспринимать как моделирование --- первое является исполнимым моделированием, а второе неисполнимым.}. Единый языка выражения программы и ее семантики позволяют легче провести верификацию программы на такой модели. Но вопросы верификации лежат уже за пределами данного сборника задач.

\chapter{RSL для императивного программирования}

% !Mode:: "TeX:UTF-8"

\head{Описание сигнатуры функции на RSL}

\begin{lstlisting}
value add: Int >< Int -~-> Int
\end{lstlisting}
Имеется одна функция add с двумя аргументами типа $\Int$ (аргументы разделяются символом $\NonDetermFn$). Функция вычисляет одно значение и это значение типа $\Int$ (несколько значений так же разделяются символом $\times$). Функция не имеет побочного эффекта.

\head{Описание алгоритма целиком}
\begin{lstlisting}
value add: Int >< Int -~-> Int
add(x,y) is x+y
\end{lstlisting}
После сигнатуры идет тело функции. Вначале идет имя функции с формальными параметрами, затем символ $\Is$ и затем выражение. Вычислением функции является вычисление этого выражения (в данном случае, сложение двух чисел-аргументов).

\head{Встроенные типы}
\textbf{Int}, \textbf{Nat}, \textbf{Real}, \textbf{Bool}, \textbf{Char}, \textbf{Text}, \textbf{Unit}.

Тип \textbf{Int} содержит в себе все возможные целые числа. Ограничений на их значения (типа MAXINT) нет.

Тип \textbf{Nat} содержит число 0 и все возможные положительные целые числа. Ограничений сверху на их значения нет. Тип поддерживает все операции, определенные для типа \textbf{Int}.

Тип \textbf{Real} содержит в себе все возможные вещественные числа. Важно понимать, что эти числа являются математической абстракцией тех вещественных чисел, которые представимы в архитектуре компьютера. Тип \textbf{Real} --- это те числа, с которыми работают математики. Следовательно, они включают и все вещественные числа, представимые в какой-угодно архитектуре компьютера. Например, в этом типе есть число <<квадратный корень из двух>>.

Тип \textbf{Bool} содержит булевские значения \textbf{true} и \textbf{false}.

Тип \textbf{Char} содержит все возможные отдельные символы. Этот тип не привязан ни к одной из кодировок (поскольку этот тип --- лишь математическая абстракция). Этот тип содержит все мыслимые символы. Поэтому не определена операция получения <<кода символа>>, привычная для многих языков программирования.

Тип \textbf{Text} является массивом символов (о массивах см.ниже).

Тип \textbf{Unit} является специальным и используется для ограниченного количества случаев (см.ниже). Основное применение --- то же, какое имеет ключевое слово \texttt{void} в сигнатурах функций языка Си.

\head{Операции над встроенными типами}
Арифметические:
\begin{lstlisting}
value
  +: Int >< Int -> Int,
  -: Int >< Int -> Int,
  *: Int >< Int -> Int,
  /: Int >< Int -~-> Int,
  \: Int >< Int -~-> Int,
  **: Int >< Int -~-> Int,
  abs: Int -> Nat,
  real: Int -> Real,

  +: Real >< Real -> Real,
  -: Real >< Real -> Real,
  *: Real >< Real -> Real,
  /: Real >< Real -~-> Real,
  **: Real >< Real -~-> Real,
  abs: Real -> Real,
  int: Real -> Int,

  <: Int >< Int -> Bool,
  <=: Int >< Int -> Bool,
  >: Int >< Int -> Bool,
  >=: Int >< Int -> Bool,

  <: Real >< Real -> Bool,
  <=: Real >< Real -> Bool,
  >: Real >< Real -> Bool,
  >=: Real >< Real -> Bool,

  ~: Bool       -> Bool,
  /\: Bool >< Bool -> Bool,
  \/: Bool >< Bool -> Bool,
  =>: Bool >< Bool -> Bool,
\end{lstlisting}

Для любого типа определены операции сравнения на равенство:
\begin{lstlisting}
type T
value
  = : T >< T -> Bool,
  ~= : T >< T -> Bool
\end{lstlisting}

Порядок вычисления операций строго определен: сначала первый аргумент, затем, если необходимо, второй и т.д.

Логика короткая. Это означает, например, что если первый аргумент конъюнкции равен \textbf{false}, то второй аргумент не вычисляется и вся конъюнкция принимает значение \textbf{false}.

\head{Глобальные переменные}
Кроме своих аргументов функция может оперировать глобальными переменными. Каждая глобальная переменная должна быть определена в разделе \textbf{variable}, а в сигнатуре функции должен быть указан режим работы функции с переменной: по чтению или по записи-чтению. Функции разрешено оперировать лишь с теми глобальными переменными, которые указаны в сигнатуре. Например,
\begin{lstlisting}
variable status : Text
value
  init : Unit -~-> write status Unit
  init() is (status := "initialized")	
\end{lstlisting}

Функция \texttt{init} не имеет аргументов --- для указания этого факта перед стрелкой в сигнатуре функции указан тип \textbf{Unit}. Также у функции нет и возвращаемого значения --- она лишь изменяет значение глобальной переменной \texttt{status}. Этот факт также указан типом \textbf{Unit} в качестве типа возвращаемого значения.

Еще пример:
\begin{lstlisting}[escapechar={|}]
variable status : Text
value
  |is\_initialized| : Unit -~-> read status Bool
  |is\_initialized|() is
	(status = "initialized")	
\end{lstlisting}

Здесь глобальная переменная лишь читается в функции, поэтому в сигнатуре переменная \texttt{status} указана с модификатором \textbf{read}.

Можно указать в сигнатуре, что функция может читать или изменять любую глобальную переменную. В этом случае вместо имени переменной в сигнатуре функции надо написать ключевое слово \textbf{any}.

\head{Выражения}
Тело функции является выражением того типа, какой должна возвращать функция. Например, функция
\begin{lstlisting}
value f: Int -~-> Int
      f(x) is
	       x+1
\end{lstlisting}
возвращает значение типа \textbf{Int}. Ее тело состоит из суммы целых чисел, а сумма возвращает значение типа Int. Этой \textbf{математической} функции соответствует, например, следующая функция на языке Си:
\begin{lstlisting}
int f( int x )
{
    return x+1;
}
\end{lstlisting}

Однако стоит понимать следующее различие: в языке Си значения практически всех типов ограничены, а в языке RSL неограничены. Поэтому вычисление выражения x+1 в программе на Си может привести к переполнению и, поэтому, неверному с точки зрения математического определения операции <<+1>> результату.

Кроме возврата значения такого простого выражения язык RSL допускает более сложные управляющие структуры. Они составляются из более простых выражений. К таким простым выражениям относятся:
\begin{itemize}
\item встроенные операции:
\begin{lstlisting}
value f: Int -~-> Int
      f(x) is x+1
\end{lstlisting}

\item вызов другой функции (со статической проверкой типов аргументов):
\begin{lstlisting}
value g: Int -~-> Int
      g(x) is 2 * f(x-1)
\end{lstlisting}

\item условный оператор: \textbf{if} логическое выражение \textbf{then} выражение1 \textbf{else} выражение2 \textbf{end} --- типы выражений <<выражение1>> и <<выражение2>> должны совпадать --- условный оператор возвращает значение этого же типа (значение одного из выражений, в зависимости от значения логического выражения):
\begin{lstlisting}
value
   Abs: Int -~-> Nat
   Abs(x) is if x > 0 then x else -x end
\end{lstlisting}

Условный оператор без \textbf{else} --- он допустим только в случае, если <<выражение1>> имеет тип \textbf{Unit}:
\begin{lstlisting}
variable status: Text
value
   init: Bool -~-> write status Unit
   init(needed) is
     if needed then status := "initialized" end
\end{lstlisting}

Условный оператор с несколькими ветками:
\begin{lstlisting}
variable status: Text
value
   op: Int >< Int -~-> read status Int
   op(x,y) is
	if status = "sum" then
               x + y
        elsif status = "mul" then
               x * y
	else
		0
	end
\end{lstlisting}

\item оператор присваивания: используется для изменения значений глобальных переменных (аргументы функции переменными не являются, ими лишь поименованы значения)
\begin{lstlisting}
variable status : Text
value
  init : Unit -~-> write status Unit
  init() is (status := "initialized")	
\end{lstlisting}

Оператор присваивания имеет тип \textbf{Unit}.

\item последовательность операторов: через точку с запятой
\begin{lstlisting}
variable status : Text
value
  next : Int -~-> write status Int
  next() is
	status := "moved";
	x+1	
\end{lstlisting}
В этой функции сначала изменяется значение переменной \texttt{status}, а затем вычисляется выражение x+1. Значением последовательности операторов является значение последнего оператора. А все остальные элементы должны иметь тип \textbf{Unit}. Поэтому, например, следующая запись будет некорректной:
\begin{lstlisting}
variable status : Text
value
  next : Int -~-> write status Int
  next() is
        x+1;
	status := "moved"
\end{lstlisting}

\item операторы циклов: while, do-until и for
\begin{lstlisting}
variable a: Nat, b : Nat
value
  euclid: Unit -~-> write a, b Nat
  euclid() is
     while a > 0 /\ b > 0 do
        if a > b then a := a - b
        else b := b - a end
     end;
     if a = 0 then b else a end
\end{lstlisting}

\begin{lstlisting}
variable n: Nat, x : Nat
value
  digits: Unit -~-> write n, x Nat
  digits() is
     n := 0;
     do n := n + 1; x := x / 10
     while x = 0;
     n
\end{lstlisting}

\begin{lstlisting}
variable sum : Nat
value
  sumN: Nat -~-> write sum Unit
  sumN(n) is
	sum := 0;
        for i in <.1 .. n.> do
	  sum := sum + i
	end;
\end{lstlisting}

\begin{lstlisting}
variable sum : Nat
value
  sumN: Nat -~-> write sum Unit
  sumN(n) is
	sum := 0; i := 0;
    do
      i := i + 1;
      sum := sum + i
    until i = n	end;
\end{lstlisting}

Способов прервать цикл типа break в RSL нет.

\item локальные переменные:
\begin{lstlisting}
variable a: Nat, b : Nat
value
  euclid: Unit -~-> write a, b Nat
  euclid() is
   local variable a1: Nat := a,
                  b1: Nat := b in
     while a1 > 0 /\ b1 > 0 do
        if a1 > b1 then a1 := a1 - b1
        else b1 := b1 - a1 end
     end;
     if a1 = 0 then b1 else a1 end
   end
\end{lstlisting}

\item выражение \textbf{let}: не вводит локальные переменные! а вводит новые синонимы значений; кроме того, выполняет сопоставление с образцами (pattern matching):
явный \textbf{let}:
\begin{lstlisting}
value simple:  Int -~-> Int
  simple(x) is
    let y = abs x in y-x end
\end{lstlisting}

Выражение в \textbf{let} вычисляется один раз перед вычислением <<тела>> оператора \textbf{let}.

неявный \textbf{let}:
\begin{lstlisting}
value
  random: Nat -~-> Nat
  random(n) is
      let r : Nat :- r <= n in
	   r
      end	
\end{lstlisting}

\end{itemize}



\head{Массивы}
Следующая функция использует массив целых чисел:
\begin{lstlisting}
value sort: Int-list -~-> Int-list
\end{lstlisting}

Обращение по индексу делается так: A(1). Нумерация индексов \textbf{с единицы}. Операция \textbf{len} возвращает текущую длину массива. Массивы в RSL могут изменять длину (путем операции конкатенации). Индекс не должен иметь большее значение, чем длина массива, и меньшее, чем 1. Синтаксис RSL запрещает изменять значения отдельных элементов массивов (например, так: <<A(i) := i>>), можно только строить новые массивы целиком (например, вместо <<A(i) := e>> можно писать <<A := $\langle$ if k=i then e else A(k) end | k in $\langle$1 .. \textbf{len} A$\rangle$ $\rangle$>>).

\begin{lstlisting}
value
  sum: Int-list -~-> Int
  sum(ls) is
    local variable s : Int := 0 in
       for i in <.1 .. len ls.> do
             s := s + ls(i)
       end;
       s
    end;
\end{lstlisting}


Эту же функцию, суммирующую элементы массива, можно записать и без использования индексов:
\begin{lstlisting}
value
  sum: Int-list -~-> Int
  sum(ls) is
    local variable s : Int := 0 in
       for l in ls do
             s := s + l
       end;
       s
    end;
\end{lstlisting}

Тип \textbf{Text} является массивом из \textbf{Char}.

Операция \textbf{tl} возвращает <<подмассив>> --- от второго элемента до последнего элемента исходного массива. Операция определена только для непустых массивов. Пустой список обозначается символом <..>. Тем самым функцию sum можно записать следующим образом с использованием рекурсии:
\begin{lstlisting}
value
  sum: Int-list -~-> Int
  sum(ls) is
    if ls = <..> then 0
    else ls(1) + sum(tl ls) end
\end{lstlisting}


\head{Структуры}
\begin{lstlisting}
type FIO ::
         name : Text
         surname : Text
value
   hello: FIO -~-> Text
   hello(fio) is
      "Hello, " ^ name(fio) ^
           "  " ^ surname(fio) ^ "!"
\end{lstlisting}

Обращение к полю делается в виде вызова функции с именем поля.

\begin{lstlisting}
type FIO ::
         name : Text
         surname : Text
value
   new_fio: Text >< Text -~-> FIO
   new_fio(n,sn) is mk_FIO(n,sn)
\end{lstlisting}

Создание структуры делается с помощью функции с предопределенным именем. Это имя начинается с <<mk\_>>, за которым идет имя типа структуры. В скобках подряд перечисляются выражения, дающие значения полям новой структуры.

\head{Перечисления (enumeration)}
\begin{lstlisting}
type Color = red | blue | white | green
value
  from_rus : Color -~-> Bool
  from_rus(c) is c = red \/ c = blue \/ c = white
\end{lstlisting}

\head{Указатели}
RSL не содержит встроенных механизмов для задания алгоритмов, работающих с указателями.


\section*{Задачи}

% !Mode:: "TeX:UTF-8"
\zhead{Выбор генераторов}

\z Для стека, определенного таким образом:
\begin{lstlisting}
type E, S == empty | push(S, E) | pop(S, E)
\end{lstlisting}
дать алгебраическую спецификацию операции проверки наличия заданного элемента
\begin{lstlisting}
value include: S >< E -> Bool
\end{lstlisting}

\textbf{Решение:}
\begin{lstlisting}
type E, S == empty | push(S, E) | pop(S)
value include: S >< E -> Bool
axiom forall e,e1:E, s:S :-
  ~ include(e, empty),
  include(e, push(s,e1)) is e = e1 \/ include(e,s),
  include(e, pop(s)) is
     if e = top(s) then count(e,s) > 1
             else count(e,s) > 0 end  pre s ~= empty
value count: E >< S -> Nat,
      top: S -~-> E
axiom forall e,e1:E, s:S :-
   count(e, empty) is 0,
   count(e, push(s,e1)) is count(e,s) +
      if e = e1 then 1 else 0 end,
   count(e, pop(s)) is count(e,s) -
      if e = top(s) then 1 else 0 end pre s ~= empty,

   top(push(s,e1)) is e1,
   top(pop(s)) is last(s,2) pre size(s) >= 2

value last: S >< Nat -~-> E,
      size: S -> Nat
axiom forall s:S, e:E, n:Nat :-
   size(empty) is 0,
   size(push(s,e)) is size(s) + 1,
   size(pop(s)) is size(s) - 1 pre s ~= empty,

   last(push(s,e), n) is
      if n = 1 then e else last(s, n-1) end
        pre n > 0 /\ n <= size(s),
   last(pop(s), n) is last(s, n+1)
        pre s ~= empty /\ n > 0 /\ n < size(s)
\end{lstlisting}

Обратите внимание, что
\begin{enumerate}
\item при помощи выбранных для описания стека генераторов значение стека будет иметь вид, например, push(push(pop(push(empty,1)),10),1); такое выражения значения в типе стек может казаться наиболее адекватным, ведь добавление и удаление элементов --- именно те операции, при помощи которых можно изменить значение (<<состояние>>) стека; однако посмотрите, насколько увеличивается спецификация и теряется ее наглядность, если в число генераторов включена лишняя функция (pop); сравните:
\begin{lstlisting}
type E, S == empty | push(S, E)
value include: S >< E -> Bool
axiom forall e,e1:E, s:S :-
  ~ include(e, empty),
  include(e, push(s,e1)) is e = e1 \/ include(e,s),
\end{lstlisting}

\item в этом примере синтаксически разные термы из генераторов могут означать одинаковые значения типа, например, push(pop(push(empty,1)),2) и push(pop(push(empty,3)),2); в таких случаях надо быть внимательными при выписывании аксиом: помнить и понимать, сколько разных возможностей есть для рекурсивной части цепочки (речь идет о переменной <<s>> в примерах) --- например, в аксиоме с генератором, удаляющим элемент, надо в том числе предполагать, что этот элемент может появиться много раз до этого, добавляться и удаляться.
\end{enumerate}

%% минимальный набор генераторов уменьшает спецификацию (оценить количество аксиом?)

%% при нескольких генераторах надо быть аккуратными

\z Для множества, определенного таким образом:
\begin{lstlisting}
type E, S == empty | add(S, E) | delete(S, E)
\end{lstlisting}
дать алгебраическую спецификацию операции проверки наличия заданного элемента
\begin{lstlisting}
value include: S >< E -> Bool
\end{lstlisting}


\zhead{Описание эффекта  на основе структуры терма}

Вы уже заметили, что основной принцип написания аксиомы --- понять, как вычисляется обсервер после последнего сработавшего генератора. При этом для выражения этой аксиомы используются аргументы, с которыми вызван обсервер и последний генератор. Однако не всегда просто описать эффект работы генератора на основе лишь аргументов последнего из них.

\z Дать алгебраическую спецификацию типа <<Ограниченная очередь>>. В эту очередь можно добавлять и удалять элементы, но только если количество хранящихся элементов не превышает заданную величину.

\textbf{Решение:}
\begin{lstlisting}
value capacity : Nat
type E, S == empty | add(E,S)
value delete: S -~-> E
axiom forall e:E, s:S :-
   delete(add(e,s)) is s
     pre size(s) < capacity

value size: S -> Nat
axiom forall e:E, s:S :-
   size(empty) is 0,
   size(add(e,s)) is size(s) + 1
       pre size(s) < capacity
\end{lstlisting}

Обратите внимание, что
\begin{enumerate}
\item пришлось добавить и описать дополнительный обсервер size;
\item существует множество способов реализации ограниченной очереди при помощи имеющихся в языках программирования средств (например, при помощи <<кольцевой очереди>>, реализованной при помощи массива и двух указателей), однако здесь предъявляется именно формализация функциональности операций, которая остается справедливой и неизменной для любой реализации <<ограниченной очереди>>;
\item данное описание не дает определения того, в каких случаях определена каждая операция в отдельности;
\item при написании аксиомы для рекурсивной части терма достаточно представлять только \emph{правильно построенные термы} --- такие термы, в которых все функции вызваны с аргументами правильных типов и в каждой функции выполнено ее предусловие; например, для аксиомы size(add(e,s)) не надо представлять, что она должна описывать и такой терм: size(add(e,add(e1,add(e2,empty)))) при capacity = 2.
\end{enumerate}

\z Дать алгебраическую спецификацию типа <<Исключающая очередь>>. В эту очередь элемент добавляется в том случае, если его не было, а если он там был, то он удаляется из очереди. Опишите операцию проверки наличия заданного элемента в такой очереди.


\zhead{<<Эффект>> операций}
%% приходится добавлять обсерверы, чтобы полностью описать эффект функции

\z В~\cite{tanenbaum_os} описаны операции с файлами, среди них описана операция Create следующим образом: <<\textsf{Create} (создание). Файл создается без данных. Этот системный вызов объявляет о появлении нового файла и позволяет установить некоторые его атрибуты.>> Напишите алгебраическую спецификацию файловой подсистемы с этой операцией. Естественно, вам понадобится сигнатура этой операции. Вот она:  \texttt{int creat(char *path, int mode)}, параметр \texttt{path} содержит полное или относительное имя файла, параметр \texttt{mode} устанавливает атрибуты прав доступа различных категорий пользователей к новому файлу при его создании (если файл уже существовал, то новый не создается), операция возвращает значение файлового дескриптора для открытого файла при нормальном завершении и значение -1 при возникновении ошибки.

Решение:
\begin{lstlisting}
scheme FS = class
  type Path, Mode, FID, FS
  value
        creat : Path >< Mode >< FS -~-> FS >< FID,
        size: FS >< FID -~-> Nat,
        known: FS >< Path -> Bool,
        access: FS >< FID -~-> Mode,
        first: FS >< FID -> FS
        first(a,b) is a
  axiom
    forall path: Path, mode: Mode, fs: FS :-
        size(creat( path, mode, fs )) is 0,
        known(first(creat( path, mode, fs )), path),
        access(creat( path, mode, fs)) is mode
end
\end{lstlisting}
Обратите внимание, что
\begin{enumerate}
  \item для описания эффекта функции \texttt{creat} были введены дополнительные операции-обсерверы;
  \item аксиомы напрямую выражают текст, описывающий операцию \texttt{creat} --- аксиомы формализуют \emph{требования} на эту операцию.
\end{enumerate}
Ответьте на следующие вопросы:
\begin{enumerate}
  \item эта спецификация неполная, почему? является ли она противоречивой? как, добавив 1 аксиому, полностью описать операцию known?
  \item имеют ли смысл сами по себе введенные дополнительные операции или они выполняют лишь вспомогательную для описания \texttt{creat} функцию?
  \item допустим, мы догадываемся, что Path = \textbf{Text}, а Mode = \textbf{Nat}; дополненная этим знанием спецификация, останется ли алгебраической ? станет ли полной ? останется ли непротиворечивой ? будет ли она соответствовать исходной постановке задачи ? не станет ли она допускать того, что не должно бы по условию ?
\end{enumerate}

\z В~\cite{tanenbaum_os} описаны операции с файлами, среди них описана операция Delete следующим образом: <<\textsf{Delete} (удаление). Когда файл уже более не нужен, его удаляют, чтобы освободить пространство на диске. Этот системный вызов присутствует в каждой операционной системе.>> Напишите алгебраическую спецификацию файловой подсистемы с этой операцией. Естественно, вам понадобится сигнатура этой операции. Вот она:  \texttt{void delete(int fid)}, параметр \texttt{fid} содержит значение файлового дескриптора.

\z В~\cite{tanenbaum_os} описаны операции с файлами, среди них описана операция Open следующим образом: <<\textsf{Open} (открытие). Прежде чем использовать файл, процесс должен его открыть. Системный вызов open позволяет системе прочитать в оперативную память атрибуты файла и список дисковых адресов для быстрого доступа к содержимому файла при последующих вызовах.>> Напишите алгебраическую спецификацию файловой подсистемы с этой операцией. Естественно, вам понадобится сигнатура этой операции. Вот она:  \texttt{void delete(int fid)}, параметр \texttt{fid} содержит значение файлового дескриптора.

\z В~\cite{tanenbaum_os} упомянут системный вызов mmap: <<Системный вызов mmap принимает на входе два параметра: имя фала и виртуальный адрес паямти, по которому операционная система отображает указанный файл. Для реализацияи отображения файлов на память изменяются системные внутренние таблицы.При обращении к памяти по адресу от 512 до 576К происходит прерывание из-за отсутствия страницы, обработчик которого предоставляет считанную в память страницу 0 файла.Если потом эта страница удаляется из памяти алгоритмом замены страниц, она записывается в соответствующее место файла.>>

%% не всегда просто понять, полна ли спецификация
\zhead{Противоречивость алгебраических спецификаций}

Если спецификация допускает подстановку и <<вычисление>> термов увеличивающейся длины, пытаться <<вычислять>> эти термы разными способами, которые допускают аксиомы, и проверять, получаются ли одинаковые результаты в разных способах. Если получились разные, значит, найдено противоречие.

\z Противоречива ли следующая спецификация?
\begin{lstlisting}
type T
value empty : T,
    put: T >< Nat -> T,
    get: T >< Nat -> Bool
axiom forall t: T, x, y, z: Nat :-
  put( put(t, x), x ) is put(t, x),
  ~get( empty, x ),
  get( put(empty, x), y ) is (x = y /\ x\2 = 0),
  get(put(put(empty,x),y),z) is (z=x /\ y\2=0) pre x\2=0,
  get( put(t, y), x ) is get(t, x) pre y ~= x,
  get( put( put(t, x), y), x ) is get( put(t,x), x ),
  ~get(put(put(t,x),x+1),x+1) pre ~get(t,x+1) /\ get(t,x)
\end{lstlisting}

\z Противоречива ли следующая спецификация?
\begin{lstlisting}
type T
value empty: T,
    put: T >< Nat -> T,
    get: T >< Nat -> Bool
axiom forall t:T, x, y, z: Nat :-
  put(put(t, x), x) is put(t, x),
  get( put (put(t, 0), x ), x) is (x > 0),
  get( put (put(t, 2*x), x ), x) is (x > 0),
  ~get( empty, x ),
  ~get( put(empty,x), y ),
  get(put(put(empty,x),y),z) is (z>0 /\ z=abs(x-y) )
\end{lstlisting}

\z Противоречива ли следующая спецификация?
\begin{lstlisting}
type T = Nat
value empty: T,
    put: T >< Nat -> T,
    get: T >< Nat -> Bool
axiom forall t: T, x, y, z: Nat :-
  put( put(t, x), y ) is put(t,x) pre y <= x,
  ~get( empty, x ),
  get( put(empty, x), y ) is (x = y),
  get(put(put(empty,x),y),z) is (z = x \/ z = y /\ y > x),
  get( put(t,x), y ) is get(t,y) pre y ~= x,
  ~get(put(t,2*y),2*y) pre get(t,y) /\
       get(t,y+1) /\~get(t,2*y),
  get( put(put(t, 0), x), x ) is get( put(t,x), x )
\end{lstlisting}

\z Противоречива ли следующая спецификация?
\begin{lstlisting}
type T
value empty: T,
    put: T >< Nat -> T,
    get: T >< Nat -> Bool
axiom forall t: T, x, y, z: Nat :-
  put( put(t, x), y ) is put(t,x) pre y <= x,
  ~get( empty, x ),
  get( put(empty, x), y ) is (x = y),
  get(put(put(empty,x),y),z) is (z = x \/ z = y /\ y > x),
  get( put(t,x), y ) is get(t,y) pre y ~= x,
  ~get( put(t, 2*y), 2*y ) pre get(t,y) /\ get(t, y+1),
  get( put(put(t, 0), x), x ) is get( put(t,x), x )
\end{lstlisting}

\z Противоречива ли следующая спецификация?
\begin{lstlisting}
type T
value empty: T,
    put: T >< Nat -> T,
    get: T >< Nat -> Bool
axiom forall t: T, x, y, z: Nat :-
  put( put(t,x), x ) is put(t,x),
  ~get( empty, x ),
  get( put(empty, x), y ) is (x = y),
  get(put(put(empty,x),y),z) is (z=x \/ z=y /\ y>2),
  get( put(t,x), y ) is get(t, y) pre y ~= x,
  get( put(t,y), y ) pre get( put(t,x), x ) /\ x <= y,
  ~get(t,x) pre get( put(t, 0), 0),
  ~get(t, x) /\ ~get(t,y) pre x ~= y /\ get(put(t,1),1)
\end{lstlisting}

\z Противоречива ли следующая спецификация?
\begin{lstlisting}
type T
value empty: T,
    put: T >< Nat -> T,
    get: T >< Nat -> Bool
axiom forall t: T, x, y, z: Nat :-
  put( put(t,x), x ) is put(t,x),
  ~get( empty, x ),
  get( put(empty, x), y ) is (x = y),
  get( put( put(empty,x), y ), z ) is (z = x \/ z = y /\ y > 0),
  get( put(t,x), y ) is get(t, y) pre y ~= x,
  get( put(t,y), y ) pre get( put(t,x), x ) /\ x <= y,
  ~get(t,x) pre get( put(t, 0), 0) /\ ~get(t,0),
  ~get(t,x) \/ ~get(t,y) pre x~=y /\ get(put(t,1),1) /\ ~get(t,1)
\end{lstlisting}

\zhead{Полнота алгебраических спецификаций}

Не забывайте, что у нас есть только система аксиом и логика, т.е. правила получения новых выражений из имеющихся. За символами имен операций не стоит никакой семантики, даже если она <<предполагалась>> автором. Наоборот, эта система аксиом должна \emph{дать} нам семантику символов, т.е. дать нам возможность сделать с этими символами некие осмысленные действия.

\z Полно ли описывает следующая спецификация тип <<Множество>> ?
\begin{lstlisting}
type E, S
value empty: S,
      add: E >< S -> S,
axiom forall e1, e2: E, s : S :-
   add(e1, add(e1, s)) is add(e1, s),
   add(e1, add(e2, s)) is add(e2, add(e1, s))
\end{lstlisting}

\textbf{Решение:}
В этой спецификации нет ни одного обсервера, поэтому вопрос о полноте для нее некорректен.

\z Полна ли следующая спецификация типа <<Очередь>> ?
\begin{lstlisting}
type E, Q
value empty: Q,
      add: E >< Q -> Q,
      size: Q -> Nat
axiom forall e: E, q : Q :-
    size(empty) is 0,
    size(add(q, e)) is size(q) + 1
\end{lstlisting}

\textbf{Решение:}
Обсервер --- size. Он определен для empty и определен для генератора add. Значит, он определен для любого терма, дающего тип Q. Значит, функция size описана полно.

\z Полна ли следующая спецификация типа <<Очередь>> ?
\begin{lstlisting}
type E, Q
value empty: Q,
      add: Q >< E -> Q,
      size: Q -> Nat
axiom forall e: E, q : Q :-
    size(empty) is 0,
    size(add(q, e)) is size(q) + 1,
    add(add(q, e), e) is add(q, e)
\end{lstlisting}

\textbf{Решение:}
Без учета последней аксиомы size(add(add(q,e),e)) is size(add(q,e)) + 1 is size(q) + 2. А теперь с последней аксиомой: size(add(add(q,e),e)) is size(add(q,e)) is size(q) + 1. Получается, что size(q) + 1 = size(q) + 2. Иными словами, из аксиом следует ложь, система аксиом противоречива. А раз так, вопрос о полноте некорректен.


\z Полна ли следующая спецификация типа <<Очередь>> ?
\begin{lstlisting}
type E, Q
value empty: Q,
      add: Q >< E -> Q,
      first: Q -~-> E,
      size: Q -> Nat
axiom forall e: E, q : Q :-
    first(add(empty, e)) is e,
    first(add(q, e)) is first(q) pre q ~= empty,
    size(empty) is 0,
    size(add(q,e)) is size(q) + 1
\end{lstlisting}

\textbf{Решение:}
Любое значение в целевом типе Q имеет один из двух видов (просто напросто, нет других функций, возвращающих Q):
\begin{itemize}
  \item empty
  \item add(add(add(...add(add(empty, e1), e2)...)
\end{itemize}

Посмотрим на first с точки зрения определения достаточной полноты. Обе аксиомы для простоты можно объединить в одну: first(add(q,e)) is if q = empty then e else first(q) end. Тогда такое выражение <<вычислить>> можно: first(add(empty,e1)) is e1 (по первой аксиоме). Посмотрим такое выражение: first(add(add(empty,e1),e2)) is if add(empty,e1) = empty then e2 else e1 end. Чтобы закончить <<вычисление>>, надо понять истинность выражения add(empty,e1) = empty. В общем случае дать ответ на этот вопрос нельзя (\emph{проблема равенства термов алгоритмически неразрешима}). Однако в данном случае ответить на этот вопрос можно.

Для этого сделаем такой хитрый ход -- \emph{<<навесим>> на эти два выражения сверху другой обсервер}: size(empty) is 0, size(add(empty,e1)) is size(empty) + 1 is 0 + 1 is 1, т.е. size(empty) ~= size(add(empty,e1)). Поскольку size --- тотальная функция, то она детерминированная, т.е. all x, y : Q :- x = y => size(x) = size(y), что то же самое, что size(x) ~= size(y) => x ~= y. Теперь в качестве x возьмем empty, а в качестве y возьмем add(empty, e1). Получим, что add(empty, e1) ~= empty. Ура, желаемое доказано! Тем самым, можно и вычислить второй терм, он равен e1. Аналогично, можно вычислить и все остальные термы.

Осталось рассмотреть единственный терм: first(empty). Если бы существовали такие q' и e', что add(q',e') равнялось empty, то было бы возможно применение аксиом. Но, как следует из первой части, такие q' и e' не существуют, значит, <<вычислить>> first(empty) на основе данных аксиом нельзя. Ответ: неполна.

\z Полна ли следующая спецификация типа <<Очередь>> ?
\begin{lstlisting}
type E, Q
value empty: Q,
      add: Q >< E -> Q,
      first: Q -~-> E,
axiom forall e: E, q : Q :-
    first(add(empty, e)) is e,
    first(add(q, e)) is first(q) pre q ~= empty,
\end{lstlisting}

\textbf{Решение:}
Рассуждая аналогично предыдущей задаче, приходим к вопросу об истинности add(empty,e1) = empty и в данной системе аксиом дать ответ на этот вопрос нельзя. Ответ: неполна.

%% вставить примеры противоречивых систем аксиом из заданий экзамена прошлого года

%% понять, как описывать недопустимое поведение

%\zhead{Спецификация отношений <<многие-ко-многим>>}
%
%Алгебраические спецификации позволяют описать многие компоненты, осуществляющие отношение <<многие-ко-многим>>, совершенно не задумываясь о том, каким образом это отношение выразить чем-нибудь более известным (например, вспомните, сколько есть различных способов представления этого отношения для реляционной модели данных!)
%
%\z Специфицируйте компонент, отвечающий за хранение и модификацию данных о студентах Университета и спецкурсах. А именно, есть студенты, они добавляются в базу внутри этого компонента. Есть спецкурсы, которые также добавляются. Как-то студенты записываются на спецкурсы. Компонент позволяет

\zhead{Рекурсивные типы}

\z Специфицируйте операцию проверки вхождения элемента в бинарное дерево.

\textbf{Решение:}
\begin{lstlisting}
type Node, Tree == empty | add(Node, Tree, Tree)
value check: Node >< Tree -> Bool
axiom forall n, n1:Node, left, right: Tree :-
   ~check( empty ),
   check(n, add(n1,left,right) ) is (n = n1) \/
             check(n, left) \/ check(n, right)
\end{lstlisting}

\z Специфицируйте операцию вычисления высоты бинарного дерева.

\z Специфицируйте операцию проверки бинарного дерева на сбалансированность.

\z Специфицируйте операцию получения предка элемента бинарного дерева.

\z Специфицируйте добавление элемента в двоичное дерево поиска.

\z Специфицируйте удаление элемента из двоичного дерева поиска.

\z Специфицируйте добавление элемента в АВЛ-дерево. Определение операции предполагается найти самостоятельно.

\z Специфицируйте удаление элемента из АВЛ-дерева. Определение операции предполагается найти самостоятельно.

\z Специфицируйте добавление элемента в 2-3-дерево. Определение операции предполагается найти самостоятельно.

\z Специфицируйте удаление элемента из 2-3-дерева. Определение операции предполагается найти самостоятельно.

\z Специфицируйте добавление элемента в декартово дерево. Определение операции предполагается найти самостоятельно.

\z Специфицируйте удаление элемента из декартова дерева. Определение операции предполагается найти самостоятельно.

\z Специфицируйте добавление элемента в красно-чёрное дерево.  Определение операции предполагается найти самостоятельно.

\z Специфицируйте удаление элемента из красно-чёрного дерева.  Определение операции предполагается найти самостоятельно.




\chapter{Логико-алгебраические спецификации}

\section{Логические модели}

Логико-алгебраические модели --- набор свойств, утверждений, аксиом~\cite{kuliamin}. Из этих аксиом путем логического вывода получаются другие свойства.

Если количество состояний системы невелико, ее удобно моделировать в виде конечного автомата. При возрастании количества состояний конечные автоматы становятся неудобными как для моделирования, так и для работы с такими моделями. До некоторого момента и для некоторых задач помогают расширенные конечные автоматы (EFSM), модели Крипке, но для систем более реальных размеров и они могут стать неудобными. Тогда вместо такого (явного) указания состояний и переходов между ними предлагается задавать \emph{правила}: о структуре состояния, о свойствах состояния, о свойствах переходов\footnote{Аналогично совершается переход от задания формальных языков перечислением его слов к формальным грамматикам, т.е. правилам конструирования слов}. Одним из основных механизмов теперь будет логический вывод новых свойств и правил из заданных в спецификации.

\head{Логическая модель на RSL}
Это набор аксиом. Аксиомы помещаются в секцию \textbf{axiom}. Пример:
\begin{lstlisting}
scheme Addition = class
        value add: Int >< Int -> Int
        axiom forall x, y: Int :-
            add(x,y) >= x,
            add(x,y) >= y,
            add(x,0) is x,
            add(10, 45) is 55,
            exists! z : Int :- add(0, z) is x
end
\end{lstlisting}

\section*{Задачи}

% !Mode:: "TeX:UTF-8"

\refstepcounter{problem_type}

Первые задачи несложные, в них надо формализовать указанные утверждения. Большинство задач взяты из~\cite{nepeivoda}.


\z Ни одному лысому не нужна расческа

Решение:
\begin{lstlisting}
scheme Exmp = class
  type Lys, R4
  value need: Lys >< R4 -> Bool
  axiom all l:Lys, r:R4 :- ~need(l,r)
end
\end{lstlisting}


\z Все мои тетки не справедливы.
\z Ни один кошмарный сон не приятен.
\z  Все битвы сопровождаются страшным шумом.
\z Не все двоечники ленивы.
\z  Каждый, кто упорно работает, добивается успеха.
\z  Ни один бездельник не станет знаменитостью.
\z  Некоторые художники не бездельники.
\z  Некоторые бездельники не художники.
\z  Некоторые подушки мягкие.
\z  Тот, кто может укрощать крокодилов, заслуживает уважения.
\z  Ни одна лягушка не имеет поэтической внешности.
\z  Ни одна тачка не комфортабельна.
\z  Всякий орел умеет летать.
\z  Некоторые свиньи не умеют летать.
\z  Некоторые свиньи --- не орлы.
\z  Ни один судья не справедлив.
\z  Ни один ребенок не любит прилежно заниматься.
\z  Все шутки для того и предназначены, чтобы смешить людей.
\z  Ни один парламентский акт не шутка.
\z  Пауки ткут паутину.
\z  Все лекарства имеют отвратительный вкус.
\z  Ни у одной ящерицы нет волос.
\z  Все свиньи прожорливы.
\z  Все, что сделано из золота, драгоценно.
\z  Некоторые секретари --- птицы.
\z  Все секретари заняты полезным делом.
\z  Ничто разумное не ставит меня в тупик.
\z  Логика часто ставит меня в тупик.
\z  Некоторые цыплята --- не кошки.
\z Точки A, B, C являются вершинами равнобедренного треугольника.
\z Иванов, Петров, Васильев и Сидоров могут вытащить эту машину из ямы, если они трезвы и видят бутылку.
\z Иванов, Петров, Васильев и Сидоров не могут решать квадратные уравнения, даже если они трезвы, но видят бутылку.
\z Не все углы, синус которых больше 1/2, больше $\pi$/6.
\z Квадратные корни из некоторых рациональных чисел иррациональны.
\z Синус и косинус равны друг другу тогда и только тогда, когда равны тангенс и котангенс.
\z  Когда кто-то поет больше часа, он надоедает.
\z Все девочки боятся лягушек и мышей.
\z Кошки бывают только белые и серые.
\z  Все ораторы либо честолюбивы, либо скучны.
\z Число делится на 25 в том и только том случае, когда оно делится на 50 либо дает при делении на 50 остаток 25.
\z Все комплексные числа действительны или становятся действительными после умножения на i.
\z Не все студенты отличники или спортсмены.
\z Для того чтобы выполнялось равенство $x = \sqrt{x^2}$, необходимо и достаточно, чтобы $x$ было положительным действительным числом.
\z Не все, что рассказывал барон К. Ф. И. фон Мюнхгаузен, ложь.
\z Некоторые людоеды — плохие люди.
\z Некоторые финансисты — мошенники, но не все.
\z Прапорщики любят порядок, и не только они.
\z Милиционеры замешаны в преступлениях, но не все.
\z Некоторые замки не отпираются, но запираются.
\z Если будешь хорошо учиться, поступишь в вуз, а иначе
провалишься.
\z Ничего не вижу, ничего не слышу, ничего не знаю.
\z Молодо—зелено.
\z Взялся за гуж --- не говори, что не дюж.
\z Чтобы не быть собакой, достаточно быть кошкой.
\z Чтобы не быть человеком, необходимо быть свиньей.
\z Некоторые кошки поют по ночам.
\z Все компакты совершенно нормальны.
\z Некоторые мюмзики не куздры.
\z Три точки A, B, C лежат на одной окружности.
\z Три точки A, B, C не лежат на одной прямой.
\z Числа a и b имеют одинаковый знак.
\z Одно из чисел a, b равно 0.
\z Числа a и b имеют разные знаки.
\z Ромео и Джульетта любят друг друга.
\z Гамлет и Клавдий ненавидят друг друга.
\z Мери любила Печорина, но не взаимно.
\z Чтобы прийти на свадьбу, необходимо приглашение жениха или невесты.
\z Некоторые лентяи не оптимисты, но жизнелюбы.
\z Все замки отпираются и запираются.
\z Никто из нашего класса не поехал в Москву и Париж.
\z Все мои одноклассники поехали в Москву и Париж.
\z Некоторые числа четные.
\z Некоторые лекции невозможно понять.
\z Всякому в Москве не перекланяешься.

\zhead{Более сложные задачи:}

\z Весна, и вы говорите: «Все парни и девушки сейчас влюблены друг в друга». Что это означает?
\z Когда Ясон кинул в середину войска, выросшего из зубов дракона, камень, все воины передрались и перебили друг друга.
Переведите это утверждение на формальный язык.
\z Ограниченная сверху и снизу на [a, b] функция непрерывна на нем.
\z Функция $f$ имеет разрыв 2-го рода в точке 0.
\z Функция, имеющая точку разрыва, не может быть непрерывной.
\z Монотонная на [a, b] функция ограничена снизу на [a, b].
\z Если последовательность ограничена, то она имеет предельную точку.
\z Чтобы установить рекорд, необходимо иметь способности, и прилежно тренироваться, и найти хорошего тренера.
\z Чтобы победить, нам необходимы и хорошие нападающие, и хорошие защитники, и хорошие вратари.
\z Если вчера Петров прогулял два занятия, то сегодня только одно.
\z Все члены Политбюро, избранного на XIV съезде ВКП(б), ненавидели друг друга.
\z Все философы критиковали друг друга.
\z Все рыцари сражались друг с другом на поединках.
\z Все начальники подсиживают друг друга.
\z Все мужчины --- подонки, а мой муж --- хороший человек.
\z Сдай экзамен на <<отлично>>, и поступишь в аспирантуру.


\section{Алгебраические спецификации}

\documentclass{article}
\usepackage[cp1251]{inputenc}
\usepackage[russian]{babel}
\usepackage{pscyr}
\usepackage{indentfirst}
\usepackage{graphicx}
\usepackage{amsthm}
\usepackage{amssymb}

\title{�������� ������������� ����� �����������}
\author{}
\date{}

\textwidth=16cm \oddsidemargin=0cm

\newtheorem{theorem}{�������}
\newtheorem{lemma}{�����}

\begin{document}
\maketitle

��������: �������� ���������� ������ -- ������������������ ��������
�������� + ��� ���� ���������. ���� ���-������: ���������
�������������, ������� ����������� � �������-�������������.

\section{������������� ��������� �� ��������� ��� �������� ��������
�������� � ���-������}\label{cache_sets}

\begin{abstract}
� ���� ������� ���� ���� � ���, ��� ������ ����� ��������� �������� ������
��� ������������������ �������� �������� � ���-������. ������ ��������
����� ���������� �� ����������� ������������� �������� �������� �
���-������, ����� �������� ��������� ��������� ��� �� �������� �
����� ������������ ���� ���������. 
\end{abstract}

����� ���� ������� ������������ ������ -- �� ��������� �������
��������� ������� ��������� � ������ ��. ��������� �������� ������
�������� ��������� ��������� ��������������� (�������� ���������
����� ������� ���������� ��������� �������, �������� �����
���-������ � ������ ��������� ��������������� ����� �������
���������� ��������� �������), �� � ������� � �������� ���������
���������� ����� ������������ �������� ���������� ���������
���������������.

���������� ����, ������ �� ������� ������������ ������� �������
���������. ������ ���������� ����� �������� �������� ��������� ���
���������� ���-������ � ������ ���������. �������� ��������� �����
�������������� <<����������>> �����������. ���������� ���-������
����� �������������� \emph{���������� �����} ���-������. ����������
���-������ ����� ��������� �� ��� ��������� -- ��������� ���
�������� ������������ ������ � ��������� ��� �������� ����� �������
������������ ������. ��� ������������� �������� �������� �
���-������ ����� �������������� ������ ��������� ��� �������� �����.

\begin{figure}[h] \center
  \includegraphics[width=0.5\textwidth]{mpset}\\
  \caption{������������� ��������� ���������������}\label{mpset}
\end{figure}

������ ����� ��������� ����� � ����, ��� ��� ������� ����������. ��
����� ���� ������� �������� �������� ����������, ���������� �
���-������� (��� �������� ����� �������������� ��������� ����������:
$L$ -- ������� ��������� ���-������ (��������� �����), $x$ --
(����������) ����� ������ � ����������):
\begin{itemize}
\item \emph{���-���������} ���������� � ��� ������, ����� ������ ��
������� ������ ������������ � ���-������; �� ����� �������� ��������
������������ ������� ��������� $x \in L$;
\item \emph{���-������} ���������� � ������, ����� ������ �� �������
������ �� ������������ � ���-������; ��� ���� �������� ��������
������������ ��������� $x \notin L$.
\end{itemize}

���������� ����� ������ � ���������� ($x$) ����� ���� ��������� ��
���������� ���������� (������ ���������). ��� ���� �������
������������ �������� ��������� � ������ ������ ����������
(��������� ��� ��������� Static single assignment form,
SSA).

����������, ��� ��������� ��������� ��� $L$ � ������ ����������
������� ��������. $L$ ��� ������ ���������� ���� ���������
���������� ���-������, ��� ���������� �������� � ������� ���������.
����� ��������� ��� ��������� ���������� $L$, � ��� ��������� --
$L'$. ����� ���� ��������� ���������� -- ���-���������, �� $L'
\equiv L$ (��� ��� ���������� �� ��������), � ���� ���������
���������� -- ���-������ � ������� $x$, �� $L' \equiv (L \setminus
\{x'\} \cup \{x\})$ (��� ��� � ���-������ ��� ������� �����������
������ �� ������� ������, � ��������� ������ �����������, $x'$ ����
����� ����������� ������). ��� ����� ���������� $x'$ ������� �
������� ����� ���������: $x' \in L \wedge displaced(x') \wedge R(x)
= R(x')$, ����� �������� $displaced$ ��������� \emph{���������
����������}, �.�. �������, �� �������� � ���-������ ����������
������, ������� ������� �������, � �� �� ����� ��������� ������,
��������� ������. ��� ���-������ ������� ����������� �����������
����������� $(R(x) = R(x')) \rightarrow displaced(x')$, ������� ���
������ ���� ���-������ ��������� $displaced(x')$ ��������� ��
�������. �������������� ������ $R$ ������������ ��� ������� ������,
�������� ��������� �����, � ���-������ ������� ����������� �
�������-������������� ���-������. �������� ����� ��������� �����
������� -- $R(x)$ ��� ��������� �������, ������� ������������ �����
���������� � ��� �� ������, ��� � ����� ������ $x$ (�����
�����������, ��� ����� �� ����� ��������������� ����� ��� ������
������ � �� ��������������� �������� ������ ������, ������ ������
����� ��������������� ������ ������). ��� ����� ��������� -- $R(x)$
��� ����� ������ ������ $x$. ��� ����������� ��������� ����� ����
������� ����� ���������. ��� ���������-������������� ���-������
��������� $R(x) = R(x')$ �������� ������������� �������, ��������� �
��� ��� ������ ������������� ������ ������.

��������� ������� ��������� ��������� ��� $L$ ��� �������������
�������� � ����������� ����������� ��� �������� �������� �
���-������:
\begin{lemma}
����� $L$ -- ��������� ��� �������� ��������� ���-������, $L_0$ --
��������� ������� ������, ������������� � ���-������ �����
����������� ���������� ��������� �������, $\{x_i\}$ -- ���������
������� ������ � ����������� � ���-���������, �������������� ��
������� ���������� � ��� �� �������, ��� � � �������� �������,
$\{x'_i\}$ -- ��������� ������� ����������� ������ � ����������� �
���-���������, �������������� �� ������� ���������� � ��� ��
�������, ��� � � �������� �������. �����
$$L = L_0 \setminus \bigcup_{i} \{x'_i\} \cup \bigcup_{i} ( \{x_i\} \setminus \cup_{j > i} \{x'_j\}).$$
\end{lemma}
\begin{proof}
//TODO

��������, ���� ����� ������ ����������� ������������� 3 ���������� �
���-��������, �� $L = L_0 \setminus \{x'_1, x'_2, x'_3\} \cup
(\{x_1\} \setminus \{x'_2, x'_3\}) \cup (\{x_2\} \setminus \{x'_3\})
\cup \{x_3\}$.
\end{proof}

\begin{theorem}
����� $L_0$ -- ��������� ������� ������, ������������� � ���-������
����� ����������� ���������� ��������� �������, $\{x_i\}$ --
��������� ������� ������ � ����������� � ���-���������,
�������������� �� ������� ���������� � ��� �� �������, ��� � �
�������� �������, $\{x'_i\}$ -- ��������� ������� ����������� ������
� ����������� � ���-���������, �������������� �� ������� ����������
� ��� �� �������, ��� � � �������� �������. �����
\begin{itemize}
\item ��� ���������� � ���-���������� ������ $x$ ������� ��������
��������� ������������ ���������:
$$
\left[
   \begin{array}{l}
    x \in L_0 \wedge x \notin \{x'_1, x'_2, ..., x'_n\} \\
    x = x_1 \wedge x \notin \{x'_2, ..., x'_n\} \\
    x = x_2 \wedge x \notin \{x'_3, ..., x'_n\} \\
    ...\\
    x = x_{n-1} \wedge x \notin \{x'_n\} \\
    x = x_n \\
   \end{array}
  \right.
$$

\item ��� ���������� � ���-�������� ������ $x$ (� �������
����������� ������ $x'$) ������� �������� ��������� �������
���������:
$$
\left\{
   \begin{array}{l}

  \left[
   \begin{array}{l}
    x \notin L_0 \wedge x \notin \{x_1, x_2, ..., x_n\} \\
    x = x'_1 \wedge x \notin \{x_2, ..., x_n\} \\
    x = x'_2 \wedge x \notin \{x_3, ..., x_n\} \\
    ...\\
    x = x'_{n-1} \wedge x \notin \{x_n\} \\
    x = x'_n \\
   \end{array}
  \right. \\

  { }\\

  \left[
   \begin{array}{l}
    x' \in L_0 \wedge x \notin \{x'_1, x'_2, ..., x'_n\} \\
    x' = x_1 \wedge x \notin \{x'_2, ..., x'_n\} \\
    x' = x_2 \wedge x \notin \{x'_3, ..., x'_n\} \\
    ...\\
    x' = x_{n-1} \wedge x \notin \{x'_n\} \\
    x' = x_n \\
   \end{array}
  \right. \\

  { }\\

  displaced(x')\\

  { }\\

  R(x) = R(x')\\

  \end{array}
\right.
$$

\end{itemize}
\end{theorem}
\begin{proof}
//TODO
\end{proof}

��������, ��� ������������ ����������� ��� ���-��������� �
���-������� ���������� ����� ��������, ���� ���������� � ��� ����
��� ���������� ��������������� �������������. ������ ���� ��������
��������� ���������� � ���� ��������� �� ��������� �����.

\subsection{�������� ��������� ����������� ��� �������� ��������
�������� � ������� ��������� �� ���������}

����, ���� �������� ������, � �������� ������� ��� ���������
���������� ��������� ���������� ��� ������������ ����, � ��� ����
���������� ���� �������� �������, ��� ������� ��� ����� ��������� �
������ ����������. ��� ������� ������������ ���������� � ����
������� (��������, � ������������) ��������� �� ��������� �����,
������� ����� ��������, �������� ��������� �������:

\begin{enumerate}
\item ���������� �� �������� ������� ���������, ������ ����������
����;
\item ������� ��������� ������������ �� ������ �������
����������, ������ �������� ���������� (�.�. �������� �������, �
������� ���������� ���������) ����� ������������ ��� �����������
��������� ���������� ������������; ������������� ������������ ��
����������� ��������� ����������, � ������� ������� ������������
������ �����;
\item �������� ������� \emph{�������� ����������} -- ��� �����
��������� �������, ������� ��������������� ������ �� ����������; �
��������� ���� ������ �� ��������� ���������� ��� ��������
���������� ��������� ���-������; ����� ��������� ���������� ����
����������, ����������� ������ ���; ������� ����������� �����
��������� ���������� �������, ������ ������������� ����������
��������� ����������;
\item \emph{(��������� ����������� ����������)} ���� ��������
���������� �������� ������ ��������� ���������� �� ����� ����
�������, ������ ����� �� �� ���������� ����;
\item \emph{(��������� �������)} �������� �������� ���������
���������� �������� �� ����� �������� ������� ��������� ������
���������; ��� ������� ����� ���� ������� ��� ��������������
���������;
\item �������� ������������� �������������� ��������
(��������, ��� ��� ������� � $R$);
\item �������������� �� ���������� ������� ������� ��������� ��
���������.
\end{enumerate}

\subsection{��������� ��� ��������� ��������� ����������}

����� ����� ������������, ��� ����������� ���� �������� ����
��������� ��� ��������� ���������, ����������� ��������� ���������
����������, �������� ����� �������������� � ���������������� -- ���
LRU (Least Recently Used), FIFO (First-In First-Out) � Pseudo-LRU.
��������� ���������� Random �� ���������������, ����� ��
��������������������� � ������������������� ������� ���������.

\subsubsection{LRU -- Least Recently Used}

LRU (Least Recently Used) -- ��� ��������� ����������, ������������
����������� ������ ��� �������� ������������. ��� ���������� ���
����������, ���������� ��������� ����������� ������, �.�. ����
������������ �� ������, � ������� ������� ����������� ���������. ���
��������� ������������, ��������, � ���������������� ����������� MIPS~\cite{mips64_II}.

��������� ���������� LRU ������ ������������ � ��������������
��������� ���������. ����� ��������, ��� ������� �������� ���-������
�������� ������� ��������� � ����. ������ ��������� �����������
�������. ����������� ����� ������� � ����������� ���������. ������
�������� � ���� ��������� �� ��������� ��������� �������� ���������
����� ������� ����������.

������ ������ �������� LRU ������� �� �������� ������� �� ���������
������ (�.�. ����� �������������� ������� ���������). ����� ������
���������� �������� ������������������� �������� ��������� ��������
(��.���.~\ref{lru1}):
\begin{itemize}
\item ��� ���-��������� �������, ��������������� ������ ����������,
������������ � ������, ��������� �������� �� ������� �� �������
���������� �� ���� �������;
\item ��� ���-������� ����������� ��������� �������, � ������
����������� �������, ��������� ������.
\end{itemize}

\begin{figure}[h] \center
  \includegraphics[width=0.6\textwidth]{lru1}\\
  \caption{��������� ���������� LRU (w - ��������������� ���-������)}\label{lru1}
\end{figure}

������ � ��� �������� �� ��������, ������ ��� ��� �����������
���������� ������� ���������.

������ ������ �������� LRU ������� �� ������ ���������� ��������� �
������������ ��������: ����� �������� ��� ��������, ���������� �
����������, ����� ����� ��������� ���������� � ���� � �����������
���� ��������� �� ���� ��������� �������� ��������� ���-������,
����� ����. ���������� �� ���������� ��������� � ���������� ��������
�������� ���������� (��. ���.~\ref{lru-ranges}).

\begin{figure}[h] \center
  \includegraphics[width=0.4\textwidth]{lru}\\
  \caption{LRU � ��������� ����������}\label{lru-ranges}
\end{figure}

������� � ���� ��������� �� ��������� ��� ������. ��������
$displaced(x')$ ����� ����������� ����������� ��������� -- ������
������� ���������� ������������� ���������� ��������� ����������.
����� ��� ��������� ���������� � ����������, ������������ � ������
$y$ ���� ��������� ����� ������� ��������� ($x_1, x_2, ..., x_n$ --
��������� �������, � ������� ���������� ��������� ������ ���������
���������� (��� � ���-�����������, ��� � � ���-���������), $L$ --
��������� ��� ��������� ���-������ ��� ����������, �����������
$x'$):
$$
\left\{
   \begin{array}{l}
    x' = y \\
    \{x_1, x_2, ..., x_n\} \cap R(y) = (L \setminus \{y\}) \cap R(y)\\
   \end{array}
  \right.
$$

�������������� ������ $R$ ������������ � ������ ��������� �������
���� �� �������. � �������������� ��������� ����� �������� ���
�������:
\begin{lemma}\label{LRU_simplification}
��� ����� �������� �������� $X$, $Y$ � $Z$ �����, ��� $X \cap Y
\subseteq Z$, ���� ���������� $y$ �����, ��� $y \in (Y \cap
Z)\setminus X$, �� $X \cap Y = (Z \setminus \{y\}) \cap Y
\Leftrightarrow Y \cap ( Z \setminus X ) = \{ y \}$.
\end{lemma}
\begin{proof}
�������������. �� ����������� ��������� �������� � ���������������
�������� ����������� �������� $X \cap Y = (Z \setminus \{y\}) \cap Y
\Leftrightarrow X \cap Y = Z \cap Y \cap \overline{\{y\}}$.
��������� $A = Z \cap Y$, $B = X \cap Y$. �������������, $B = A
\setminus \{y\}$. �� ������� $y \notin B$ � $y \in A$. ������, $A =
B \sqcup \{y\}$. ������ $A \setminus B = \{y\}$. �������� ��������,
��� $A \setminus B = (Z \setminus X ) \cap Y$ : $A \setminus B = A
\cap \overline{B} = Z \cap Y \cap \overline{X \cap Y} = Z \cap Y
\cap (\overline{X} \cup \overline{Y}) = (Z \cap Y \cap \overline{X})
\cup (Z \cap Y \cap \overline{Y}) = Z \cap \overline{X} \cap Y = (Z
\setminus X ) \cap Y$.

�������������. ��������� $A = Z \cap Y$, $B = X \cap Y$. �
�������������� ����������� �������� ��� ����������� � �� �������
�������� $X \cap Y \subseteq Z \Leftrightarrow (X \cap Y) \setminus
Z = \varnothing \Leftrightarrow X \cap Y \cap \overline{Z} =
\varnothing \Leftrightarrow X \cap Y \cap (\overline{Z} \cup
\overline{Y}) = \varnothing \Leftrightarrow B \setminus A =
\varnothing$. ����� ����, �� ������� $A \setminus B = \{y\}$.
�������������, $A = (A \setminus B) \cup (A \cup B) = \{y\} \cup (B
\setminus (B \setminus A)) = \{y\} \cup (B \setminus \varnothing) =
\{y\} \cup B$. ����� �������, $A = B \cup \{y\}$. ����� ����, $y
\notin B$, ������, $A = B \sqcup \{y\}$, �������������, $B = A
\setminus \{y\}$. ���������� ����������� �������� $A$ � $B$,
��������: $X \cap Y = (Z \cap Y) \setminus \{y\} = Z \cap Y \cap
\overline{\{y\}} = (Z \setminus \{y\}) \cap Y$.
\end{proof}

\begin{lemma}[���������� ��������� ����������]\label{includedranges}
//TODO
\end{lemma}
\begin{proof}
//TODO
\end{proof}

\begin{lemma}[� ������������ �������
�����~\label{LRU_simplification} ��� ���������� ����������] $L
\supseteq \{x_1, x_2, ..., x_n\} \cap R(y)$
\end{lemma}
\begin{proof}[\proofname~(�� ����������)]
����� ����� $x_1, x_2, ..., x_n$ ���� $x_i$ �����, ��� $x_i \notin L
\wedge x_i \in R(y)$. ����� $L_{i+1}$ -- ��������� ���-������ �����
��������� � $x_i$. �����, ��� $x_i \in L_{i+1}$, �� $x_i \notin L$,
�������������, $x_i$ ��� �������� ����� $x_{i+1}$ � $x_n$. �����
�������, ����� $x_1, x_2, ..., x_n$ ���� �������, ��� ��������
���������� ������ � �������� ���������� $y$. �� ��������
�����~\ref{includedranges} ��� ����������. ������������.
\end{proof}

����� �������, ����� ��������� �����~\ref{LRU_simplification} ���
��������� ��������� ��� lru. � ���������� ��������, ��� ��� �������
��������� ���������� ����� �������� ��������� ������� ���������:
$$
\left\{
   \begin{array}{l}
    x' = y \\
    R(y) \cap (L \setminus \{x_1, x_2, ..., x_n\} ) = \{y\}\\
   \end{array}
  \right.
$$

\begin{theorem}
����������� LRU ����� ��������� ���������� � ����� ������
������������.
\end{theorem}
\begin{proof}
//TODO

��������, ��� �������� �����~\ref{includedranges} ����� ������������
$L$ ����� ������ ���������.
\end{proof}

\subsubsection{FIFO -- First-In First-Out}

FIFO (First-In First-Out) -- ��� ��������� ����������, ������������
����������� ������ �������� �������� ������� FIFO. ///��� ������������???

��������� FIFO ����� ���� ������� �� ������ ������� �� ���������
������ (�.�. ����� �������������� ������� ���������). ����� ������
���������� �������� ������������������� �������� ��������� ��������
(��.���.~\ref{fifo1}):
\begin{itemize}
\item ��� ���-��������� ������� ��������� �� ��������;
\item ��� ���-������� ����������� ��������� �������, � ������
����������� �������, ��������� ������.
\end{itemize}

\begin{figure}[h] \center
  \includegraphics[width=0.6\textwidth]{fifo1}\\
  \caption{��������� ���������� FIFO (w - ��������������� ���-������)}\label{fifo1}
\end{figure}

������� �� LRU ���� � ���, ��� ��� FIFO �� ���������� ������������
��������� ������ ��� ������������� ���-���������.

��� ��������� ����� ���� �������� ��������� � ��������������
��������� �� ���������. ��������� ��������� ���������� ��� FIFO ���
��������� ���������� �� �������� ������ � ���-������ �� ���
����������. ������ �������� �� ���� ��� ���������� � ���-�����������
(��� �� ������ ������� ���� � ����� ������ FIFO). ����� \emph{FIFO
����� ��������� � ��� ������, ����� � ��������� ����������� ���
������ ��������� ���-������ ����� ����������� ��� ������
������������ ������}.

������� � ���� ��������� �� ��������� ��� ������. ��������
$displaced(x')$ ����� ����������� ����������� ��������� -- ������
������� ���������� ������������� ���������� ��������� ����������.
����� ��� ��������� ���������� � ����������, ������������ � ������
$y$ ���� ��������� ����� ������� ��������� ($x_1, x_2, ..., x_n$ --
��������� �������, � ������� ���������� ��������� ������ ���������
���������� \textbf{� ���-���������}, $L$ -- ��������� ��� ���������
���-������ ��� ����������, ����������� $x'$):
$$
\left\{
   \begin{array}{l}
    x' = y \\
    \{x_1, x_2, ..., x_n\} \cap R(y) = (L \setminus \{y\}) \cap R(y)\\
   \end{array}
  \right.
$$

�������������� ������ $R$ ������������ � ������ ��������� �������
���� �� �������.

��� FIFO ����������� ��� ����� � ���������� ����������,
���������������� ��� LRU. � ���������, � �������������� �� �������
��������� ��� ��������� ���������� ����� ���� ���������� ���������
�������:
$$
\left\{
   \begin{array}{l}
    x' = y \\
    R(y) \cap (L \setminus \{x_1, x_2, ..., x_n\}) = \{y\}\\
   \end{array}
  \right.
$$

\begin{theorem}
����������� FIFO ����� ��������� ���������� � ����� ������
������������.
\end{theorem}
\begin{proof}
//TODO

��������, ��� �������� �����~\ref{includedranges} ����� ������������
$L$ ����� ������ ���������.
\end{proof}

\subsubsection{Pseudo-LRU}

//TODO

\pagebreak
\section{���������� ���������, ����������� �������� �������� �
���-������; ������� �������������}

\begin{abstract}
������ �������� ����� ����������� �������� ��� ������ �������� ������
��� �������� ����������������. ��������������� ��� ����� ������ ���������
�������� ������, ������������ ����������� ������ ���������. ������������, ���
����������� ������� ��������� �� ��������, � ��������� ��� ���-������ � TLB
����������� �������� ������������� �� �������. ������������� ������ ����
���������� � ����� ��������-solver'�� � �������� ������ ��� ��� ���� �
����������� �� �������� ������.
\end{abstract}

��������: ��� ����� TLB.

\subsection{����������� ���������� ���������� ��������� � ������
�� ����������� ����������������}

� ���������� ��������� � ������ � ����������� ���������������� �������������
�� ���� ����������. ���������� ���������� ��������� � ������ ����� ������� �� ��� ����� -- ���������� ����������� ������ � ���������� ��������� � �������
(��.���.~\ref{memoryAccess}).

\begin{figure}[h] \center
  \includegraphics[width=0.5\textwidth]{instr}\\
  \caption{������ ���������� ���������� ��������� � ������}\label{memoryAccess}
\end{figure}

���������� ����������� ������ �������� � ���� ������������ ������������ ������ ������, � �������� ���������� ��������� ��������. ����������� ����� ����������� �� ������ ���������� ���������. ����� ���������� ������������ ����������� ������ �� ������ ������������ ������ � �������������� TLB. �� ���� ����������� ����� ����������� �� ����� ����������� �������� � �������� ������ ��������, �����, ��������� TLB, ���������� ����� ������������ �� ���������������� ������ ����������� ����� � ���� �� �������� ������ ��������. TLB �������� ��������� ���������� ���, �������� ������������ ������ �������� ����������� ������ � ����������� �����. ������ ����� �������� � ����������� ������ � ���������� ������ ���������, ������� �������� ������ �������� ������������ � ���������� ������ ��� ��������� �� ��������� � ����������� �������.

����� ���������� ����� �����, �������������� ��������� � �������: �������� ������ �� ������ ��� ���������� ������ � ������. ��� ���� ���� ������ �� ����������� ������ ������� � ���-������, �������� ������ ����� �������� ������������. ��� ������� ��� ��������� ������������� ������ � �������� �������.

\subsection{��������� �������� ��������� �������� ������}

�� ������ �������������� ������ ���������� ��������� � ������ ����� ��������� ����������� ��� ������� ����, ������� ��� ����� \emph{��������� ��������} ��������� �������� ������. ����� ���������, ����� $\{(I_i, R_i, \{As\}_i, C_i, T_i)\}_{i=1, 2, ..., n}$ -- �������� ������, $\{I_i\}_{i=1, 2, ..., n}$ -- ������������������ ����������, $R_i$ -- ������� � �������, $\{As\}_i$ -- ��������� ����������, �������� ����� � ������, $\{C_i\}_{i=1, 2, ..., n}$ -- ������������������ �������� �������� � ���-������, $\{T_i\}_{i=1, 2, ..., n}$ -- ������������������ �������� �������� � TLB. ��������� TLB ����� ��������� �������������� ������, ������� ���� ��� ���-������, �� � TLB ����� �������� ���-��������� � ���-�������. ����� ���������� ����� ���� ������������ � ���� ��������� ��������� ��� ������� $i$:
$$
\left\{
   \begin{array}{l}
    virtualAddress_i = CalculateVirtualAddress(\{As\}_i) \\
    AddressTranslation(T_i, physicalAddress_i, virtualAddress_i)\\
    CacheAccess(C_i, physicalAddress_i)\\
    MemoryAccess(I_i, R_i, physicalAddress_i)\\
   \end{array}
  \right.
$$
��� $virtualAddress_i$ � $physicalAddress_i$ -- ����� ����������, $CalculateVirtualAddress$ -- �������, ����������� ����������� ����� �� ������ ���������� ����������. $AddressTranslation$ -- ��������, ����������� ���������� ������������ ������ � ���������� (����� ����� ���� ������������ TLB). $CacheAccess, MemoryAccess$ -- ���������, ����������� ��������� � ������ (� ������ ����� ���� ������������� ���-������, �� ������ -- �������� ������).

��������� ������� ��������� ��� ��������� ������� ������������ ��� ���������� ������ ��� ������ ����������, �� ������� �� ���������� ����� ���� �������� � ��������� ��������� (� ���� ����������� �����������). ����� ������� ���������� ��������� ���������:
\begin{itemize}
\item \emph{������ �� TLB}
$$
\left\{
   \begin{array}{l}
    AddressTranslation(T_1, physicalAddress_1, virtualAddress_1)\\
    AddressTranslation(T_2, physicalAddress_2, virtualAddress_2)\\
    ...\\
    AddressTranslation(T_n, physicalAddress_n, virtualAddress_n)\\
   \end{array}
  \right.
$$
\item \emph{������ �� ���-������}
$$
\left\{
   \begin{array}{l}
    CacheAccess(C_1, physicalAddress_1)\\
    CacheAccess(C_2, physicalAddress_2)\\
    ...\\
    CacheAccess(C_n, physicalAddress_n)\\
   \end{array}
  \right.
$$
\item \emph{������ �� �������� ������}
$$
\left\{
   \begin{array}{l}
    MemoryAccess(I_1, R_1, physicalAddress_1)\\
    MemoryAccess(I_2, R_2, physicalAddress_2)\\
    ...\\
    MemoryAccess(I_n, R_n, physicalAddress_n)\\
   \end{array}
  \right.
$$
\end{itemize}

��� ��� ������ �� ���-������ ��� �������� ������~\ref{cache_sets}. � ��� ���� ��������, ��� ��� ���� ������ ���������� ������ ���������� � ��������� ���������� ���-������. ��� ��������� ����� ������� � ������� ������ (��������, � �������-����������� ���-������) � ������������ ����� ������������� ������� ��������. ��������, ���� ...............

������ �� �������� ������ ������ ������������ ����� ���������� ���������, ����������� �������� � ���������� ����� ����������� ������. ���� ����������� �������� ������ � ���� ����������� ������� $memory$, ���������� � ������� ���� �� ���������� �������, ��
\begin{itemize}
\item ��� ����������, �������������� �������� �� ������, $MemoryAccess$ ����� ������������ ��� $R_i := memory[physicalAddress_i]$;
\item ��� ����������, �������������� ���������� � ������, $MemoryAccess$ ����� ������������ ��� $memory[physicalAddress_i] := R_i$.
\end{itemize}
����� �������, ���������� ������������������ ������������, ������� ����� ���� ������������� � ������� ��������� � ������� \emph{�������� ���������} (��� \emph{��������������})~\cite{Ackermann}. � ������,
\begin{itemize}
\item ��� ������ ������������� ���� ���������� (�� ����������� ����������� ������ � �������� �������, �� � ��� �� �������) $STORE(R_1, p_1)$ � $LOAD(R_2, p_2)$ ��������� �����������
    $$ (p_1 = p_2 \wedge p_2 \notin \{ p_{(1)}, p_{(2)}, ..., p_{(k)}\}) \rightarrow R_1 = R_2$$
    ��� $p_{(1)}, p_{(2)}, ..., p_{(k)}$ -- ���������� ������ ���������� $STORE$, ������������� ����� ����� ������������ ���� ����;
\item ��� ������ ������������� ���� ���������� (�� ����������� ����������� ������ � �������� �������, �� � ��� �� �������) $LOAD(R_1, p_1)$ � $LOAD(R_2, p_2)$ ��������� �����������
    $$ (p_1 = p_2 \wedge p_2 \notin \{ p_{(1)}, p_{(2)}, ..., p_{(k)}\}) \rightarrow R_1 = R_2$$
    ��� $p_{(1)}, p_{(2)}, ..., p_{(k)}$ -- ���������� ������ ���������� $STORE$, ������������� ����� ����� ������������ ���� ����.
\end{itemize}

������ �� TLB ������ �������� � ���� ����������� ������������ ����� ��������� ���������� (����������) TLB, ������������ ��������, ����������� �������� � ��������� � TLB �������������� � ������ �������. ��������� ���������� TLB ����� ����� ���� ����������� � ���� ������� �������, �� ��� ������ �� TLB ���� ��������� ��������������. ����� ����, ����� ����� ����� ���� ������������ ������ ���������� ��������� �� ��������� ����� ��� �������� �������� �������� �� ������, ������� ����� ���� ��� ���-������, ���� ������� ������������ � TLB.

��� ���������� ���������� ����������� ����������� �������� CSP (Constraint Satisfaction Problem)~\cite{CSP}. ......��� ��� ��, ��� ����� CSP, ��� �������� CSP, ������� ������, ����� ����������� ����� ������ � ������� CSP (���������) - ���������� ������� CSP.

������������ ���������� ��������� �������� �������� ���������� �����������. ������ ������������ �������� ��� �������� �� ��������� � ���������� ������ ��������� ���������������. ��� �������� ������� Genesys-Pro...\cite{GenesysPro},\cite{DeepTrans} ��� ��������� - � �������� �������� ��� ����� �����������........

������ � ���������� ��������� ���� 

\subsection{����������� �������� ��������� �������� ������: ������� 0}

\subsection{����������� �������� ��������� �������� ������ ����� ������� �������}

\pagebreak
\section{��������� ����� �� ���-������ � �����������
���������������}

\pagebreak
\bibliographystyle{plain}
\bibliography{theor}

\end{document}


\section*{Задачи}

% !Mode:: "TeX:UTF-8"

\zhead{Посчитать значения функций}

\z Для следующей алгебраической спецификации посчитать значения указанных выражений:
\begin{lstlisting}
scheme Sets = class
  type Set, Elem
  value
        empty: Unit -> Set,
        add: Elem >< Set-> Set,
        delete: Elem >< Set -> Set,
        inset: Elem >< Set -> Bool
  axiom
    forall s: Set, x, y: Elem :-
        add(x, add(x, s)) is add(x, s),
        delete(x, empty() ) is delete(x, empty()),
        delete(x, add(x, s)) is delete(x, s),
        delete(x, add(y, s)) is add(y, delete(x,s)),
        ~inset(x, empty() ),
        inset(x, add(y, s)) is (x = y \/ inset(x,s)),
        inset(x, delete(y,s)) is x ~= y /\ inset(x,s)
end
\end{lstlisting}
\begin{itemize}
  \item inset(1, add(1, empty()));
  \item inset(1, delete(1, empty()));
  \item inset(1, delete(1, add(1, empty())));
  \item delete(1, add(1, empty()));
  \item delete(1, add(2, empty()));
  \item delete(1, add(1, add(1, empty())));
  \item delete(1, add(2, add(1, empty())));
  \item add(2, delete(1, add(2, add(1, empty()))));
  \item inset(3, add(2, delete(1, add(2, add(1, empty())))));
  \item inset(3, add(2, delete(3, add(2, add(1, empty())))));
  \item inset(3, add(3, delete(1, add(2, add(1, empty())))));
  \item inset(3, add(3, delete(3, add(3, add(3, empty())))));
\end{itemize}

Решение (1):
\begin{lstlisting}
inset(1, add(1, empty())) is
   1 = 1 \/ inset(1, empty()) is true
\end{lstlisting}

Решение (3):
\begin{lstlisting}
inset(1, delete(1, add(1, empty()))) is
    1 ~= 1 /\ inset(1, add(1, empty())) is false
\end{lstlisting}

Решение (4):
\begin{lstlisting}
delete(1, add(1, empty())) is delete(1, empty())
\end{lstlisting}
В данном случае произвести другие упрощения невозможно.

Обратите внимание, что
\begin{enumerate}
  \item аксиома delete(x, empty() ) is delete(x, empty()) бесполезная, потому что является тождественной истиной; ее можно безболезненно исключить из спецификации;
  \item легко выписать набор аксиом, с которым крайне тяжело работать, из них тяжело извлечь полезный смысл (тяжело ответить на нужные вопросы) --- они сводятся не к произведению полезных выводов (в т.ч. вычислений), а бездумным переписываниям.
\end{enumerate}



% !Mode:: "TeX:UTF-8"

\zhead{Ошибки в спецификациях}

\z Специфицируется АТД <<множество>> с операциями добавления и удаления элементов. Правильна ли следующая спецификация?
\begin{lstlisting}
scheme Sets = class
  type Set, Elem
  value
        empty: Unit -> Set,
        add: Elem >< Set-> Set,
        delete: Elem >< Set -> Set
  axiom
    forall s: Set, x, y: Elem :-
        add(x, add(x, s)) is add(x, s),
        delete(x, empty() ) is empty(),
        delete(x, add(x, s)) is s,
        delete(x, add(y, s)) is add(y, delete(x,s))
                pre x ~= y
end
\end{lstlisting}

Решение: не соответствует (delete может оставить во множестве элемент во множестве, не удалив его, если он добавлялся несколько раз): введем функцию
\begin{lstlisting}
inset : Elem >< Set -> Bool
\end{lstlisting}
Она истинна тогда и только тогда, когда элемент есть во множестве.
\begin{lstlisting}
delete(1, add(1, add(1, empty))) is add(1, empty)
\end{lstlisting} по третьей аксиоме, но тогда
\begin{lstlisting}
inset(1, delete(1, add(1, add(1, empty)))) is
    inset(1, add(1, empty)) is true
\end{lstlisting}
а должен быть \textbf{false}, т.к. единицу должны были удалить.

Обратите внимание, что:
\begin{enumerate}
  \item методом исследования алгебраических спецификаций является исследование значений термов;
  \item чтобы получать следствия, не обязательно <<уменьшать>> терм --- может потребоваться <<навесить>> на него другие операции.
\end{enumerate}

Ответьте на вопросы:
\begin{enumerate}
  \item является ли спецификация полной?
  \item описывает ли она какие-нибудь операции полностью?
  \item является ли спецификация непротиворечивой?
\end{enumerate}

\z Специфицируется АТД <<множество>> с операциями добавления и удаления элементов. Правильна ли следующая спецификация? Полна ли? Согласована ли?
\begin{lstlisting}
scheme Sets = class
  type Set, Elem
  value
        empty: Unit -> Set,
        add: Elem >< Set-> Set,
        delete: Elem >< Set -> Set
  axiom
    forall s: Set, x, y: Elem :-
        add(x, add(x, s)) is add(x, s),
        delete(x, empty() ) is empty(),
        delete(x, add(x, s)) is delete(x, s),
        delete(x, add(y, s)) is add(y, delete(x,s))
                pre x ~= y
end
\end{lstlisting}

\z Специфицируется АТД <<множество>> с операциями добавления и удаления элементов. Правильна ли следующая спецификация? Полна ли? Согласована ли?
\begin{lstlisting}
scheme Sets = class
  type Set, Elem
  value
        empty: Unit -> Set,
        add: Elem >< Set-> Set,
        delete: Elem >< Set -> Set
  axiom
    forall s: Set, x, y: Elem :-
        add(x, add(x, s)) is add(x, s),
        delete(x, empty() ) is delete(x, empty()),
        delete(x, add(x, s)) is delete(x, s),
        delete(x, add(y, s)) is add(y, delete(x,s))
end
\end{lstlisting}

\z Специфицируется АТД <<очередь>> с операциями добавления и удаления элементов. Правильна ли следующая спецификация? Полна ли? Согласована ли?
\begin{lstlisting}
scheme Queue = class
  type Q, Elem
  value
        empty: Unit -> Q,
        add: Elem >< Q-> Q,
        delete: Elem >< Q -> Q
  axiom
    forall s: Q, x, y: Elem :-
        add(x, add(x, s)) is add(x, s),
        delete(x, empty() ) is empty(),
        delete(x, add(x, s)) is s,
        delete(x, add(y, s)) is add(y, delete(x,s))
                pre x ~= y
end
\end{lstlisting}

\z Специфицируется АТД <<стек>> с операциями добавления и удаления элементов. Правильна ли следующая спецификация? Полна ли? Согласована ли?
\begin{lstlisting}
scheme List = class
  type L, Elem
  value
        empty: Unit -> L,
        add: Elem >< L -> L,
        delete: Elem >< L -> L
  axiom
    forall s: L, x, y: Elem :-
        add(x, add(x, s)) is add(x, s),
        delete(x, empty() ) is empty(),
        delete(x, add(x, s)) is s,
        delete(x, add(y, s)) is add(y, delete(x,s))
                pre x ~= y
end
\end{lstlisting}


% !Mode:: "TeX:UTF-8"

\zhead{Неформальные интерпретации}

\z Для данной алгебраической спецификации дать неформальную интерпретацию ее типов (терминов) и операций; старайтесь увидеть случаи, в которых поведение функций не определяется в спецификации:
\begin{lstlisting}[escapechar={|}]
type E, S
value capacity: Nat,
    empty: Unit -~-> S,
    |is\_empty|: S -> Bool,
    push: S >< E -~-> S,
    pop: S -~-> S >< E,
    top: S -> Nat,
    elem: S >< Nat -~-> E,
    f: S >< E -> S f(s,e) is s,
    sc: S >< E -> S sc(s,e) is e
axiom forall s: S, e: E, i: Nat :-
  top( empty() ) is 0,
  top( push(s, e) ) is top(s) + 1
            pre top(s) < capacity ,
  top( pop(s, e) ) is top(s) - 1
            pre ~|is\_empty|(s) ,
  elem( push(s, e), top(s) + 1 ) is e
            pre top(s) < capacity ,
  elem( push(s,e), i ) is elem(s, i)
        pre i ~= top(s) + 1 /\ top(s) < capacity,
  elem( f(pop(s)), i ) is elem(s, i)
            pre i ~= top(s),
  elem( s, top(s) ) is sc(pop(s))
            pre ~|is\_empty|(s),
  |is\_empty|( empty() ),
  ~|is\_empty|( push(s,e) ),
  capacity > 0
\end{lstlisting}

Решение:
\begin{itemize}
  \item S --- структура конечного размера (capacity --- максимальный размер);
  \item empty возвращает пустую структуру;
  \item is\_empty проверяет, пуста ли структура;
  \item push помещает элемент в структуру, если после этого размер не переполнится;
  \item pop удаляет из структуры последний добавленный элемент и возвращает его вместе с измененной структурой, если структура не пустая;
  \item top вычисляет количество элементов (размер структуры);
  \item elem возвращает элемент, хранящийся в структуре по данному индексу, если значение индекса не выходит за допустимые рамки.
\end{itemize}

\z Для данной алгебраической спецификации дать неформальную интерпретацию ее типов (терминов) и операций; старайтесь увидеть случаи, в которых поведение функций не определяется в спецификации:
\begin{lstlisting}
type Owner, Auto, Warrant
value
    empty: DB,
    add: Owner >< DB -~-> DB,
    known: Owner >< DB -> Bool,
    known: Auto >< DB -> Bool,
    known: Warrant >< DB -> Bool,
    add: Owner >< Auto >< DB -~-> DB,
    add: Auto >< Warrant >< DB -~-> DB,
    get_owner: Auto >< DB -~-> Owner
axiom forall o1, o2 : Owner,
    a, a2 : Auto, db : DB, w, w2 : Warrant :-
    ~ known(o1, empty),
    known(o1, add(o2, db)) is o1 = o2 \/ known(o1, db)
            pre ~known(o2, db),
    known(o1, add(o2,a,db)) is known(o1,db)
            pre known(o2,db) /\ ~known(a,db),
    known(o1, add(a,w,db)) is known(o1,db)
            pre known(a,db) /\ ~known(w,db),

    ~ known(a, empty),
    known(a, add(o1,db)) is known(a,db)
        pre ~known(o1,db),
    known(a, add(o2,a2,db)) is a = a2 \/ known(a,db)
        pre known(o2,db) /\ ~known(a2,db),
    known(a, add(a2,w,db)) is known(a,db)
        pre known(a2,db) /\ ~known(w,db),

    ~known(w, empty),
    known(w, add(o, db)) is known(w, db)
        pre ~known(o, db),
    known(w, add(a,w2, db)) is w = w2 \/ known(w,db)
        pre known(a,db) /\ ~known(w2,db),
    known(w, add(o, a, db)) is known(w,db)
        pre known(o, db) /\ ~known(a,db),

    get_owner(a, add(o,db)) is get_owner(a, db)
        pre ~known(o, db) /\ known(a, db),
    get_owner(a, add(o, a2, db)) is if a = a2 then o
        else get_owner(a, db) end
        pre known(o,db) /\ ~known(a2, db) /\
            (a = a2 \/ known(a,db)),
    get_owner(a, add(a2,w,db)) is get_owner(a,db)
        pre known(a2,db) /\ ~known(w,db) /\
            known(a,db)
\end{lstlisting}

\z Для данной алгебраической спецификации дать неформальную интерпретацию ее типов (терминов) и операций; старайтесь увидеть случаи, в которых поведение функций не определяется в спецификации:
\begin{lstlisting}
type GTree, Person
value
    empty: Person -> GTree,
    add_child: Person >< Person >< GTree -~-> GTree,
    add_parent: Person >< Person >< GTree -~-> GTree,
    known: Person >< GTree -> Bool,
    get_childs: Person >< GTree -~-> Person-set
axiom forall p,p1,p2: Person, g: GTree :-
    known(p, empty(p1)) is p = p1,
    known(p, add_child(p1, p2, g)) is p = p2 \/ known(p,g)
        pre ~known(p2, g) /\ known(p1,g),
    known(p, add_parent(p1,p2,g)) is p = p2 \/ known(p,g)
        pre ~known(p2,g) /\ known(p1,g),

    get_childs(p, empty(p)) is {},
    get_childs(p, add_child(p1, p2,g)) is
      (if p = p1 then {p2} else {} end) union get_childs(p,g)
      pre known(p1,g) /\ ~known(p2,g) /\ known(p,g),
    get_childs(p, add_parent(p1,p2,g)) is get_childs(p,g)
      pre known(p1,g) /\ ~known(p2,g) /\ known(p,g)
\end{lstlisting}

\z Для данной алгебраической спецификации дать неформальную интерпретацию ее типов (терминов) и операций; старайтесь увидеть случаи, в которых поведение функций не определяется в спецификации:
\begin{lstlisting}
type RwS, Station, Train, TrainNumber,
    Time = Nat, Platform
value
    station: Station,
    platforms: Platform-set,
    getStart: Train -> Station,
    getEnd: Train -> Station,
    getNumber: Train -> TrainNumber,
    getStay: Train -~-> Time >< Time,
    empty: RwS,
    add_train: Train >< Platform >< RwS -~-> RwS,
    free: Platform >< Time >< Time >< RwS -~-> Bool,
    trains_tos: Station >< RwS -~-> Train-set
axiom forall t1,t2 : Time, t, tr1, tr2 : Train,
                    r : RwS, s: Station :-
    free(p, t1, t2, empty) pre p isin platforms,
    free(p, t1, t2, add_train(t,p2,r)) is
        if p ~= p2 then free(p, t1, t2,r)
        else let (s1,s2) = getStay(t) in
            t2 < s1 \/ s2 < t1 end end ),
        pre {p,p2} <<= platforms /\ t1 < t2,

    trains_tos(s, empty) is {},
    trains_tos(s, add_train(t,p,r)) is
        (if s = getEnd(t) then {t} else {} end)
            union trains_tos(s,r)
            pre p isin platforms /\
            free(p, getStart(t), getEnd(t), r),

    let (s1,s2) = getStay(t) in s1 <  s2 end,
    (getNumber(tr1) = getNumber(tr2)) is tr1 = tr2,
    platforms ~= {}
\end{lstlisting}



% !Mode:: "TeX:UTF-8"
\zhead{Выбор генераторов}

\z Для стека, определенного таким образом:
\begin{lstlisting}
type E, S == empty | push(S, E) | pop(S, E)
\end{lstlisting}
дать алгебраическую спецификацию операции проверки наличия заданного элемента
\begin{lstlisting}
value include: S >< E -> Bool
\end{lstlisting}

\textbf{Решение:}
\begin{lstlisting}
type E, S == empty | push(S, E) | pop(S)
value include: S >< E -> Bool
axiom forall e,e1:E, s:S :-
  ~ include(e, empty),
  include(e, push(s,e1)) is e = e1 \/ include(e,s),
  include(e, pop(s)) is
     if e = top(s) then count(e,s) > 1
             else count(e,s) > 0 end  pre s ~= empty
value count: E >< S -> Nat,
      top: S -~-> E
axiom forall e,e1:E, s:S :-
   count(e, empty) is 0,
   count(e, push(s,e1)) is count(e,s) +
      if e = e1 then 1 else 0 end,
   count(e, pop(s)) is count(e,s) -
      if e = top(s) then 1 else 0 end pre s ~= empty,

   top(push(s,e1)) is e1,
   top(pop(s)) is last(s,2) pre size(s) >= 2

value last: S >< Nat -~-> E,
      size: S -> Nat
axiom forall s:S, e:E, n:Nat :-
   size(empty) is 0,
   size(push(s,e)) is size(s) + 1,
   size(pop(s)) is size(s) - 1 pre s ~= empty,

   last(push(s,e), n) is
      if n = 1 then e else last(s, n-1) end
        pre n > 0 /\ n <= size(s),
   last(pop(s), n) is last(s, n+1)
        pre s ~= empty /\ n > 0 /\ n < size(s)
\end{lstlisting}

Обратите внимание, что
\begin{enumerate}
\item при помощи выбранных для описания стека генераторов значение стека будет иметь вид, например, push(push(pop(push(empty,1)),10),1); такое выражения значения в типе стек может казаться наиболее адекватным, ведь добавление и удаление элементов --- именно те операции, при помощи которых можно изменить значение (<<состояние>>) стека; однако посмотрите, насколько увеличивается спецификация и теряется ее наглядность, если в число генераторов включена лишняя функция (pop); сравните:
\begin{lstlisting}
type E, S == empty | push(S, E)
value include: S >< E -> Bool
axiom forall e,e1:E, s:S :-
  ~ include(e, empty),
  include(e, push(s,e1)) is e = e1 \/ include(e,s),
\end{lstlisting}

\item в этом примере синтаксически разные термы из генераторов могут означать одинаковые значения типа, например, push(pop(push(empty,1)),2) и push(pop(push(empty,3)),2); в таких случаях надо быть внимательными при выписывании аксиом: помнить и понимать, сколько разных возможностей есть для рекурсивной части цепочки (речь идет о переменной <<s>> в примерах) --- например, в аксиоме с генератором, удаляющим элемент, надо в том числе предполагать, что этот элемент может появиться много раз до этого, добавляться и удаляться.
\end{enumerate}

%% минимальный набор генераторов уменьшает спецификацию (оценить количество аксиом?)

%% при нескольких генераторах надо быть аккуратными

\z Для множества, определенного таким образом:
\begin{lstlisting}
type E, S == empty | add(S, E) | delete(S, E)
\end{lstlisting}
дать алгебраическую спецификацию операции проверки наличия заданного элемента
\begin{lstlisting}
value include: S >< E -> Bool
\end{lstlisting}


\zhead{Описание эффекта  на основе структуры терма}

Вы уже заметили, что основной принцип написания аксиомы --- понять, как вычисляется обсервер после последнего сработавшего генератора. При этом для выражения этой аксиомы используются аргументы, с которыми вызван обсервер и последний генератор. Однако не всегда просто описать эффект работы генератора на основе лишь аргументов последнего из них.

\z Дать алгебраическую спецификацию типа <<Ограниченная очередь>>. В эту очередь можно добавлять и удалять элементы, но только если количество хранящихся элементов не превышает заданную величину.

\textbf{Решение:}
\begin{lstlisting}
value capacity : Nat
type E, S == empty | add(E,S)
value delete: S -~-> E
axiom forall e:E, s:S :-
   delete(add(e,s)) is s
     pre size(s) < capacity

value size: S -> Nat
axiom forall e:E, s:S :-
   size(empty) is 0,
   size(add(e,s)) is size(s) + 1
       pre size(s) < capacity
\end{lstlisting}

Обратите внимание, что
\begin{enumerate}
\item пришлось добавить и описать дополнительный обсервер size;
\item существует множество способов реализации ограниченной очереди при помощи имеющихся в языках программирования средств (например, при помощи <<кольцевой очереди>>, реализованной при помощи массива и двух указателей), однако здесь предъявляется именно формализация функциональности операций, которая остается справедливой и неизменной для любой реализации <<ограниченной очереди>>;
\item данное описание не дает определения того, в каких случаях определена каждая операция в отдельности;
\item при написании аксиомы для рекурсивной части терма достаточно представлять только \emph{правильно построенные термы} --- такие термы, в которых все функции вызваны с аргументами правильных типов и в каждой функции выполнено ее предусловие; например, для аксиомы size(add(e,s)) не надо представлять, что она должна описывать и такой терм: size(add(e,add(e1,add(e2,empty)))) при capacity = 2.
\end{enumerate}

\z Дать алгебраическую спецификацию типа <<Исключающая очередь>>. В эту очередь элемент добавляется в том случае, если его не было, а если он там был, то он удаляется из очереди. Опишите операцию проверки наличия заданного элемента в такой очереди.


\zhead{<<Эффект>> операций}
%% приходится добавлять обсерверы, чтобы полностью описать эффект функции

\z В~\cite{tanenbaum_os} описаны операции с файлами, среди них описана операция Create следующим образом: <<\textsf{Create} (создание). Файл создается без данных. Этот системный вызов объявляет о появлении нового файла и позволяет установить некоторые его атрибуты.>> Напишите алгебраическую спецификацию файловой подсистемы с этой операцией. Естественно, вам понадобится сигнатура этой операции. Вот она:  \texttt{int creat(char *path, int mode)}, параметр \texttt{path} содержит полное или относительное имя файла, параметр \texttt{mode} устанавливает атрибуты прав доступа различных категорий пользователей к новому файлу при его создании (если файл уже существовал, то новый не создается), операция возвращает значение файлового дескриптора для открытого файла при нормальном завершении и значение -1 при возникновении ошибки.

Решение:
\begin{lstlisting}
scheme FS = class
  type Path, Mode, FID, FS
  value
        creat : Path >< Mode >< FS -~-> FS >< FID,
        size: FS >< FID -~-> Nat,
        known: FS >< Path -> Bool,
        access: FS >< FID -~-> Mode,
        first: FS >< FID -> FS
        first(a,b) is a
  axiom
    forall path: Path, mode: Mode, fs: FS :-
        size(creat( path, mode, fs )) is 0,
        known(first(creat( path, mode, fs )), path),
        access(creat( path, mode, fs)) is mode
end
\end{lstlisting}
Обратите внимание, что
\begin{enumerate}
  \item для описания эффекта функции \texttt{creat} были введены дополнительные операции-обсерверы;
  \item аксиомы напрямую выражают текст, описывающий операцию \texttt{creat} --- аксиомы формализуют \emph{требования} на эту операцию.
\end{enumerate}
Ответьте на следующие вопросы:
\begin{enumerate}
  \item эта спецификация неполная, почему? является ли она противоречивой? как, добавив 1 аксиому, полностью описать операцию known?
  \item имеют ли смысл сами по себе введенные дополнительные операции или они выполняют лишь вспомогательную для описания \texttt{creat} функцию?
  \item допустим, мы догадываемся, что Path = \textbf{Text}, а Mode = \textbf{Nat}; дополненная этим знанием спецификация, останется ли алгебраической ? станет ли полной ? останется ли непротиворечивой ? будет ли она соответствовать исходной постановке задачи ? не станет ли она допускать того, что не должно бы по условию ?
\end{enumerate}

\z В~\cite{tanenbaum_os} описаны операции с файлами, среди них описана операция Delete следующим образом: <<\textsf{Delete} (удаление). Когда файл уже более не нужен, его удаляют, чтобы освободить пространство на диске. Этот системный вызов присутствует в каждой операционной системе.>> Напишите алгебраическую спецификацию файловой подсистемы с этой операцией. Естественно, вам понадобится сигнатура этой операции. Вот она:  \texttt{void delete(int fid)}, параметр \texttt{fid} содержит значение файлового дескриптора.

\z В~\cite{tanenbaum_os} описаны операции с файлами, среди них описана операция Open следующим образом: <<\textsf{Open} (открытие). Прежде чем использовать файл, процесс должен его открыть. Системный вызов open позволяет системе прочитать в оперативную память атрибуты файла и список дисковых адресов для быстрого доступа к содержимому файла при последующих вызовах.>> Напишите алгебраическую спецификацию файловой подсистемы с этой операцией. Естественно, вам понадобится сигнатура этой операции. Вот она:  \texttt{void delete(int fid)}, параметр \texttt{fid} содержит значение файлового дескриптора.

\z В~\cite{tanenbaum_os} упомянут системный вызов mmap: <<Системный вызов mmap принимает на входе два параметра: имя фала и виртуальный адрес паямти, по которому операционная система отображает указанный файл. Для реализацияи отображения файлов на память изменяются системные внутренние таблицы.При обращении к памяти по адресу от 512 до 576К происходит прерывание из-за отсутствия страницы, обработчик которого предоставляет считанную в память страницу 0 файла.Если потом эта страница удаляется из памяти алгоритмом замены страниц, она записывается в соответствующее место файла.>>

%% не всегда просто понять, полна ли спецификация
\zhead{Противоречивость алгебраических спецификаций}

Если спецификация допускает подстановку и <<вычисление>> термов увеличивающейся длины, пытаться <<вычислять>> эти термы разными способами, которые допускают аксиомы, и проверять, получаются ли одинаковые результаты в разных способах. Если получились разные, значит, найдено противоречие.

\z Противоречива ли следующая спецификация?
\begin{lstlisting}
type T
value empty : T,
    put: T >< Nat -> T,
    get: T >< Nat -> Bool
axiom forall t: T, x, y, z: Nat :-
  put( put(t, x), x ) is put(t, x),
  ~get( empty, x ),
  get( put(empty, x), y ) is (x = y /\ x\2 = 0),
  get(put(put(empty,x),y),z) is (z=x /\ y\2=0) pre x\2=0,
  get( put(t, y), x ) is get(t, x) pre y ~= x,
  get( put( put(t, x), y), x ) is get( put(t,x), x ),
  ~get(put(put(t,x),x+1),x+1) pre ~get(t,x+1) /\ get(t,x)
\end{lstlisting}

\z Противоречива ли следующая спецификация?
\begin{lstlisting}
type T
value empty: T,
    put: T >< Nat -> T,
    get: T >< Nat -> Bool
axiom forall t:T, x, y, z: Nat :-
  put(put(t, x), x) is put(t, x),
  get( put (put(t, 0), x ), x) is (x > 0),
  get( put (put(t, 2*x), x ), x) is (x > 0),
  ~get( empty, x ),
  ~get( put(empty,x), y ),
  get(put(put(empty,x),y),z) is (z>0 /\ z=abs(x-y) )
\end{lstlisting}

\z Противоречива ли следующая спецификация?
\begin{lstlisting}
type T = Nat
value empty: T,
    put: T >< Nat -> T,
    get: T >< Nat -> Bool
axiom forall t: T, x, y, z: Nat :-
  put( put(t, x), y ) is put(t,x) pre y <= x,
  ~get( empty, x ),
  get( put(empty, x), y ) is (x = y),
  get(put(put(empty,x),y),z) is (z = x \/ z = y /\ y > x),
  get( put(t,x), y ) is get(t,y) pre y ~= x,
  ~get(put(t,2*y),2*y) pre get(t,y) /\
       get(t,y+1) /\~get(t,2*y),
  get( put(put(t, 0), x), x ) is get( put(t,x), x )
\end{lstlisting}

\z Противоречива ли следующая спецификация?
\begin{lstlisting}
type T
value empty: T,
    put: T >< Nat -> T,
    get: T >< Nat -> Bool
axiom forall t: T, x, y, z: Nat :-
  put( put(t, x), y ) is put(t,x) pre y <= x,
  ~get( empty, x ),
  get( put(empty, x), y ) is (x = y),
  get(put(put(empty,x),y),z) is (z = x \/ z = y /\ y > x),
  get( put(t,x), y ) is get(t,y) pre y ~= x,
  ~get( put(t, 2*y), 2*y ) pre get(t,y) /\ get(t, y+1),
  get( put(put(t, 0), x), x ) is get( put(t,x), x )
\end{lstlisting}

\z Противоречива ли следующая спецификация?
\begin{lstlisting}
type T
value empty: T,
    put: T >< Nat -> T,
    get: T >< Nat -> Bool
axiom forall t: T, x, y, z: Nat :-
  put( put(t,x), x ) is put(t,x),
  ~get( empty, x ),
  get( put(empty, x), y ) is (x = y),
  get(put(put(empty,x),y),z) is (z=x \/ z=y /\ y>2),
  get( put(t,x), y ) is get(t, y) pre y ~= x,
  get( put(t,y), y ) pre get( put(t,x), x ) /\ x <= y,
  ~get(t,x) pre get( put(t, 0), 0),
  ~get(t, x) /\ ~get(t,y) pre x ~= y /\ get(put(t,1),1)
\end{lstlisting}

\z Противоречива ли следующая спецификация?
\begin{lstlisting}
type T
value empty: T,
    put: T >< Nat -> T,
    get: T >< Nat -> Bool
axiom forall t: T, x, y, z: Nat :-
  put( put(t,x), x ) is put(t,x),
  ~get( empty, x ),
  get( put(empty, x), y ) is (x = y),
  get( put( put(empty,x), y ), z ) is (z = x \/ z = y /\ y > 0),
  get( put(t,x), y ) is get(t, y) pre y ~= x,
  get( put(t,y), y ) pre get( put(t,x), x ) /\ x <= y,
  ~get(t,x) pre get( put(t, 0), 0) /\ ~get(t,0),
  ~get(t,x) \/ ~get(t,y) pre x~=y /\ get(put(t,1),1) /\ ~get(t,1)
\end{lstlisting}

\zhead{Полнота алгебраических спецификаций}

Не забывайте, что у нас есть только система аксиом и логика, т.е. правила получения новых выражений из имеющихся. За символами имен операций не стоит никакой семантики, даже если она <<предполагалась>> автором. Наоборот, эта система аксиом должна \emph{дать} нам семантику символов, т.е. дать нам возможность сделать с этими символами некие осмысленные действия.

\z Полно ли описывает следующая спецификация тип <<Множество>> ?
\begin{lstlisting}
type E, S
value empty: S,
      add: E >< S -> S,
axiom forall e1, e2: E, s : S :-
   add(e1, add(e1, s)) is add(e1, s),
   add(e1, add(e2, s)) is add(e2, add(e1, s))
\end{lstlisting}

\textbf{Решение:}
В этой спецификации нет ни одного обсервера, поэтому вопрос о полноте для нее некорректен.

\z Полна ли следующая спецификация типа <<Очередь>> ?
\begin{lstlisting}
type E, Q
value empty: Q,
      add: E >< Q -> Q,
      size: Q -> Nat
axiom forall e: E, q : Q :-
    size(empty) is 0,
    size(add(q, e)) is size(q) + 1
\end{lstlisting}

\textbf{Решение:}
Обсервер --- size. Он определен для empty и определен для генератора add. Значит, он определен для любого терма, дающего тип Q. Значит, функция size описана полно.

\z Полна ли следующая спецификация типа <<Очередь>> ?
\begin{lstlisting}
type E, Q
value empty: Q,
      add: Q >< E -> Q,
      size: Q -> Nat
axiom forall e: E, q : Q :-
    size(empty) is 0,
    size(add(q, e)) is size(q) + 1,
    add(add(q, e), e) is add(q, e)
\end{lstlisting}

\textbf{Решение:}
Без учета последней аксиомы size(add(add(q,e),e)) is size(add(q,e)) + 1 is size(q) + 2. А теперь с последней аксиомой: size(add(add(q,e),e)) is size(add(q,e)) is size(q) + 1. Получается, что size(q) + 1 = size(q) + 2. Иными словами, из аксиом следует ложь, система аксиом противоречива. А раз так, вопрос о полноте некорректен.


\z Полна ли следующая спецификация типа <<Очередь>> ?
\begin{lstlisting}
type E, Q
value empty: Q,
      add: Q >< E -> Q,
      first: Q -~-> E,
      size: Q -> Nat
axiom forall e: E, q : Q :-
    first(add(empty, e)) is e,
    first(add(q, e)) is first(q) pre q ~= empty,
    size(empty) is 0,
    size(add(q,e)) is size(q) + 1
\end{lstlisting}

\textbf{Решение:}
Любое значение в целевом типе Q имеет один из двух видов (просто напросто, нет других функций, возвращающих Q):
\begin{itemize}
  \item empty
  \item add(add(add(...add(add(empty, e1), e2)...)
\end{itemize}

Посмотрим на first с точки зрения определения достаточной полноты. Обе аксиомы для простоты можно объединить в одну: first(add(q,e)) is if q = empty then e else first(q) end. Тогда такое выражение <<вычислить>> можно: first(add(empty,e1)) is e1 (по первой аксиоме). Посмотрим такое выражение: first(add(add(empty,e1),e2)) is if add(empty,e1) = empty then e2 else e1 end. Чтобы закончить <<вычисление>>, надо понять истинность выражения add(empty,e1) = empty. В общем случае дать ответ на этот вопрос нельзя (\emph{проблема равенства термов алгоритмически неразрешима}). Однако в данном случае ответить на этот вопрос можно.

Для этого сделаем такой хитрый ход -- \emph{<<навесим>> на эти два выражения сверху другой обсервер}: size(empty) is 0, size(add(empty,e1)) is size(empty) + 1 is 0 + 1 is 1, т.е. size(empty) ~= size(add(empty,e1)). Поскольку size --- тотальная функция, то она детерминированная, т.е. all x, y : Q :- x = y => size(x) = size(y), что то же самое, что size(x) ~= size(y) => x ~= y. Теперь в качестве x возьмем empty, а в качестве y возьмем add(empty, e1). Получим, что add(empty, e1) ~= empty. Ура, желаемое доказано! Тем самым, можно и вычислить второй терм, он равен e1. Аналогично, можно вычислить и все остальные термы.

Осталось рассмотреть единственный терм: first(empty). Если бы существовали такие q' и e', что add(q',e') равнялось empty, то было бы возможно применение аксиом. Но, как следует из первой части, такие q' и e' не существуют, значит, <<вычислить>> first(empty) на основе данных аксиом нельзя. Ответ: неполна.

\z Полна ли следующая спецификация типа <<Очередь>> ?
\begin{lstlisting}
type E, Q
value empty: Q,
      add: Q >< E -> Q,
      first: Q -~-> E,
axiom forall e: E, q : Q :-
    first(add(empty, e)) is e,
    first(add(q, e)) is first(q) pre q ~= empty,
\end{lstlisting}

\textbf{Решение:}
Рассуждая аналогично предыдущей задаче, приходим к вопросу об истинности add(empty,e1) = empty и в данной системе аксиом дать ответ на этот вопрос нельзя. Ответ: неполна.

%% вставить примеры противоречивых систем аксиом из заданий экзамена прошлого года

%% понять, как описывать недопустимое поведение

%\zhead{Спецификация отношений <<многие-ко-многим>>}
%
%Алгебраические спецификации позволяют описать многие компоненты, осуществляющие отношение <<многие-ко-многим>>, совершенно не задумываясь о том, каким образом это отношение выразить чем-нибудь более известным (например, вспомните, сколько есть различных способов представления этого отношения для реляционной модели данных!)
%
%\z Специфицируйте компонент, отвечающий за хранение и модификацию данных о студентах Университета и спецкурсах. А именно, есть студенты, они добавляются в базу внутри этого компонента. Есть спецкурсы, которые также добавляются. Как-то студенты записываются на спецкурсы. Компонент позволяет

\zhead{Рекурсивные типы}

\z Специфицируйте операцию проверки вхождения элемента в бинарное дерево.

\textbf{Решение:}
\begin{lstlisting}
type Node, Tree == empty | add(Node, Tree, Tree)
value check: Node >< Tree -> Bool
axiom forall n, n1:Node, left, right: Tree :-
   ~check( empty ),
   check(n, add(n1,left,right) ) is (n = n1) \/
             check(n, left) \/ check(n, right)
\end{lstlisting}

\z Специфицируйте операцию вычисления высоты бинарного дерева.

\z Специфицируйте операцию проверки бинарного дерева на сбалансированность.

\z Специфицируйте операцию получения предка элемента бинарного дерева.

\z Специфицируйте добавление элемента в двоичное дерево поиска.

\z Специфицируйте удаление элемента из двоичного дерева поиска.

\z Специфицируйте добавление элемента в АВЛ-дерево. Определение операции предполагается найти самостоятельно.

\z Специфицируйте удаление элемента из АВЛ-дерева. Определение операции предполагается найти самостоятельно.

\z Специфицируйте добавление элемента в 2-3-дерево. Определение операции предполагается найти самостоятельно.

\z Специфицируйте удаление элемента из 2-3-дерева. Определение операции предполагается найти самостоятельно.

\z Специфицируйте добавление элемента в декартово дерево. Определение операции предполагается найти самостоятельно.

\z Специфицируйте удаление элемента из декартова дерева. Определение операции предполагается найти самостоятельно.

\z Специфицируйте добавление элемента в красно-чёрное дерево.  Определение операции предполагается найти самостоятельно.

\z Специфицируйте удаление элемента из красно-чёрного дерева.  Определение операции предполагается найти самостоятельно.



\section{Полнота и непротиворечивость алгебраических спецификаций}

% !Mode:: "TeX:UTF-8"
\head{Непротиворечивые алгебраические модели}

Алгебраическая модель непротиворечива, если из нее, как теории, нельзя вывести тождественную ложь.

При исследовании на непротиворечивость не стоит забывать утверждений об операциях и типах, не выраженных явно. Например, для операции f: T1 -> T2, определенной тотальным образом, справедливо, что all x, y: T1 :- f(x) $\Not$= f(y) => x $\Not$= y. Другой случай --- ограничения на типы входных и выходных параметров функции из сигнатуры (например, если тип Nat, то соответтсвующее значение должно быть всегда больше или равно нулю).

\head{Полные алгебраические модели}

В~\cite{mayer} написано: <<Не существует формального определения интуитивно ясного понятия <<полноты>> спецификации абстрактного типа данных. Строго определяемое понятие достаточной полноты как правило обеспечивает удовлетворительный ответ. ... Вопрос о полноте: <<Как узнать, что уже специфицировано достаточно свойств и можно остановиться?>> ... Для математика некоторая теория является полной, если ее аксиомы и правила вывода являются достаточно мощными, чтобы доказать истинность или ложность любой формулы, выразимой в языке данной теории. ... Здесь <<язык теории>> --- это множество правильно построенных выражений, т.е. тех выражений, которые можно построить, используя функции АТД, применяемые к аргументам соответствующих типов.

Спецификация АТД T является \emph{достаточно полной} тогда и только тогда, когда аксиомы ее теории позволяют для каждого выражения expr \textbf{без свободных переменных типа T} решить следующие задачи:

\begin{itemize}
\item[(S1)] Определить, является ли expr корректным (синтаксически и не нарушено не одно предусловие);
\item[(S2)] Если expr обсерверного вида и в пункте S1 установлена его корректность, то представить значение expr в виде, не включающем никаких значений типа T.
\end{itemize}

\head{Метод построения достаточно полных алгебраических спецификаций}

\begin{enumerate}
\item составить список требований на компонент (<<features>>);
\item выделить целевой и нецелевые типы, операции-генераторы, операции-обсерверы: операций должно быть достаточно для спецификации всех требований;
\item составить все аксиомы вида <<обсервер(генератор)>> для всех корректных пар обсерверов и генераторов, не забыв про предусловия;
\item проверить, все ли требования специфицированы; если не все, ввести дополнительные операции-обсерверы или аксиомы над цепочками генераторов и повторить предыдущий шаг.
\end{enumerate}

Написание аксиомы для терма <<обсервер(генератор)>> --- это по сути задание вопроса о поведении специфицируемого компонента (<<как он себя поведет, если сделать сначала это, а потом спросить про это? всегда ли он поведет себя одинаковым образом?>>) и формализация ответа на этот вопрос.

Для задания того, что ни одни данные, которые передаются генераторам, не теряются (т.е. попадают в состояние), удобно использовать тотальный обсервер типа индикатора. Такой обсервер по состоянию и значению данных возвращает истину, если данные присутствуют в состоянии, и ложь в противном случае.

Для задания отношений компонент внутри состояния <<1-к-1>> удобно использовать пару тотальных обсерверов. Первый по состоянию и первой компоненте вычисляет вторую компоненту, а второй обсервер, наоборот, по состоянию и второй компоненте вычисляет первую компоненту.

Для задания упорядоченности компонент внутри состояния удобно использовать обсервер типа итератора. Он по очередной компоненте возвращает следующую компоненту.

Следует помнить следующий <<психологический эффект>>: мы хотим в спецификации выразить одно, а выражаем лишь его часть (это уже не <<однозначная>> спецификация, а <<недоопределенная>>). Выделив обсерверы (т.е. по сути атрибуты состояния), и описав аксиомы, надо проверить, точно ли описаны требования на операцию-обсервер: не описан ли \emph{более общий} или \emph{более частный} случай, нежели требовалось (например, вместо для хранилища элементов без порядка не выражено отсутствие порядка). И не забывать, что термы в спецификациях --- это лишь символы: если автор спецификации предполагает, что эта-та функция добавляет нечто куда-то, эта проверяет вхождение, то, если должным образом не сузить различные варианты совпадений значений различных термов из генераторов, то спецификация окажется неправильной, хотя и составленной автором с полной уверенностью в своей правоте.

\head{Конструкторы}

Зачастую полезно функцию, которая задает <<структуру>> состояния. Например, список --- это всегда цепочка неких элементов. Эту цепочку можно составлять по порядку по одному элементу. Дерево --- это неким образом организованные вершины (а в вершинах полезная информация). Тогда дерево можно составлять, добавляя по одной вершине. Операции по такой <<инициализации>> целевого типа будем называть \emph{конструкторами}.

Цепочка вызовов конструкторов дает значение в целевом типе. Причем конструктором может быть только та операция, термами из которых можно получить все значения в целевом типе.

Аналогами конструкторов в языке Java и некоторых других языках программирования могут быть конструкторы объектов (\texttt{tree = new Tree(n, left, right)}).

Замечено, что если удается выделить конструктор вместо генераторов, то спецификации получаются более качественными. А именно, предлагается определять обсерверы и генераторы через конструкторы, т.е. вместо аксиом вида <<обсервер(генератор)>> нужно написать все аксиомы вида <<обсервер(конструктор)>> и <<генератор(конструктор)>>.

Если конструкторы являются тотальными функциями, удобно объединить их вместе с целевым типом в виде следующего \emph{вариантного определения}, при этом писать отдельно сигнатуры конструкторов не нужно:
\begin{lstlisting}
type E, T == empty | cons(E, T)
\end{lstlisting}

При этом удобно определить сразу же ряд обсерверов для получения данных, с которыми было сконструировано значение целевого типа (в этом примере определяется обсервер elem, возвращающий E по целевому типу T, если T был создан при помощи cons):
\begin{lstlisting}
type E, T == empty | cons(elem:E, T)
\end{lstlisting}

Будьте внимательны! В некоторых изданиях <<генераторами>> называют то, что здесь называется конструкторами, а <<трансформерами>> --- то, что здесь названо генераторами.



\section*{Задачи}

% !Mode:: "TeX:UTF-8"

\zhead{Выбор генераторов}

\z Для стека, определенного таким образом:
\begin{lstlisting}
type E, S
value empty : S,
  push: S >< E -> S,
  pop: S >< E -~-> S
\end{lstlisting}
дать достаточно полную алгебраическую спецификацию операции проверки наличия заданного элемента
\begin{lstlisting}
value include: S >< E -> Bool
\end{lstlisting}

\textbf{Решение:}
\begin{lstlisting}
type E, S
value empty : S,
  push: S >< E -> S,
  pop: S >< E -~-> S,
  include: S >< E -> Bool
axiom forall e,e1:E, s:S :-
  ~ include(e, empty),
  include(e, push(s,e1)) is e = e1 \/ include(e,s),
  include(e, pop(s)) is
     if e = top(s) then count(e,s) > 1
             else count(e,s) > 0 end  pre s ~= empty
value count: E >< S -> Nat,
      top: S -~-> E
axiom forall e,e1:E, s:S :-
   count(e, empty) is 0,
   count(e, push(s,e1)) is count(e,s) +
      if e = e1 then 1 else 0 end,
   count(e, pop(s)) is count(e,s) -
      if e = top(s) then 1 else 0 end pre s ~= empty,

   top(push(s,e1)) is e1,
   top(pop(s)) is last(s,2) pre size(s) >= 2

value last: S >< Nat -~-> E,
      size: S -> Nat
axiom forall s:S, e:E, n:Nat :-
   size(empty) is 0,
   size(push(s,e)) is size(s) + 1,
   size(pop(s)) is size(s) - 1 pre s ~= empty,

   last(push(s,e), n) is
      if n = 1 then e else last(s, n-1) end
        pre n > 0 /\ n <= size(s),
   last(pop(s), n) is last(s, n+1)
        pre s ~= empty /\ n > 0 /\ n < size(s)
\end{lstlisting}

Обратите внимание, что
\begin{enumerate}
\item при помощи выбранных для описания стека генераторов значение стека будет иметь вид, например, push(push(pop(push(empty,1)),10),1); такое выражения значения в типе стек может казаться наиболее адекватным, ведь добавление и удаление элементов --- именно те операции, при помощи которых можно изменить значение (<<состояние>>) стека; однако посмотрите, насколько увеличивается спецификация и теряется ее наглядность, если в число генераторов включена лишняя функция (pop); сравните:
\begin{lstlisting}
type E, S
value empty : S,
  push: S >< E -> S
value include: S >< E -> Bool
axiom forall e,e1:E, s:S :-
  ~ include(e, empty),
  include(e, push(s,e1)) is e = e1 \/ include(e,s),
\end{lstlisting}

\item в этом примере синтаксически разные термы из генераторов могут означать одинаковые значения типа, например, push(pop(push(empty,1)),2) и push(pop(push(empty,3)),2); в таких случаях надо быть внимательными при выписывании аксиом: помнить и понимать, сколько разных возможностей есть для рекурсивной части цепочки (речь идет о переменной <<s>> в примерах) --- например, в аксиоме с генератором, удаляющим элемент, надо в том числе предполагать, что этот элемент может появиться много раз до этого, добавляться и удаляться.
\end{enumerate}

%% минимальный набор генераторов уменьшает спецификацию (оценить количество аксиом?)

%% при нескольких генераторах надо быть аккуратными

\z Для множества, определенного таким образом:
\begin{lstlisting}
type E, S
value empty : S,
  add: S >< E -> S,
  delete: S >< E -> S
\end{lstlisting}
дать алгебраическую спецификацию операции проверки наличия заданного элемента
\begin{lstlisting}
value include: S >< E -> Bool
\end{lstlisting}

\zhead{Описание эффекта  на основе структуры терма}

Вы уже заметили, что основной принцип написания аксиомы --- понять, как вычисляется обсервер после последнего сработавшего генератора. При этом для выражения этой аксиомы используются аргументы, с которыми вызван обсервер и последний генератор. Однако не всегда просто описать эффект работы генератора на основе лишь аргументов последнего из них.

\z Дать достаточно полную алгебраическую спецификацию типа <<Ограниченная очередь>>. В эту очередь можно добавлять и удалять элементы, но только если количество хранящихся элементов не превышает заданную величину.

\textbf{Решение:}
\begin{lstlisting}
value capacity : Nat
type E, S == empty | add(E,S)
value delete: S -~-> E
axiom forall e:E, s:S :-
   delete(add(e,s)) is s
     pre size(s) < capacity

value size: S -> Nat
axiom forall e:E, s:S :-
   size(empty) is 0,
   size(add(e,s)) is size(s) + 1
       pre size(s) < capacity
\end{lstlisting}

Обратите внимание, что
\begin{enumerate}
\item пришлось добавить и описать дополнительный обсервер size;
\item существует множество способов реализации ограниченной очереди при помощи имеющихся в языках программирования средств (например, при помощи <<кольцевой очереди>>, реализованной при помощи массива и двух указателей), однако здесь предъявляется именно формализация функциональности операций, которая остается справедливой и неизменной для любой реализации <<ограниченной очереди>>;
\item данное описание не дает определения того, в каких случаях определена каждая операция в отдельности;
\item при написании аксиомы для рекурсивной части терма достаточно представлять только \emph{правильно построенные термы} --- такие термы, в которых все функции вызваны с аргументами правильных типов и в каждой функции выполнено ее предусловие; например, для аксиомы size(add(e,s)) не надо представлять, что она должна описывать и такой терм: size(add(e,add(e1,add(e2,empty)))) при capacity = 2.
\end{enumerate}

\z Дать достаточно полную алгебраическую спецификацию типа <<Исключающая очередь>>. В эту очередь элемент добавляется в том случае, если его не было, а если он там был, то он удаляется из очереди. Опишите операцию проверки наличия заданного элемента в такой очереди.


%% не всегда просто понять, полна ли спецификация
\zhead{Противоречивость алгебраических спецификаций}

Если спецификация допускает подстановку и <<вычисление>> термов увеличивающейся длины, пытаться <<вычислять>> эти термы разными способами, которые допускают аксиомы, и проверять, получаются ли одинаковые результаты в разных способах. Если получились разные, значит, найдено противоречие.

Во всех задачах следует предполагать, что в нецелевых типах данных достаточно много различных значений.

\z Противоречива ли следующая спецификация?
\begin{lstlisting}
type E, T
value empty : T,
      add: T >< E -> T,
      check: T >< E -> Bool
axiom forall e,e1: E, t: T :-
   add(add(t,e),e) is add(t,e),
   add(add(t,e),e1) is add(add(t,e1),e),
   ~check(empty, e),
   check( add(t,e1), e) is e = e1 \/ check(t,e),
\end{lstlisting}

\textbf{Решение:}
Она непротиворечива. В доказательство достаточно привести модель. Например, такую:
\begin{lstlisting}
type E, T = E-set
value empty : T = {},
      add: T >< E -> T add(t,e) is t union {e},
      check: T >< E -> Bool check(t,e) is e isin t
\end{lstlisting}

\z Противоречива ли следующая спецификация?
\begin{lstlisting}
type E, T
value empty : T,
      add: T >< E -> T,
      choose: T -~-> E
axiom forall e,e1: E, t: T :-
   add(add(t,e),e) is add(t,e),
   add(add(t,e),e1) is add(add(t,e1),e),
   choose(add(t,e)) is e
\end{lstlisting}

\textbf{Решение:}
Противоречивая. Докажем это. Поскольку верна вторая аксиома, то верно, что choose(add(add(t,e1),e2)) is choose(add(add(t,e2),e1)) для e1 $\Not$= e2. По третьей аксиоме левая часть эквивалентна e2, а правая e1, т.е. e2 is e1 при e1 $\Not$= e2, получено противоречие.

Важным моментом здесь является то, что хотя choose может быть недетерминированной, но она эквивалентным образом должна себя вести на эквивалентных аргументах (для сравнения на равенство этот факт был бы ложным).

Сформулируйте противоречие данной системы аксиом <<на содержательном>> уровне.

\z Противоречива ли следующая спецификация?
\begin{lstlisting}
type Integer = Int
value MAXINT : Integer :- MAXINT > 0
axiom all i: Integer :- abs i < MAXINT
value add: Integer >< Integer -~-> Integer
axiom forall x, y: Integer :-
   add(x, 0) is x,
   add(x, y+1) is add(x,y)+1
\end{lstlisting}

\textbf{Решение:}
Из сигнатуры add следует, что $\All$ x, y : $\Int$ ~$\SuchAs$~ \textbf{abs} x < MAXINT $\wedge$ \textbf{abs} y < MAXINT $\wedge$ (add определена для x и y => abs add(x,y) < MAXINT). Из первой аксиомы следует, что add(MAXINT-1, 0) определен и равен MAXINT-1. Тогда по второй аксиоме add(MAXINT-1,1) определен и равен MAXINT --- получили противоречие с тем, что аргументы функции допустимы и функция для них определена, но ее значение выходит за рамки типа. Ответ: противоречива.

\z Противоречива ли следующая спецификация?
\begin{lstlisting}
type T
value empty : T,
    put: T >< Nat -> T,
    get: T >< Nat -> Bool
axiom forall t: T, x, y, z: Nat :-
  put( put(t, x), x ) is put(t, x),
  ~get( empty, x ),
  get( put(empty, x), y ) is (x = y /\ x\2 = 0),
  get(put(put(empty,x),y),z) is (z=x /\ y\2=0) pre x\2=0,
  get( put(t, y), x ) is get(t, x) pre y ~= x,
  get( put( put(t, x), y), x ) is get( put(t,x), x ),
  ~get(put(put(t,x),x+1),x+1) pre ~get(t,x+1) /\ get(t,x)
\end{lstlisting}

\textit{Подсказка:} для определения противоречивости системы аксиом можно применить следующие соображения:
\begin{itemize}
  \item добавить следствие аксиомы путем подстановки в ее различные переменные одинаковых переменных (если это не противоречит предусловию аксиомы);
  \item добавить следствие аксиомы путем подстановки в ее переменные некоторых значений (если это не противоречит предусловию аксиомы), например, вместо целевого типа подставить empty;
  \item попробовать <<свернуть рекурсию>>;
  \item если получилось две аксиомы с одинаковой левой частью и непустым пересечением предусловий, добавить следствие этих двух аксиом в виде равенства правых частей с предусловием, равным пересечению предусловий исходных аксиом.
\end{itemize}

Если после некоторого количества таких <<манипуляций>> пришли к противоречию (например, 1 = 0 при непустом предусловии), то система аксиом противоречива.

\z Противоречива ли следующая спецификация?
\begin{lstlisting}
type T
value empty: T,
    put: T >< Nat -> T,
    get: T >< Nat -> Bool
axiom forall t:T, x, y, z: Nat :-
  put(put(t, x), x) is put(t, x),
  get( put (put(t, 0), x ), x) is (x > 0),
  get( put (put(t, 2*x), x ), x) is (x > 0),
  ~get( empty, x ),
  ~get( put(empty,x), y ),
  get(put(put(empty,x),y),z) is (z>0 /\ z=abs(x-y) )
\end{lstlisting}

\z Противоречива ли следующая спецификация?
\begin{lstlisting}
type T = Nat
value empty: T,
    put: T >< Nat -> T,
    get: T >< Nat -> Bool
axiom forall t: T, x, y, z: Nat :-
  put( put(t, x), y ) is put(t,x) pre y <= x,
  ~get( empty, x ),
  get( put(empty, x), y ) is (x = y),
  get(put(put(empty,x),y),z) is (z = x \/ z = y /\ y > x),
  get( put(t,x), y ) is get(t,y) pre y ~= x,
  ~get(put(t,2*y),2*y) pre get(t,y) /\
       get(t,y+1) /\~get(t,2*y),
  get( put(put(t, 0), x), x ) is get( put(t,x), x )
\end{lstlisting}

\z Противоречива ли следующая спецификация?
\begin{lstlisting}
type T
value empty: T,
    put: T >< Nat -> T,
    get: T >< Nat -> Bool
axiom forall t: T, x, y, z: Nat :-
  put( put(t, x), y ) is put(t,x) pre y <= x,
  ~get( empty, x ),
  get( put(empty, x), y ) is (x = y),
  get(put(put(empty,x),y),z) is (z = x \/ z = y /\ y > x),
  get( put(t,x), y ) is get(t,y) pre y ~= x,
  ~get( put(t, 2*y), 2*y ) pre get(t,y) /\ get(t, y+1),
  get( put(put(t, 0), x), x ) is get( put(t,x), x )
\end{lstlisting}

\z Противоречива ли следующая спецификация?
\begin{lstlisting}
type T
value empty: T,
    put: T >< Nat -> T,
    get: T >< Nat -> Bool
axiom forall t: T, x, y, z: Nat :-
  put( put(t,x), x ) is put(t,x),
  ~get( empty, x ),
  get( put(empty, x), y ) is (x = y),
  get(put(put(empty,x),y),z) is (z=x \/ z=y /\ y>2),
  get( put(t,x), y ) is get(t, y) pre y ~= x,
  get( put(t,y), y ) pre get( put(t,x), x ) /\ x <= y,
  ~get(t,x) pre get( put(t, 0), 0),
  ~get(t, x) /\ ~get(t,y) pre x ~= y /\ get(put(t,1),1)
\end{lstlisting}

\z Противоречива ли следующая спецификация?
\begin{lstlisting}
type T
value empty: T,
    put: T >< Nat -> T,
    get: T >< Nat -> Bool
axiom forall t: T, x, y, z: Nat :-
  put( put(t,x), x ) is put(t,x),
  ~get( empty, x ),
  get( put(empty, x), y ) is (x = y),
  get( put( put(empty,x), y ), z ) is (z = x \/ z = y /\ y > 0),
  get( put(t,x), y ) is get(t, y) pre y ~= x,
  get( put(t,y), y ) pre get( put(t,x), x ) /\ x <= y,
  ~get(t,x) pre get( put(t, 0), 0) /\ ~get(t,0),
  ~get(t,x) \/ ~get(t,y) pre x~=y /\ get(put(t,1),1) /\ ~get(t,1)
\end{lstlisting}

\zhead{Полнота алгебраических спецификаций}

Не забывайте, что у нас есть только система аксиом и логика, т.е. правила получения новых выражений из имеющихся. За символами имен операций не стоит никакой семантики, даже если она <<предполагалась>> автором. Наоборот, эта система аксиом должна \emph{дать} нам семантику символов, т.е. дать нам возможность сделать с этими символами некие осмысленные действия.

\z Полно ли описывает следующая спецификация тип <<Множество>> ?
\begin{lstlisting}
type E, S
value empty: S,
      add: E >< S -> S,
axiom forall e1, e2: E, s : S :-
   add(e1, add(e1, s)) is add(e1, s),
   add(e1, add(e2, s)) is add(e2, add(e1, s))
\end{lstlisting}

\textbf{Решение:}
В этой спецификации нет ни одного обсервера, поэтому вопрос о полноте для нее некорректен.

\z Полна ли следующая спецификация типа <<Очередь>> ?
\begin{lstlisting}
type E, Q
value empty: Q,
      add: E >< Q -> Q,
      size: Q -> Nat
axiom forall e: E, q : Q :-
    size(empty) is 0,
    size(add(q, e)) is size(q) + 1
\end{lstlisting}

\textbf{Решение:}
Обсервер --- size. Он определен для empty и определен для генератора add. Значит, он определен для любого терма, дающего тип Q. Значит, функция size описана полно.

\z Полна ли следующая спецификация типа <<Очередь>> ?
\begin{lstlisting}
type E, Q
value empty: Q,
      add: Q >< E -> Q,
      size: Q -> Nat
axiom forall e: E, q : Q :-
    size(empty) is 0,
    size(add(q, e)) is size(q) + 1,
    add(add(q, e), e) is add(q, e)
\end{lstlisting}

\textbf{Решение:}
Без учета последней аксиомы size(add(add(q,e),e)) is size(add(q,e)) + 1 is size(q) + 2. А теперь с последней аксиомой: size(add(add(q,e),e)) is size(add(q,e)) is size(q) + 1. Получается, что size(q) + 1 = size(q) + 2. Иными словами, из аксиом следует ложь, система аксиом противоречива. А раз так, вопрос о полноте некорректен.


\z Полна ли следующая спецификация типа <<Очередь>> ?
\begin{lstlisting}
type E, Q
value empty: Q,
      add: Q >< E -> Q,
      first: Q -~-> E,
      size: Q -> Nat
axiom forall e: E, q : Q :-
    first(add(empty, e)) is e,
    first(add(q, e)) is first(q) pre q ~= empty,
    size(empty) is 0,
    size(add(q,e)) is size(q) + 1
\end{lstlisting}

\textbf{Решение:}
Любое значение в целевом типе Q имеет один из двух видов (просто напросто, нет других функций, возвращающих Q):
\begin{itemize}
  \item empty
  \item add(add(add(...add(add(empty, e1), e2)...)
\end{itemize}

Посмотрим на first с точки зрения определения достаточной полноты. Обе аксиомы для простоты можно объединить в одну: first(add(q,e)) is if q = empty then e else first(q) end. Тогда такое выражение <<вычислить>> можно: first(add(empty,e1)) is e1 (по первой аксиоме). Посмотрим такое выражение: first(add(add(empty,e1),e2)) is if add(empty,e1) = empty then e2 else e1 end. Чтобы закончить <<вычисление>>, надо понять истинность выражения add(empty,e1) = empty. В общем случае дать ответ на этот вопрос нельзя (\emph{проблема равенства термов алгоритмически неразрешима}). Однако в данном случае ответить на этот вопрос можно.

Для этого сделаем такой хитрый ход -- \emph{<<навесим>> на эти два выражения сверху другой обсервер}: size(empty) is 0, size(add(empty,e1)) is size(empty) + 1 is 0 + 1 is 1, т.е. size(empty) $\Not$= size(add(empty,e1)). Поскольку size --- тотальная функция, то она детерминированная, т.е. all x, y : Q :- x = y => size(x) = size(y), что то же самое, что size(x) $\Not$= size(y) => x $\Not$= y. Теперь в качестве x возьмем empty, а в качестве y возьмем add(empty, e1). Получим, что add(empty, e1) $\Not$= empty. Ура, желаемое доказано! Тем самым, можно и вычислить второй терм, он равен e1. Аналогично, можно вычислить и все остальные термы.

Осталось рассмотреть единственный терм: first(empty). Если бы существовали такие q' и e', что add(q',e') равнялось empty, то было бы возможно применение аксиом. Но, как следует из первой части, такие q' и e' не существуют, значит, <<вычислить>> first(empty) на основе данных аксиом нельзя. Ответ: неполна.

\z Полна ли следующая спецификация типа <<Очередь>> ?
\begin{lstlisting}
type E, Q
value empty: Q,
      add: Q >< E -> Q,
      first: Q -~-> E,
axiom forall e: E, q : Q :-
    first(add(empty, e)) is e,
    first(add(q, e)) is first(q) pre q ~= empty,
\end{lstlisting}

\textbf{Решение:}
Рассуждая аналогично предыдущей задаче, приходим к вопросу об истинности add(empty,e1) = empty и в данной системе аксиом дать ответ на этот вопрос нельзя. Ответ: неполна.






%попробовать описать такие функции для множества как длина, вложение, вхождение элемента во множество


\section{Cпецификация рекурсивных типов}

%\head{Рекурсивное определение типов (деревьев)}
Алгебраические модели и АТД являются удобным способом рекурсивного задания типов и операций их обработки. Для этого представленный только что формальный метод построения достаточно полных алгебраических спецификаций прочитывается следующим способом:
\begin{enumerate}
  \item выделить и составить сигнатуры конструкторов (пустое значение | составление нового <<узла дерева>>);
  \item выделить и составить сигнатуры функций-обработчиков рекурсивного типа;
  \item описать результат обработки для каждого конструктора.
\end{enumerate}

Например, спецификация функции, вычисляющей глубину бинарного дерева, согласно этому методу получается такой:
\begin{lstlisting}
type Node, Tree == empty | mk_tree(Node, Tree, Tree)
value depth: Tree -> Nat
axiom forall n: Node, left: Tree, right: Tree :-
  depth( empty ) is 0,
  depth( mk_tree(n, left, right) ) is
          max( depth(left), depth(right) )
value max: Nat >< Nat -> Nat
   max(x,y) is if x > y then x else y end
\end{lstlisting}

Конструктор предназначена лишь для структурных целей, т.е. определение дерева в виде типа Tree из этого примера пойдет и для бинарного дерева произвольного вида, и для бинарного сбалансированного дерева, и для других бинарных деревьев. А уже менее тривиальные функции составления деревьев (например, чтобы не нарушалась сбалансированность) определяются в виде функций-обработчиков.

\section*{Задачи}
%
\zhead{}%Рекурсивные типы}
%
\z Специфицируйте операцию проверки вхождения элемента в бинарное дерево.

\textbf{Решение:}
\begin{lstlisting}
type Node, Tree == empty | add(Node, Tree, Tree)
value check: Node >< Tree -> Bool
axiom forall n, n1:Node, left, right: Tree :-
   ~check( empty ),
   check(n, add(n1,left,right) ) is (n = n1) \/
             check(n, left) \/ check(n, right)
\end{lstlisting}

\z Специфицируйте операцию вычисления высоты бинарного дерева.

\z Специфицируйте операцию проверки бинарного дерева на сбалансированность.

\z Специфицируйте операцию получения предка элемента бинарного дерева.

\z Специфицируйте добавление элемента в двоичное дерево поиска.

\z Специфицируйте удаление элемента из двоичного дерева поиска.

\z Специфицируйте добавление элемента в АВЛ-дерево. Определение операции предполагается найти самостоятельно.

\z Специфицируйте удаление элемента из АВЛ-дерева. Определение операции предполагается найти самостоятельно.

\z Специфицируйте добавление элемента в 2-3-дерево. Определение операции предполагается найти самостоятельно.

\z Специфицируйте удаление элемента из 2-3-дерева. Определение операции предполагается найти самостоятельно.

\z Специфицируйте добавление элемента в декартово дерево. Определение операции предполагается найти самостоятельно.

\z Специфицируйте удаление элемента из декартова дерева. Определение операции предполагается найти самостоятельно.

\z Специфицируйте добавление элемента в красно-чёрное дерево.  Определение операции предполагается найти самостоятельно.

\z Специфицируйте удаление элемента из красно-чёрного дерева.  Определение операции предполагается найти самостоятельно.

% специфицировать стратегии вытеснения



\chapter{Моделе-ориентированные спецификации}

\section{Множества, списки, отображения}

\head{Множества}
Множество -- это контейнер элементов одного типа, который обладает свойствами уникальности и неупорядоченности его элементов. Множества бывают конечными и бесконечными.

Типовое выражение для конечного множества: $typeexpr\Set$. Типовое выражение для бесконечного множества: $typeexpr\Infset$. Конечное множество -- подтип бесконечного множества.

Операции над множествами:
\begin{list}{}{}
\item $= : T\Infset \DP T\Infset \Fn \Bool$ -- сравнение на равенство
\item $\neq : T\Infset \DP T\Infset \Fn \Bool$ -- сравнение на не равенство
\item $\Union: T\Infset \DP T\Infset \Fn T\Infset$ -- объединение множеств
\item $\Inter: T\Infset \DP T\Infset \Fn T\Infset$ -- пересечение множеств
\item $\Minus: T\Infset \DP T\Infset \Fn T\Infset$ -- вычитание множеств
\item $\Isin: T \DP T\Infset \Fn \Bool$ -- проверка на принадлежность
\item $\NotIsin: T \DP T\Infset \Fn \Bool$ -- проверка на непринадлежность
\item $\SetL: T\Infset \DP T\Infset \Fn \Bool$ -- проверка вложения
\item $\SetG: T\Infset \DP T\Infset \Fn \Bool$ -- проверка вложения
\item $\SetLE: T\Infset \DP T\Infset \Fn \Bool$ -- проверка вложения
\item $\SetGE: T\Infset \DP T\Infset \Fn \Bool$ -- проверка вложения
\item $\Card: T\Infset \NonDetermFn \Nat$ -- количество элементов
($\Chaos$ для бесконечных множеств)
\end{list}


Конструкторы множеств:
\begin{list}{}{}
\item (\emph{пустое множество}) \{\}
\item (\emph{перечисление}) \{0, 1, 2\} - множество, состоящее из трех целых чисел (не
натуральных!) - нуля, единицы и двойки.
\item (\emph{диапазон}) \{0..2\} -  то же, что и \{0, 1, 2\};
\{0..0\} $\Iden$ \{0\}, \{1..0\} $\Iden$ \{\}
\item (\emph{<<сокращенная запись>>}) $\{ expr\_with\_var | var : typeexpr
\SuchAs boolexpr\}$ (например, $\{2 \star n | n : \Nat \SuchAs n < 3
\}$, что эквивалентно \{0, 2, 4\})
\end{list}

\head{Списки}
Список -- это контейнер элементов одного типа, который обладает свойствами упорядоченности элементов. Для задания списка надо указать не только сами элементы, но и их порядок. Списки бывают конечными и бесконечными.

Типовое выражение для конечного списка: $typeexpr\List$. Типовое выражение для бесконечного списка: $typeexpr\Inflist$. Конечный список -- подтип бесконечного списка.

Операции:
\begin{list}{}{}
\item $=: T\Inflist \DP T\Inflist \Fn \Bool$ -- проверка на равенство
\item $\neq: T\Inflist \DP T\Inflist \Fn \Bool$ -- проверка на не равенство
\item $(.) : T\Inflist \DP \Int \NonDetermFn T$ -- взятие элемента по
индексу ($\Chaos$ для индекса, отсутствующего в списке, индексы
нумеруются \textbf{с единицы})
\item $\Concat: T\List \DP T\Inflist \Fn T\Inflist$ -- конкатенация списков
\item $\Hd: T\Inflist \NonDetermFn T$ -- головной элемент списка ($\Chaos$ для пустого списка)
\item $\Tl: T\Inflist \NonDetermFn T\Inflist$ -- хвостовая часть списка ($\Chaos$ для пустого списка)
\item $\Len: T\Inflist \NonDetermFn \Nat$ -- количество элементов списка ($\Chaos$ для бесконечного списка)
\item $\Elems: T\Inflist \Fn T\Infset$ -- множество элементов списка (без повторений!)
\item $\Inds: T\Inflist \Fn \Nat\Infset$ -- множество индексов элементов списка
\end{list}

Конструкторы списков:
\begin{list}{}{}
\item (\emph{пустой список}) $\LL\LR$
\item (\emph{перечисление}) $\LL 0, 1, 2 \LR$ - список, состоящий из трех целых чисел (не натуральных!) - нуля, единицы и двойки - в порядке увеличения.
\item (\emph{диапазон}) $\LL 0..2 \LR$ -  то же, что и $\LL 0, 1, 2 \LR$; $\LL 0..0 \LR \Iden \LL 0 \LR$, $\LL 1..0 \LR \Iden \LL\LR$
\item (\emph{<<сокращенная запись>>}) $\LL expr\_with\_var | var \In listexpr \SuchAs boolexpr\LR$ (например, $\LL 2 \star n | n \In \LL 0..2 \LR \LR$, что эквивалентно $\LL 0, 2, 4 \LR$)
\end{list}


\head{Отображения}
Отображение -- это множество пар элементов, у которого первые компоненты не повторяются. Отображения бывают конечными и бесконечными, детерминированными и недетерминированными (это те, в которых первые компоненты всё же могут повторяться в разных парах).

Типовое выражение для детерминированного отображения: $$typeexpr_1 \Map typeexpr_2$$ Типовое выражение для недетерминированного отображения: $$typeexpr_1 \NonDeterMap typeexpr_2$$

Операции:
\begin{list}{}{}
\item $=: (T_1 \Map T_2) \DP (T_1 \Map T_2) \Fn \Bool$ -- сравнение отображений на равенство
\item $\neq: (T_1 \Map T_2) \DP (T_1 \Map T_2) \Fn \Bool$ -- сравнение отображений на не равенство
\item $(.): (T_1 \Map T_2) \DP T_1 \NonDetermFn T_2$ -- взятие значения по индексу ($\Chaos$, если значение по этому индексу не
определено)
\item $\Dom: (T_1 \Map T_2) \Fn T_1\Infset$ -- \emph{область определения} отображения
\item $\Rng: (T_1 \Map T_2) \Fn T_2\Infset$ -- \emph{область значения} отображения
\item $\Upd: (T_1 \Map T_2) \DP (T_1 \Map T_2) \Fn (T_1 \Map T_2)$ -- обновление отображения
\item $\Union: (T_1 \Map T_2) \DP (T_1 \Map T_2) \Fn (T_1 \NonDeterMap T_2)$ -- объединение отображений
\item $\Minus: (T_1 \Map T_2) \DP T_1\Infset \Fn (T_1 \Map T_2)$ -- уменьшение отображения
\item $/: (T_1 \Map T_2) \DP T_1\Infset \Fn (T_1 \Map T_2)$ -- проекция отображения
\item $\Superp: (T_2 \Map T_3) \DP (T_1 \Map T_2) \Fn (T_1 \Map T_2)$ -- композиция отображений
\end{list}

Конструкторы отображений:
\begin{list}{}{}
\item (\emph{пустое отображение}) []
\item (\emph{перечисление}) $[0 \mapsto 1, 1 \mapsto 2]$ -- отображение, состоящий из двух пар целых чисел (не натуральных!) - из 0 в 1 и из 1 в 2.
\item (\emph{<<сокращенная запись>>}) $[expr_1\_with\_var \mapsto expr_2\_with\_var | var : typeexpr \SuchAs boolexpr]$ (например, $[n \mapsto n+1 | n : \Nat \SuchAs n < 3]$, что эквивалентно $[0 \mapsto 1, 1 \mapsto 2, 2 \mapsto 3]$)
\end{list}


\section*{Задачи}

\zhead{Вычислить выражения с множествами}

\z $\{1, 2\} = \{3, 1\}$
\z $\{1, 2\} = \{2, 1\}$
\z $\{1, 2, 1\} = \{2, 2, 1\}$
\z $\{1, 2\} \Union \{3, 4\}$
\z $\{1, 2\} \Union \{2, 3\}$
\z $\{1, 2\} \Inter \{3, 4\}$
\z $\{1, 2\} \Inter \{2, 3\}$
\z $\{1..30\} \Union \{10..-10\}$
\z $\{1..30\} \Inter \{10..-10\}$
\z $\{1..30\} \Union \{x~|~ x: \Int \SuchAs \Abs x < 11 \}$
\z $\{1..30\} \Inter \{x~|~ x: \Int \SuchAs \Abs x < 11 \}$
\z $\{x+10 ~|~ x: \Int \} = \{x ~|~ x: \Int\}$
\z $\{5*k + 2 ~|~ k : \Int \} \Inter \{3*k - 1 ~|~ k : \Int\}$
\z $\{5*k + 2 ~|~ k : \Int \} \Minus \{3*k - 1 ~|~ k : \Int\}$
\z $\{5*k + 2 ~|~ k : \Int \} \SetL \{3*k - 1 ~|~ k : \Int\}$
\z $\{5*k + 2 ~|~ k : \Int \} \SetGE \{3*k - 1 ~|~ k : \Int\}$
\z $\Not (\{1, 2\} \SetLE \{2, 1, 1\})$
\z $\Card \{1..30\}$
\z $\Card \{5*k + 2 ~|~ k : \Int \SuchAs k * k \Isin \{-10..10\}\}$
\z $\Card \{5*k + 2 ~|~ k : \Int \SuchAs k * k \Isin \{10..-10\}\}$
\z $\Card \{k*k - 2 * k ~|~ k : \Int \SuchAs k * k \Isin \{-10..10\}\}$
\z $\All x: \Nat ~\SuchAs~ x \Isin \{x ~|~ x : \Int\}$
\z $\All x: \Int ~\SuchAs~ x \Isin \{x ~|~ x : \Nat\}$


\zhead{Решите уравнения}

\z $\{1\} \Union x = \{\}$
\z $\{1\} \Union x = \{1\}$
\z $\{1\} \Union x = \{1, 2\}$
\z $\{1\} \Inter x = \{\}$
\z $\{1\} \Inter x = \{1\}$
\z $\{1\} \Inter x = \{1, 2\}$
\z $\{2, 1\} \Minus x = \{1\}$
\z $\Card~x = 0$
\z $\Card~x = 1$


\zhead{Какие из следующих выражений истинные}
Считайте, что свободные переменные располагаются под квантором
всеобщности

\z $(A \cup B) \setminus C = (A \setminus C) \cup (B \setminus C)$
\z $(A \setminus B) \cap C = (A \cap C) \setminus B$
\z $(A \cup B) \setminus C \Iden (A \setminus C) \cup (B \setminus C)$
\z $A \cup \{\} = A$
\z $\{\} \cup A = A$
\z $(A \cup B) \cap C = (A \cap C) \cup (B \cap C)$
\z $\Card \Nat < \Card \Int$
\z $\Card \Nat = \Card \Int$
\z $\Card \Nat > \Card \Int$
\z $\Card \Nat \Iden \Card \Int$
\z $\Card \{n ~|~ n:\Nat\} \Iden \Card \{n ~|~ n:\Int\}$


\zhead{Записать на RSL следующие множества}

\z Пустое множество;
\z Множество чисел 1, 2, 3 (а также множество чисел от 1 до 3);
\z Множество всех чётных чисел;
\z Множество всех чётных чисел, не превышающих 10 (привести в виде перечисления и нескольких различных сокращённых формах);
\z Множество всех простых натуральных чисел;
\z Множество всех пар взаимнопростых натуральных чисел;
\z Множество всех троек, в каждой из которых есть одинаковые элементы;
\z Множество всех степеней двойки, не превышающих 100 (привести в виде перечисления и нескольких сокращённых формах);
\z Множество всех IP-адресов класса А (B, C, D, E);
\z Множество всех точек плоскости, образующих
    \begin{enumerate}
    \item Прямую -- биссектрису I и III квадрантов;
    \item Правую полуплоскость;
    \item Нижнюю полуплоскость;
    \item Единичную окружность;
    \item Единичный круг;
    \item Единичный квадрат со сторонами, параллельными осям
    координат.
    \end{enumerate}

\zhead{Построить явную спецификацию на RSL для следующих задач}

В этих задачах, если удаётся, следует привести два варианта решения: с использованием сокращённой записи множества
и с помощью рекурсии.

\z Дано множество чисел. Вернуть множество квадратов чисел данного множества.
\z Дано множество чисел. Вернуть множество чисел, представимых суммами каких-либо двух элементов исходного множества.
\z Дано множество точек, заданных координатами в плоской декартовой системе координат. Получить проекцию этого множества на ось \textsc{Ox}.
\z Дано множество всех простых чисел. Для данного числа вернуть множество его простых делителей, используя множество всех простых чисел.
\z Дано множество чисел. Дано число. Существует ли подмножество данного множества чисел, сумма элементов которого, равна данному числу.
\z Дано множество множеств. Вернуть множество всех элементов внутренних множеств.
\z \textit{(Задача о рюкзаке)} Дано множество предметов. Каждый предмет имеет массу (массы заданы множеством пар предмет >< масса). Есть $n$ рюкзаков. Каждый рюкзак имеет вместимость $K$ кг. Распределить все предметы по рюкзакам, не превышая вместимости каждого рюкзака.
\z Рабочая группа компьютеров задаётся IP-адресом (4 целых числа, каждое от 0 до 255) и маской (целое число от 0 до 32). Программа по данному заданию строит множество всех возможных в ней IP-адресов.


\zhead{Вычислить выражения со списками}
\z $\LL 1 \LR = \LL 1, 1 \LR$
\z $\LL 1, 2 \LR = \LL 3, 1\LR$
\z $\LL 1, 2\LR = \LL 2, 1\LR$
\z $\LL 1, 2, 3 \LR (2)$
\z $\LL 1 \LR (0)$
\z $\LL 1 \LR (1)$
\z $\LL 1, 2\LR \Concat \LL 3, 4\LR$
\z $\LL 1, 2\LR \Concat \LL 2, 3\LR$
\z $\LL x+10 | x \In \LL 1, 2 \LR \LR = \LL x | x \In \LL 1 \LR \LR$
\z $\Hd \LL 1, 2, 3 \LR$
\z $\Tl \LL 1, 2, 3 \LR$
\z $\Len \LL 1, 2, 3 \LR$
\z $\Elems \LL 1, 2, 3 \LR$
\z $\Inds \LL 1, 2, 3 \LR$
\z $\Len \LL 1..30\LR$
\z $\Let~x = \LL 1, 2, 3\LR~\In~\Elems~x~\Inter~\Inds~x~\rslEnd$
\z $\Let~x = \LL 0, 1, 2\LR~\In~\Elems~x~\Inter~\Inds~x~\rslEnd$
\z $\Let~x : \Int\Inflist~\SuchAs~(\All i_1, i_2: \Nat~\SuchAs~\Card \{i_1, i_2\} = \Card \{x(i_1), x(i_2)\})\\\In~x~\rslEnd$
\z $\Let~x : \Int\Inflist~\SuchAs~(\All i_1, i_2: \Nat~\SuchAs~\Card \{i_1, i_2\} = \Card \{x(i_1), x(i_2)\})\\\In~\Elems~x~\Inter~\Inds~x~\rslEnd$
\z $\All x : T\Inflist~\SuchAs~x(0) = \Hd x$
\z $\All x : T\Inflist~\SuchAs~x(0) \Iden \Hd x$

\zhead{Решите уравнения}
\z $\LL 1 \LR \Concat x  = \LL 1 \LR$
\z $\LL 1 \LR \Concat x  = \LL 1, 2, 3 \LR$
\z $\LL 1 \LR \Concat x  = \LL 3, 2, 1 \LR$
\z $\Tl x = \LL 1 \LR$
\z $\Hd x = 1$
\z $x \Concat \Tl x = x$
\z $\Tl x = \LL \Hd x \LR$
\z $\Elems x = \Inds x$
\z $\Elems\Tl x = \Elems x$
\z $\Len x = \Card \Elems x$
\z $\Len x = \Card \Inds x$
\z $\Len x \Iden \Card \Inds x$


\zhead{Записать на RSL следующие выражения}

\z Пустой список;
\z Список из чисел 1, 2, 3 (попробуйте привести как можно больше различных решений);
\z Список всех простых натуральных чисел в порядке увеличения их значения;
\z Список всех пар взаимнопростых натуральных чисел в порядке увеличения их суммы;
\z Список номеров групп 5го курса факультета ВМиК МГУ в порядке увеличения номера группы;
\z Количество простых чисел от 1 до 10 (тремя различными способами).

\zhead{Напишите явные спецификации следующих функций}

\z Определить, является ли бесконечным данный список.
\z Вычислить длину списка без использования функции $\Len$.
\z Вычислить сумму элементов списка.
\z Вычислить произведение элементов списка.
\z Дан список. Построить список из элементов исходного списка, элементы которого идут в обратном порядке по отношению к исходному списку
\z Дана строка и символ. Определить, встречается ли символ в данной строке (предложить два различных способа решения).
\z Дана строка. Определить самый часто встречающийся в ней символ.
\z Отсортировать данный список вещественных чисел в порядке возрастания.
\z Построить список всех целых чисел в порядке неубывания модуля.
\z Определить $\sup$ списка вещественных чисел (в случае конечного списка это будет и максимум).
\z Выдать число, не встречающееся в данном списке.
\z Дан список чисел. Построить по нему список квадратов, расположив элементы
    \begin{enumerate}
    \item с сохранением порядка исходного списка
    \item в порядке убывания модуля
    \item так, чтобы не было трёх подряд чисел, расположенных в  порядке возрастания или убывания.
    \end{enumerate}
\z Дано натуральное число. Построить список степеней его простых делителей, в котором на местоположение степени есть её основание (а значение, соответственно, показатель).
\z Дан список показателей степеней (местоположение -- основание степени). Построить соответствующее натуральное число.
\z Дан список натуральных чисел, отличных от нуля. Вернуть список, в котором на месте №$i$ находится количество раз, которое $i$ встречается в исходном списке.
\z Дано множество чисел. Построить из него список, расположив элементы множества
    \begin{enumerate}
    \item в порядке убывания
    \item в порядке убывания модуля
    \end{enumerate}
\z Дан список чисел. Построить из него новый список, расположив элементы исходного списка
    \begin{enumerate}
    \item в порядке убывания
    \item в порядке убывания модуля
    \end{enumerate}
\z Дан список из чисел. Построить из него множество, использовав все элементы данного списка.
\z Дан множество пар (ключ, объект) и список ключей. Построить соответствующий ему список объектов. Считайте, что во множество пар ключи не повторяются.
\z Дан список чисел. Можно ли суммой некоторых его элементов получить
    \begin{enumerate}
    \item четное число (для списка из целых чисел)
    \item простое число (для списка из целых чисел)
    \item целое число (для списка из вещественных чисел)
    \end{enumerate}


\zhead{Упростить выражения}
\z $\langle x ~|~ x~\In~\langle \Card \{a..b\} \rangle \rangle$
\z $\langle c ~|~ c~\In~\langle 'a',~'b' \rangle \rangle$
\z $\langle c ~|~ c~\In~\langle '$ " $~' \rangle \rangle$
\z $\langle c ~|~ c~\In~\langle \rangle \rangle$

\newcommand{\Masha}{\mbox{\textrm{<<Маша>>}}}
\newcommand{\Sveta}{\mbox{\textrm{<<Света>>}}}
\newcommand{\Misha}{\mbox{\textrm{<<Миша>>}}}
\newcommand{\Slava}{\mbox{\textrm{<<Слава>>}}}
\newcommand{\Anna}{\mbox{\textrm{<<Аня>>}}}
\newcommand{\Petr}{\mbox{\textrm{<<Петя>>}}}
\newcommand{\Lesha}{\mbox{\textrm{<<Леша>>}}}
\newcommand{\Victor}{\mbox{\textrm{<<Витя>>}}}

\zhead{Вычислить}

\z $\map{n \mapsto 1}{n : \Nat \SuchAs n \Isin \{1..3\} }$
\z $\map{n \mapsto n}{n : \Nat \SuchAs n \Isin \{5..5\} }$
\z $\map{n \mapsto n+1}{n : \Nat \SuchAs n \Isin \{100..90\} }$
\z $\map{n \mapsto m}{n, m : \Nat \SuchAs n \backslash m = 0 \And m > 2}$
\z $\map{n \mapsto m}{n, m : \Nat \SuchAs n \Isin \{1..3\} \And m \Isin \{1..n\}}$
\z $\map{n \mapsto (p, q)}{n, p, q : \Nat \SuchAs n \Isin \{1..100\} \And p + q = n \And p \backslash q = 0}$
\z $[1 \mapsto 2,~2 \mapsto 3,~ 3 \mapsto 1] (3)$
\z $[1 \mapsto 2,~2 \mapsto 3,~ 3 \mapsto 1] (4)$
\z $[ \Masha \mapsto 30, \Sveta \mapsto 15, \Masha \mapsto 30] (\Masha)$
\z $[ \Masha \mapsto 30,  \Sveta \mapsto 15, \Masha \mapsto 10 ] (\Masha)$
\z $[1 \mapsto [1 \mapsto 1,~ 2 \mapsto 2,~ 3 \mapsto 3], 2 \mapsto [1 \mapsto 4,~ 2 \mapsto 5,~ 3 \mapsto 6]~]~ (2)~(1)$
\z $\Dom [3 \mapsto 1, 5 \mapsto 0, 2 \mapsto 88]$
\z $\Rng [3 \mapsto 1, 5 \mapsto 0, 2 \mapsto 88]$
\z $\Dom \map{n \mapsto 2 * n}{n : \Nat}$
\z $\Rng \map{n \mapsto 2 * n}{n : \Nat}$
\z $[1 \mapsto 20, 2 \mapsto 30] \Union [1 \mapsto 30, 2 \mapsto 20]$
\z $[1 \mapsto 20, 2 \mapsto 30] \Upd [1 \mapsto 30, 2 \mapsto 20]$
\z $[ \Misha \mapsto 170, \Slava \mapsto 200, \Victor \mapsto 195 ] ~\backslash \\ \{\Misha, \Petr\}$
\z $[ \Misha \mapsto 170, \Slava \mapsto 200, \Victor \mapsto 195 ] ~/ \\\{\Misha, \Petr\}$
\z Пусть Friend = $[\Misha \mapsto \Anna, \Anna \mapsto \Lesha, \Lesha \mapsto \Misha]$. Найти Friend $\Superp$ Friend. Какой смысл этого значения ?
\z $\Card~\Dom~[1 \mapsto 2, 1 \mapsto 3, 2 \mapsto 3]$
\z $\Dom ([10 \mapsto 100, 20 \mapsto 50] \Upd [20 \mapsto 60, 30 \mapsto 90])$
\z $\Card~\Rng~( [1 \mapsto 2] \Superp [3 \mapsto 2, 4 \mapsto 1] \Superp [1 \mapsto 2, 3 \mapsto 4])$

\zhead{Какие из следующих выражений истинны}

Если выражение ложно, привести контпример и дополнительные ограничения на входящие переменные, чтобы условие стало верным. Считать, что все переменные стоят под кванторами всеобщности.

\z $\Dom~ (X \Union Y) \Iden \Dom X ~\Union~ \Dom Y$
\z $\Rng~ (X \Union Y) \Iden \Rng X ~\Union~ \Rng Y$
\z $\Dom~ (X \Upd Y) \Iden \Dom X ~\Upd~ \Dom Y$
\z $\Rng~ (X \Upd Y) \Iden \Rng X ~\Upd~ \Rng Y$
\z $X = Y ~\Impl~ \Dom X = \Dom Y$
\z $X \NEq Y ~\Impl~ \Dom X ~\NEq~ \Dom Y$
\z $X \NEq Y ~\Impl~ \Rng X ~\NEq~ \Rng Y$
\z $\Dom (X \backslash Y) ~\Iden~ (\Dom X) \backslash Y$
\z $\Dom (X / Y) ~\Iden~ Y$
\z $\Dom (X \Upd Y) ~\Iden~ \Dom X \Union Y$
\z $(X \Union Y) \Union Z ~\Iden~ X \Union (Y \Union Z)$
\z $(X \Upd Y) \Upd Z ~\Iden~ X \Upd (Y \Upd Z)$
\z $(X \Superp Y) \Superp Z ~\Iden~ X \Superp (Y \Superp Z)$
\z $X \Union Y ~\Iden~ Y \Union X$

\zhead{Записать на RSL следующие константы и определения типов}
\z Записать отображение -- перестановку первых $N$ натуральных чисел. Считать $N$ константой с sort-определением.
\z Записать отображение-<<сдвиг>> : [1 $\mapsto$ 2, 2 $\mapsto$ 3, ..., $N \mapsto$ 1]. Считать $N$ константой с sort-определением.
\z Записать отображение всех полных квадратов в своё основание. [1 $\mapsto$ 1, 4 $\mapsto$ 2, 9 $\mapsto$ 3, ...]
\z Записать отображение из любого натурального числа в его простой делитель.
\z Записать отображение любого натурального числа в большее его простое натуральное число. [1 $\mapsto$ 5, 2 $\mapsto$ 7, 3 $\mapsto$ 5, ...]
\z Записать отображение <<бесконечная перестановка>>.
\z Записать отображение любого натурального числа в своё <<зеркало>> -- число из цифр исходного числа, записанных в обратном порядке. [1 $\mapsto$ 1, ..., 12 $\mapsto$ 21, ..., 832 $\mapsto$ 238, ...]
\z Записать отображение любого текста в его реверсию. $[..., \mbox{\textrm{"стол"}} \mapsto \mbox{\textrm{"лотс"}}, \mbox{\textrm{"книга"}} \mapsto \mbox{\textrm{"агинк"}}, ...]$
\z Не меняя описания предыдущей константы, описать новое отображение любого текста, начинающегося с 'a', в его реверсию.
\z Записать тип <<Англо-русский словарь>> так, как Вы его представляете. Учтите, что слово может иметь несколько переводов.

\zhead{Запишите спецификацию функций в явном виде}
\z Подсчитать количество элементов в данном отображении, если оно
    \begin{enumerate}
    \item детерминированное
    \item ${^\star}$ недетерминированное
    \end{enumerate}
\z По отображению \{[$a \mapsto b$]\} построить отображение \{[$a^2 \mapsto b^2$]\}.
\z Дано отображение \Nat $\Map$ \Nat. Вернуть количество элементов, отображающих в 0.
\z По данному отображению \{[$a \mapsto b$]\} построить отображение \{[$b \mapsto x$]\}, где $b$ - правая часть некоторого элемента исходного отображения, а $x$ - количество раз, которое $b$ встретилось в исходном отображении.
\z Дано отображение \Nat $\Map$ \Nat. Вернуть количество различных элементов, в которые осуществляется отображение.
\z Дано отображение \Nat $\Map$ \Real, представляющее основание и показатель степени в разложении числа на простые множители. Вернуть число, которое представлено таким отображением.
\z Дано натуральное число. Построить по нему отображение \Nat $\Map$ \Real, представляющее основание и показатель степени в разложении его на простые сомножители.
\z Дано отображение строк \{[$s \mapsto t$]\}. Оставить в нём только те элементы, левая часть которого начинается с $s_0$.
\z Проверить, является ли данное отображение детерминированным.
\z Проверить, является ли данное отображение конечным.
\z Проверить, является ли данное отображение взаимнооднозначным.
\z Проверить, есть ли в данном отображении элемент $x \mapsto y$, где $x$ -- наибольший из всех левых частей, а $y$ -- наименьший среди всех правых частей.
\z Проверить, является ли данное отображение перестановкой.
\z Проверить, является ли данное отображение перестановкой первых $N$ натуральных чисел.
\z Дано отображение \{[$(x, y) \mapsto L$]\} ($L$ -- множество). Проверить, верно ли, что все элементы в $L$ не больше $y$ и не меньше $x$.
\z Дано отображение. Построить его максимальное детерминированное подотображение.
\z Реляционное отношение задано отображением $T \Map A \DP B \DP C$. Проверить, есть ли среди атрибутов $A$, $B$, $C$ возможные ключи. Домены атрибутов считать sort-определенными.
\z Реализуйте функцию $lower$, переводящую символы в нижний регистр. Какова, по Вашему, будет сигнатура этой функции?


\section{Моделирование данных}
TODO

% турнир

% генеалогическое дерево

% кэш-память с ассоциативностями и секциями, тегами и строками

% код хаффмана! отображения + рекурсивные типы данных

%\z Формализуйте определения терминов, приведенных в этом тексте --- они выделены курсивом (он взят из~\cite{structures_algorithms}). \emph{Дерево} --- это совокупность элементов, называемых \emph{узлами} (один из которых определен как \emph{корень}), и отношений (<<родительских>>), образующих иерархическую структуру узлов. Узлы, так же, как и элементы списков, могут быть элементами любого типа. Мы часто будем изображать узлы буквами, строками или числами. Формально дерево можно рекуррентно определить следующим образом.
%\begin{enumerate}
%\item Один узел является деревом. Этот же узел также является корнем этого дерева.
%\item Пусть $n$ --- это узел, а $T_1, T_2, ..., T_k$ --- деревья с корнями $n_1, n_2, ..., n_k$ соответственно. Можно построить новое дерево, сделав $n$ родителем узлов $n_1, n_2, ..., n_k$. В этом дереве $n$ будет корнем, а $T_1, T_2, ..., T_k$ --- \emph{поддеревьями} этого корня. Узлы $n_1, n_2, ..., n_k$ называются \emph{сыновьями} узла $n$.
%\end{enumerate}
%
%Часто в это определение включают понятие \emph{нулевого дерева}, т.е. <<дерева>> без узлов.
%
% http://informatics.mccme.ru/moodle/course/view.php?id=18

0. крестики-нолики.  (координата1 >< координата2 >< отметка в клетке)-set - это первый взгляд, но когда мы будем писать функции, можем понять, что лучше (координата1 >< координата2) -m-> отметка в клетке.

1. SMS. У нее есть текст и номер, кому она отправлена. Кроме того, у SMS есть идентификатор, он присваивается в момент регистрации SMS, чтобы идентифицировать SMS с одинаковым текстом и  и изначально его нет. Кроме того, у SMS есть статус (например, обработана/не обработана). Статус может меняться. Задача состоит в том, чтобы описать термины "SMS" и "база SMS" на RSL (а именно, модель + инварианты).

1) генеалогическое дерево. Дерево должно удовлетворять следующим ограничениям:

    * каждый человек имеет не более двух родителей;
    * родители каждого человека имеют разный пол;
    * каждый человек младше каждого из своих родителей;
    * даты жизни родителей каждого человека пересекаются; * а что если родителей меньше 2х?
    * в корне дерева стоит один человек.
2) военная система. Подчиненные, единственный генерал, у солдат нет подчиненных. воинского подразделения. Иерархия должна удовлетворять следующим ограничениям:
- военнослужащие могут иметь только следующие звания (в порядке возрастания старшинства):
     - рядовой солдат (soldier)
     - лейтенант (lieutenant)
     - капитан (captain)
     - майор (major)
     - полковник (colonel)
     - генерал (general)
- у каждого военнослужащего, за исключением рядовых солдат, есть подчиненные
- у каждого военнослужащего, за исключением генералов, есть командир, причем ровно
один, звание которого выше звания военнослужащего
- все непосредственные подчиненные одного военнослужащего имеют одинаковые звания
- во главе подразделения стоит один военнослужащий. У него нет командиров
***возможно тут сформулированы не все требования - надо выделить их самим.



\section{Программные контракты}

Программный контракт --- это указание свойств результата, а не алгоритма его вычисления.

В оригинале: предусловие --- это


\head{Правильная трактовка предусловий}


\head{Советы по написанию постусловий}

RSL позволяет описывать меньший список через большой. Если f(x) меньше х, то описываем f(x) через перебор элементов х -- если f(x) больше x, то описываем x через перебор элементов f(x) (см. пример - вставка символа при повторе).

При спецификации функций, генерирующих одни множества по другим, полезно специфицировать пересечения и разницы этих множеств (как раз постоянная и меняющаяся часть).

Операторы inds, hd и tl используйте и тогда, когда не помните, с "1" или с "0" считаются индексы в списке. Плюс оператор inds дает дополнительную семантику переменной - что это индекс!

Композиция -- это способ построить выборку по двум критериям.

Не бояться использовать бесконечные сущности в спецификации.

Есть предположение, что если для структуры нужно делать поиск и добавление, то для ее моделирования лучше всего использовать отображение. Критерий поиска становится ключом, а добавляемые элементы - значениями.

\head{Схемы сокращения записи}

Следующие схемы рекомендуется применять в случае требований выполнения/невыполнения нескольких условий одновременно. Справа в схеме

\paragraph{схема пересечения}

$$x~\Inter~ \{f(t) ~|~ t:T\}$$ вместо $$\{a ~|~ a: X ~\SuchAs~ a \Isin x ~\And~ ( \Exists t: T  ~\SuchAs~ x = f(t) ) \}$$

\paragraph{схема вычитания-1}

$$x ~\Minus~ \{f(t) ~|~ t:T\}$$ вместо $$\{a ~|~ a: X ~\SuchAs~ a \Isin x ~\And~  \Not( \Exists t: T  ~\SuchAs~ x = f(t) ) \}$$

\paragraph{схема вычитания-2}

$$\{f(t) ~|~ t:T\} ~\Minus~ x$$ вместо $$\{a ~|~ a: X ~\SuchAs~ a ~\Isin x ~\And~ ( \Exists t: T  ~\SuchAs~ x = f(t) ) \}$$

\paragraph{схема выборки по ключам (на примере)}
Из исходного отображения выделить подотображение с определенными ключами:
$$t / \{1, 2\}$$ вместо $$[a \mapsto t(a) ~|~ a:X ~\SuchAs~ a \Isin \Dom x ~\And~ a \Isin \{1, 2\} ]$$

\paragraph{схема выборки по значению}
Из исходного отображения выделить подотображение с определенными значениями:
$$[1 \mapsto 1,~ 2 \mapsto 2] \Superp t$$ вместо $$[a \mapsto t(a) ~|~ a:X ~\SuchAs~ a \Isin \Dom x ~\And~ t(a) \Isin \{1,2\} ]$$

%(( all x:X :- ~exists y1, y2: X-set :- y1 ~= y2 /\ x isin y1 inter y2 )) можно записать короче:
%(( ~exists y1, y2: X-set :- y1 ~= y2 /\ y1 inter y2 ~= {} ))
%
%all x:X :- ~exists y:X-set :- x isin y ::::::: ~exists y:X-set :- y ~={} ::::::: all y:X-set :- y = {}

Простая задачка: выстроить в каком угодно порядке элементы множества в список (каждый элемент в списке будет повторяться
  * а) любое число раз
  * б) всего 1 раз
value
    set2list: T-set -> T-list
    set2list( s ) as l
    post
    a)    elems l = s             -- недоспецификация (неизвестен порядок и количество)
    б)    elems l = s /\ len l = card s    -- тоже недоспецификация (неизвестен порядок)

\head{Советы по получению понятных спецификаций}

\begin{itemize}
    \item Выбирать <<говорящие>> имена для параметров, функций, имен переменных и типов.
    \item Строить дополнительные функции, позволяющие <<читать>> выражения, в которых они встречаются.
    \item Выбирать одну из наиболее коротких форм записи выражений.
\end{itemize}

Пример: специфицируется система <<Электронное школьное расписание>>, в ней есть операция добавления нового класса (класс --- это имя и некая дополнительная информация); надо проверить, что имя класса еще не встречалось. Неудачный вариант спецификации этой проверки:
\begin{lstlisting}
check: Class-set >< ClassName >< ClassInfo -~-> Class-set
check(cset, cn, ci) is cset union {(cn,ci)}
pre (all c: Class :- c isin cset =>  let (cn2, ci2) = c in cn ~= cn2 )
\end{lstlisting}

Тут есть две проблемы:
\begin{enumerate}
    \item нечитабельные имена;
    \item длинная запись условия в предикате.
\end{enumerate}

Улучшенная версия:
\begin{lstlisting}
check: Class-set >< ClassName >< ClassInfo -~-> Class-set
check(classes, classname, classinfo) is
    classes union {(classname,classinfo)}
pre classname ~isin names(classes),

names: Class-set -> ClassName-set
names(classes) is
    { name(class) | class : Class :- class isin classes },

name: Class -> ClassName
name((classname,_)) is classname
\end{lstlisting}

Выбраны <<говорящие>> имена для параметров и добавлены 2 новые функции, с помощью которых предусловие стало возможным <<читать>>.

Еще пример для того же <<электронного расписания>>: пусть есть тип <<База данных>>, в нем несколько согласованных частей (одна хранит классы учащихся, другая учителей, третья аудитории); специфицируется функция, вносящая изменение в одну из частей (добавление нового класса учащихся). Неудачный вариант спецификации этой функции:
\begin{lstlisting}
add_class: DB >< Class -> DB
add_class(db, class) is
   let (classes, teachers, rooms, ....... ) = db in
        (classes union {class}, teachers, rooms, .... )
   end
\end{lstlisting}

текста получается много, взгляд разбегается, не знает, за что ухватиться, самое главное не высвечено.

Удачный вариант спецификации этой функции:
\begin{lstlisting}
add_class: DB >< Class -> DB
add_class(db, class) as db2
post
    classes(db2) = classes(db) union {class} /\
    teachers(db2) = teachers(db) /\
    rooms(db2) = rooms(db) /\ ....
\end{lstlisting}

Здесь постусловие (вместо явного задания) и выбор дополнительных <<говорящих>> функций позволяет выделить важное и кратко их записать.



\section*{Задачи}

\zhead{Написать неявную спецификацию следующих функций}

\zhead{Написать спецификацию функции без использования рекурсии}

\z Вычислить максимум множества натуральных чисел

\textbf{Решение:}
\begin{lstlisting}
type T = Nat
value max: T-set -~-> T
max(ns) as n
post  n isin ns /\ (all x: T :- x isin ns => x <= n)
pre   ns ~= {}
\end{lstlisting}
а теперь более короткое решение:
\begin{lstlisting}
value max:T-set -~-> T
max(ns) as n
post n isin ns /\ {0..n} >>= ns
pre ns ~= {}
\end{lstlisting}

\z Вычислить минимум множества натуральных чисел
%\begin{lstlisting}
%value min:T-set -~-> T
%min(ns) as n
%post n isin ns /\ {0..n-1} inter ns = {}
%pre ns ~= {}
%\end{lstlisting}

\z <<особые фамилии>>

Отобрать фамилии из данного списка, заканчивающиеся на <<ов>>.

%value f : Text-set -> Text-set
%длинное решение:
%f(x) is {y | y: Text :- y isin x /\ (exists z : Text :- y = z ^ "ов") }
%короткое решение (применена "схема пересечения"):
%f(x) is x inter {z ^ "ов" | z: Text}
%
%1. сортировка (++)
%
%type T = Real
%value sort: T-list -> T-list
%sort(x) as y
%post
%(all t:T :- <.tt|tt in x :- tt = t.> = <.tt|tt in y :- tt=t.>) /\
%(all i,j : Nat :- i <= j  /\ {i,j} << inds y =>  y(i) <= y(j) )
%
%вариант первой строчки через отображения:
%( exists m: T-m->T :- dom m = rng m /\ dom m = inds x /\ y(i) = m(x(i)) ) /\
%
%2. все подстроки (+)
%
%value allsubstr: Text -> Text-set
%allsubstr(full) is
%{ sub | sub : Text :- exists i : Nat :- i isin inds sub /\ (all k: Nat :- k < len sub => full(i+k) = sub(k+1)) }
%запись внутреннего квантора короче:
%{ s | s: Text :- exists i:Nat :- i < len full - len s /\  s = <.full(k)|k in <.i+1..i+1+len s.>.> }
%или так:
%{ s | s: Text :- exists i:Nat :- s is <.full(k)|k in <.i..i+len s.>.> }
%более короткое решение:
%{<.full(i)|i in <.1..len full.> :- b(i).> | b : Bool-list :- len b = len full }
%
%одна подстрока:
%value substr: Text >< Text -> Bool
%substr(sub, full) is
%(exists i:Nat :- sub is <.full(k)|k in <.i..i+len sub.>.>)
%
%
%[задача о рюкзаке]
%
%дано множество предметов, информация о них (вес (и стоимость)), ограничение на вес рюкзака
%построить множество предметов из данного множества, дающих максимум стоимости
%или: построить множество множеств предметов (распределение) так, чтобы минимизировать количество рюкзаков
%// ничего сложного, просто много писанины
%
%value  rucksack: (T-set) >< (T-m-> Nat) >< Nat -> T-set-set
%rucksack( objs, ws, w ) as rs
%post
%rs = {
%
%
%3. объединение классов (задача взята из Software Engineering I, Dines Bjorner)  (++)
%
%type A, P = (A-set)-set
%value mergeC: P >< A >< A -~-> P
%mergeC(p1, a1, a2) as p2
%post
%(all s: A-set :- s isin p1 /\ s inter {a1, a2} = {} => s isin p2) /\  ////вариант: p1 inter p2 = { s | s : A-set :- s isin p1 /\ s inter {a1, a2} = {} } /\
%card p2 = card p1 - 1 /\ (all s1, s2: A-set :- {s1, s2} <<= p1 /\ a1 isin s1 /\ a2 isin s2 => (s1 union s2) isin p2)
%короче:
%(exists s1, s2: A-set :- p1\p2 = {s1, s2} /\ s1 ~= s2 /\ a1 isin s1 /\ a2 isin s2 /\ p2\p1 = {s1 union s2})
%pre
%(exists s1,s2: A-set :- {s1,s2} <<= p1 /\ s1 ~= s2 /\ a1 isin s1 /\ a2 isin s2)
%
%4.  разметка текста-I (+)
%
%если встречается буква 'a', заменить ее на 'b'
%
%value replace1: Text >< Char >< Char -> Text
%replace1(x, a, b) as y
%post
%len x = len y /\
%(all i: Nat :- i isin inds x => y(i) = b /\ x(i) = a \/ y(i) = x(i) /\ x(i) ~= a)
%
%5. разметка текста-II
%
%если встречается слово "abc", заменить его на "def"
%
%type TextTuplet = {| tl:Text-list :- len(tl) = 2 |}
%value
%  replace2 : Text >< Text >< Text -> Text
%  replace2(txt, str, res) as newtxt
%  post
%      if txt = "" then newtxt = ""
%      else
%         (exists lst : TextTuplet-list :-
%            hd lst = <. txt, "" .> /\
%   lst(len lst) = <."", newtxt.> /\
%            (all i : Nat :- {i,i+1} << inds lst =>
%if (exists t:Text :- lst(i)(1) = str ^ t)
%then (exists t:Text :- lst(i)(1) = str ^ t /\ lst(i+1)(1) = t /\ lst(i+1)(2) = lst(i)(2) ^ res)
%else lst(i+1)(1) = tl lst(i)(1) /\ lst(i+1)(2) = lst(i)(2) ^ <.hd lst(i)(1) .>
%end
%   )
%         )
%      end
%
%
%
%6. слияние (+)
%
%даны два упорядоченных списка чисел, построить упорядоченный список из объединения этих чисел
%
%type T = Real
%value merge: T-list >< T-list -~-> T-list
%merge(x,y) as z
%post
%(all i,j:Nat :- {i,j} <<= inds z => (i <= j => z(i) <= z(j))) /\
%(all t: T :- <.tt|tt in x :- tt = t.> ^ <.tt|tt in y :- tt = t.>  = <.tt|tt in z :- tt = t.> )
%pre
%(all i,j:Nat :- {i,j} << inds x => (i <= j => x(i) <= x(j))) /\
%(all i,j:Nat :- {i,j} << inds y => (i <= j => y(i) <= y(j)))
%
%7. выделение слов из текста (++)
%
%разделитель задается в качестве параметра
%
%value parse: Text >< Char -> Text-set
%parse(t,d) is { w | w : Text :-
%<.d.> ^ w ^ <.d.> isin allsubstr(<.d.> ^ t ^ <.d.>) /\ d ~isin elems w }
%
%8. факториал без рекурсии (++)
%
%value f: Nat -> Nat
%f(x) as y
%post
%(exists n : Nat-list :- len n = x /\ hd n = 1 /\
%(all i : Nat :- i < len n - 1 => n(i+2) = n(i+1)*(i+2)) /\ y = n(len n))
%
%9? количество раз, которое подстрока встречается в другой строке
%
%10 инъектирование (+)
%
%после каждого вхождения в строку заданного символа вставить другой символ
%
%value inject: Text >< Char >< Char -> Text
%inject(t, c1, c2) as t2
%post
%if c1 ~= c2 then t = <. t2(i) | i in <.1 .. len t2.> :- ~(t2(i-1) = c1 /\ t2(i) = c2) .>  ////не работает при c1 = c2
%else t = <. t2(i) | i in <.1 .. len t2.> :- t2(i) ~= c1 \/ (exists ll:Nat :- len <.t2(j)|j in <. i-ll+1 .. i .>.> = 1 /\ t2(i-ll) ~= c1 /\ ll \ 2 = 1 .>) end
%
%
%11 представители (+)
%
%для каждого символа в строке оставить только первое его вхождение
%
%value m: Text -> Text
%m(t) as t2
%post
%/////////elems t = elems t2 /\ card elems t2 = len t2 /\ сохраняется порядок первых вхождений ??????????
%t2 = <. t(i) | i in <.1 .. len t.> :- t(i) ~isin {t(j)|j : Nat :- j isin {1..i-1}} .>
%
%12. Set Intersection—Returns a new array containing elements common to the two arrays, with no duplicates.
%[ 1, 1, 3, 5 ] & [ 1, 2, 3 ] #=> [ 1, 3 ]
%value intersection : T-list >< T-list -> T-list
%intersection(a, b) as c
%post elems c = elems a inter elems b /\ len c = card elems c
%
%13-14. Repetition—With a String argument, equivalent to self.join(str). Otherwise, returns a new array built by concatenating the int copies of self.
%int_repetion: [ 1, 2, 3 ] * 3 #=> [ 1, 2, 3, 1, 2, 3, 1, 2, 3 ]
%str_repetion: [ 1, 2, 3 ] * "," #=> "1,2,3"
%value int_repetion: T-list >< Nat -> T-list
%int_repetion( t, c ) as t2
%post len t2 = len t * c /\ all m : Nat :- m isin inds t  => { t2[n*len t + m] | n : Nat :- n < c } = { t2[m] }
%
%value str: T -> Text,
%str_repetion: T-list >< Text -> Text
%str_repetion( t, c ) as t2
%post  ????
%
%http://ruby-doc.org/core/classes/Array.html
%
%**** Проверить, что заданная строчка является правильным скобочным выражением.
%value parmatch: Text -> Bool
%parmatch(t) is elems t <<= elems "()" /\ ( t = "" \/ (exists t1, t2 : Text :- t = "("^t1^")"^t2 /\ parmatch(t1) /\ parmatch(t2)))
%
%
%**) получение строкового представления числа
%пишется именно что спецификация, а не реализация! т.е. как проверить данную строку. Плохой симптом, если постусловие пишется в стиле "ответ равен результату работы такого-то другого алгоритма". Иными словами, хотели проверить один алгоритм, а вместо этого пишем другой - и везде можем совершить ошибку. Вместо этого надо заниматься не вычислением, а определением, т.е. выделением набора характеристик и их сравнением.
%
%value
%    digit : Char -m-> Int = ['0'+> 0, ...., '9' +> 9 ],
%    itoa: Int -> Text
%    itoa(x) as s
%    post
%        elems s <<= elems "-0123456789" /\ s ~= "" /\
%        "-" ~isin tl s /\  ( x < 0  is  hd s = "-" ) /\
%        let s1 = if x < 0 then tl s else s end, s2 = reverse(s1) in
%            all i : Nat :- i isin inds s2 => digit(s2(i)) = (abs x \ 10**i) / 10**(i-1)
%        end
%
%обратная задача: вернуть целое число из строки
%    atoi : Text -~-> Int
%    atoi(s) as x
%    post
%        ( x < 0  is  hd s = "-" ) /\
%        let s1 = if x < 0 then tl s else s end, s2 = reverse(s1) in
%            all i : Nat :- i isin inds s2 => digit(s2(i)) = (abs x \ 10**i) / 10**(i-1)
%        end
%    pre
%        elems s <<= elems "-0123456789" /\ s ~= "" /\ "-" ~isin tl s
%
%=======================================================
%
%граф задан отображением V -m-> V-set
%
%задействованы операторы: \ dom rng () # /
%не задействованы (1): !!
%
%12 поиск пути в графе (++)
%
%type G = {| g : V -m-> V-set :- all v : V :- v isin dom g => g(v) <<= dom g |}
%type G = {| g : V -m-> V-set :- union rng g <<= dom g |} ///более короткая запись, но rsltc не поддерживает префиксный union:(
%
%value path : G >< V >< V -> V-list
%path(g, v1, v2) as p
%post
%elems p <<= dom g /\ hd p = v1 /\ p(len p) = v2 /\
%(all i:Nat :- i < len p - 1 => p(i+2) isin g(p(i+1)))
%/// вариант последней строки: (all i: Nat :- {i, i+1} <<= inds p => p(i+1) isin g(p(i)))
%
%12a. построение гамильтонова пути (+)
%
%value gpath : G-> V-list
%path(g) as p
%post
%elems p = dom g /\ len p = card elems p /\
%(all i:Nat :- i < len p - 1 => p(i+2) isin g(p(i+1)))
%
%13. проверка, что граф является деревом (т.е. без циклов с одним корнем)
%
%value istree: G -> Bool
%istree(g) is
%// корень - это вершина, на которую никто не указывает
%card { v | v : V :- v isin dom g =>  ~exists x: V :- x isin dom g /\ v isin g(x) } = 1 /\ ?????????? [[ без циклов ]]
%короче:
%card { v | v : V :- v isin dom g =>  [vs+>0|vs: V-set :- v isin vs] # g = [] } = 1 /\ ?????????? [[ без циклов ]]
%еще короче:
%card (dom g \ {v|v : V :- exists vs : V-set :- vs isin rng g /\ v isin vs }) = 1 /\
%
%~exists vs: V-list :- ( elems vs << dom g /\ len vs > 1 /\ vs(1) = vs(len vs) /\
%all i:Nat :- i < len vs - 1 => vs(i+2) isin g(vs(i+1)) )                                                           ////переписать отсутствие цикла как-то более просто
%//// ацикличность: недостижимость любой вершины из самой себя
%
%/// надо еще добавить, что граф должен быть связным! (с одной компонентой связности)
%
%14. ограниченная достижимость (++)
%
%построить множество вершин, достижимых из данной вершины с не более N промежуточными вершинами
%value  med: G >< V >< Nat -~-> V-set
%med(g, v, n) as vs
%post
%(exists r: (V-m->V-set)-list :- len r = n /\ hd r = g /\ (all i :- Nat :- i < n-1 => r(i+2) = g#r(i+1)) /\ vs =  union elems <. e(v) | e in r.>)
%другой вариант на основе префиксного union (получилось короче?):
%(exists r: (V-set)-list :- len r = n /\ hd r = g(v) /\ (all i: Nat :- i < n-1 => r(i+2) = union rng g/r(i+1)) /\ vs = union elems r)
%или без композиции, но это более длинная спецификация:
%(all a: V-set :- a isin dom g => (a isin vs = ( exists path: V-list :- elems path <<= dom g /\ len path <= n /\ hd path =  v /\ path(len path) = a /\ (all i:Nat :- {i,i+1} <<= inds path => path(i+1) isin g(path(i)) )) ))
%
%частный случай для n=1:
%post  vs = g(v) union (g#g)(v)
%или vs = g(v) union union rng (g/g(v))
%или (переписан префиксный union) vs = g(v) union { v1| v1 : V :- v1 isin rng (g/g(v))}
%pre v isin dom g
%
%15. добавление элемента в дерево поиска
%
%16. инвертирование дуг графа


\zhead{По данной явной спецификации построить эквивалентную неявную спецификацию}

\z \begin{lstlisting}
variable lst: Int-list
value f: Unit -> write lst Int-list
  f() is local variable lst2:Int-list := <..> in
    for e in lst do
      if e ~isin elems lst2 then lst2 := lst2 ^ <.e.> end
    end; lst2  end
\end{lstlisting}

\textbf{Решение:} Функция возвращает список первых вхождений в обратном порядке. Идея записи постусловия: для каждого элемента списка-результата:
\begin{itemize}
    \item все предыдущие элементы не должны встречаться перед первым его вхождением;
    \item все последующие элементы должны встретиться перед первым его вхождением.
\end{itemize}

\begin{lstlisting}
variable lst: Int-list
value f: Unit -> write lst Int-list
f() as lst2
post lst = lst` /\
  let lst2 = card elems lst /\ elems lst2 = elems lst /\
  ( all i:Nat :- i isin inds lst2 =>
    ( exists j: Nat :- j isin inds lst /\ lst(j) = lst2(i) /\
               lst(j) ~isin { lst(k) | k:Nat :- k isin {1..j-1} } /\
       (all p:Nat :- p isin inds lst2 /\ p < i =>
         lst2(p) ~isin { lst(k) | k:Nat :- k isin {1..i-1} } ) /\
       (all p:Nat :- p isin inds lst2 /\ p > i =>
         lst2(p)  isin { lst(k) | k:Nat :- k isin {1..i-1} } ) ) )
\end{lstlisting}

\z \begin{lstlisting}
variable lst: Int-list
value f: Unit -> write lst Int-list
   f() is
    local variable lst2:Int-list := <..>, lst3: Int-list := lst in
    for e in lst do
        if lst3 ~= <..> then
           lst3 := tl lst3;
           if e ~isin elems lst3 then lst2 := lst2 ^ <.e.> end
       end
   end; lst2  end
\end{lstlisting}

\z \begin{lstlisting}
value f: Int-list -> Nat
  f(lst) is local variable m:Nat := 0,
         m2:Nat := 0, b:Bool := true in
    for i in <.1 .. len lst.> do
        m := 1; b := true;
        for j in <.i+1 .. len lst.> do
          if b /\ lst(j) >= lst(j-1)
            then m := m + 1
            else b := false end
          end;
          if m2 < m then m2 := m end
       end; m2 end
\end{lstlisting}

\z \begin{lstlisting}
value f: Nat-list -> Nat
  f(lst) is local variable lst2:Nat-list := lst,
             k:Nat := 0, m:Nat := 0 in
     for e in lst do
        if e > k then m := m + 1; k := e end;
    end; m end
\end{lstlisting}

\zhead{По данной алгебраической спецификации предложить модели типов и явные или неявные определения функций}
Выделить наислабейшие предусловия для функций.

% спецификация функций стандартной библиотеки Си: http://www.open-std.org/JTC1/SC22/wg14/www/docs/n1124.pdf

% спецификация протоколов (формат сообщений и правила преобразования сообщений)

% описание недетерминированности в детерминированных предусловиях (разное поведение, мы еще не выбрали из них окончательное)
% --- как в играх ("крестики-нолики")


%%\chapter{Поведенческие спецификации}
%%TBD
%
%\chapter{Специфицирование систем реального размера}
%Спецификация для функции append в <<исполнимой>> манере:
%\begin{lstlisting}
%type E, L = E*
%value append: L >< L -> L
%    append(x, y) is x ^ y
%\end{lstlisting}
%
%Спецификация для функции append в <<логической>> манере:
%\begin{lstlisting}
%type E, L = E*
%value append: L >< L -> L
%    append(x, y) as z
%    post
%        len z = len x + len y /\
%        (all i: Nat :- i isin inds x => z(i) = x(i)) /\
%        (all i: Nat :- i isin inds y => z(i+len x) = y(i))
%\end{lstlisting}
%
%Спецификация для функции reverse в <<исполнимой>> манере:
%\begin{lstlisting}
%type E, L = E*
%value reverse: L -> L
%    reverse(x) is
%        local variable y : L :- y = <..> in
%            for xi in x do
%                y := <.xi.> ^ y
%            end;
%            y;
%        end
%\end{lstlisting}
%
%Спецификация для функции reverse в <<логической>> манере:
%\begin{lstlisting}
%type E, L = E*
%value reverse: L -> L
%  reverse(x) as y
%  post len x = len y /\
%    (all i: Nat :- i isin inds x => y(i) = x(len x + 1 - i))
%\end{lstlisting}
%
%Спецификация для функции reverse в <<алгебраической>> манере:
%\begin{lstlisting}
%type E, L = E*
%value reverse: L -> L
%axiom
%    reverse(<..>) is <..>,
%    all x, y: L :- reverse(x^y) is reverse(y) ^ reverse(x)
%\end{lstlisting}
%
%Что из этого короче? Что понятнее? Что лучше? (зависит от задачи)

% задача - выбрать и правильно смоделировать самое главное в алгоритме:
% gzip, протокол MESI, map/reduce, mp3, google chrome... (в зависимости от варианта задания--посмотреть,что интересует молодежь на Хабре) -- так, чтобы другой человек мог сам понять и получить это новое знание. В качестве примера, можно рассмотреть то, что я писал в диссертации - поймут студенты?
% DOM:
%http://www.w3.org/TR/DOM-Level-3-Core/core.html#ID-184E7107–???DOM Core 3
%http://ru.wikipedia.org/wiki/Document_Object_Model-

\appendix
% !Mode:: "TeX:UTF-8"
\chapter{Приоритет операций в RSL}\label{sec:priority}

Операции с меньшим приоритетом обрабатываются раньше.

%\begin{table}[h]
%\caption{
Приоритет операций в типовых выражениях
%}
\begin{center}
\begin{tabular}{|c|c|c|}
\hline приоритет & операторы & ассоциативность\\
\hline  4 & , & правая\\
\hline  3 & $\Map \NonDeterMap$ & правая\\
\hline  2 & $\DP$ & \\
\hline  1 & ${\ }\Set {\ }\Infset {\ }\List {\ }\Inflist$ & \\
\hline
\end{tabular}
\end{center}
%\end{table}

%\begin{table}[h]
%\caption{
Приоритет операций в выражениях со значением
%}
\begin{center}
\begin{tabular}{|c|c|c|}
\hline приоритет & операторы & ассоциативность\\
\hline 13 & $\All \Exists \ExistsOne$ & правая \\
\hline 12 & $\Iden$ & \\
\hline 11 & $\Exc \Nd || \Ilock$ & правая \\
\hline 10 & ; & правая \\
\hline 9 & := & \\
\hline 8 & $\Impl$ & правая \\
\hline 7 & $\Or$ & правая \\
\hline 6 & $\And$ & правая \\
\hline 5 & $= \neq > < \geqslant\leqslant \SetL \SetG \SetLE \SetGE
\Isin \NotIsin$ &  \\
\hline 4 & $+ - \Minus \Concat \Union \Upd$ & левая \\
\hline 3 & $* / \Superp \Inter$ & левая \\
\hline 2 & $\Power$ & \\
\hline 1 & $\Not$ префиксы суффиксы& \\
\hline
\end{tabular}
\end{center}
%\end{table} 

\pagebreak
\addcontentsline{toc}{chapter}{Литература}
\bibliographystyle{gost780s}
\bibliography{rslbooks}

\end{document}
